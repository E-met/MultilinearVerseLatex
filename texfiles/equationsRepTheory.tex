\section{$G$-varieties and equations}
 \label{repTheory-section-Gvarieties}
% Author: Fulvio Gesmundo
In a number of settings, algebraic varieties of interest are invariant under the action of an algebraic group. In these cases, representation theory, and in particular \ref{introduction-lemma-Schur} can be of great help in the study of equations for the variety of interest.

\begin{definition}
 \label{repTheory-definition-Gvariety}
% Author: Fulvio Gesmundo
 Let $G$ be a group and let $V$ be a $G$-representation. We say that a variety $X \subseteq \bbP V$ is a $G$-variety if $X$ is invariant under the action of $G$, that is $G \cdot X = X$. 
\end{definition}
We point out that the points of a $G$-variety $X$ are not in general fixed by the action of $G$. If this is the case, we say that $X$ is point-wise invariant by the action of $G$.

The Segre-Veronese varieties $\nu_{d_1 \vvirg d_k} ( \bbP V_1 \ttimes \bbP V_k)$ of \ref{introduction-definition-SegreVeronese}, their secant varieties (see \ref{geometrySecants-chapter-BasicDefinitions}), tangential varieties, joins of such, and more generally varieties constructed \emph{functorially} from them are $G$-varieties, with $G = \GL(V_1) \ttimes \GL(V_d)$. Orbit-closures for the action of a group $G$, in the sense of \ref{introduction-definition-orbitsdegenerations}, are $G$-variaties as well.

The action of a group $G$ on a space $V$ induces an action of the polynomial ring $\bbC[V]$ via pull-back: given $F \in \bbC[V]$, $g \in G$, one has 
\[
g \cdot F = F \circ g^{-1}.
\]
If the action is linear, it restricts to the homogeneous components of $\bbC[V] = \bigoplus_{d \geq 0} S^d V^*$. If $X$ is a $G$-variety, then the homogeneous components of its ideal are $G$-representations as well.
\begin{lemma}
\label{repTheory-lemma-GactionOnIdeal}
% Author: Fulvio Gesmundo
 Let $X \subseteq \bbP V$ be a $G$-variety. Let $I(X) \subseteq \bbC[V]$ be the ideal of $X$. Then for every $g \in G$ and every $F \in \bbC[V]$, we have that $F \in I(X)$ if and only if $g \cdot F \in I(X)$. In particular, the homogeneous components $I(X)_d \subseteq  S^d V^*$ are subrepresentations of $G$.
\end{lemma}
Because of this result, every equation $F$ of a $G$-variety yields a \emph{module} of equations, that is the smallest $G$-subrepresentation containing $F$. The term equation is often used loosely, in the general sense of \emph{Zariski closed condition}.

In fact, it is often possible to realize the homogeneous components of the ideals of a $G$-variety as kernels (or images) of $G$-equivariant maps defined naturally.  A typical example is the one of equations arising from flattenings, discussed in \ref{RepTheory-chapter-flattenings}, which are particularly effective to find equations of secant varieties. We discuss here two other examples:
\begin{itemize}
 \item Kostant's Theorem, which determines equations for rational homogeneous varieties in purely representation theoretic terms;
 \item the Foulkes map, which appears in algebraic combinatorics and whose kernel describes the ideal of the Chow variety of completely reducible forms.
\end{itemize}

\section{Kostant's Theorem}
\label{RepTheory-section-kostant}
% Author: Fulvio Gesmundo
We say that a $G$-variety $X$ is \emph{homogeneous} for the action of $G$ if the group $G$ acts transitively on $X$. Important examples of homogeneous varieties include the varieties of rank one elements introduced in \ref{introduction-section-decomposable_tensors}. In fact, those varieties are examples of rational homogeneous varieties.
\begin{definition}
 \label{reptheory-definition-rationalhomogeneousvariety}
 Let $X \subseteq \bbP V$ be a $G$-variety, with $V$ irreducible representation of a semisimple algebraic group $G$. We say that $X$ is a {\it rational homogeneous variety} if $X$ is the $G$-orbit of an element $p \in \bbP V$.
\end{definition}
A rational homogeneous variety $X \subseteq \bbP V$ is uniquely characterized by the highest weight $\lambda$ of the $G$-representation $V$. Indeed, if $v \in V$ is a highest weight vector, then it is easy to see $X = G \cdot [v]$. 

\begin{example}
 \label{reptheory-example-RHV}
% Author: Fulvio Gesmundo
 The Segre-Veronese variety $X = \nu_{d_1 \vvirg d_k} (\bbP V_1 \ttimes \bbP V_k) \subseteq \bbP (S^{d_1} V_1 \ootimes S^{d_k}V_k)$ is a rational homogeneous variety under the action of $G = \GL(V_1) \ttimes \GL(V_k)$. Indeed, up to the choice of a torus in $\GL(V_i)$ any vector $v_i \in V_i$ can be regarded as a highest weight vector: then $X = G \cdot (v_1^{d_1} \ootimes v_k^{d_k})$. 
\end{example}

By \ref{repTheory-lemma-GactionOnIdeal}, the ideal of a rational homogeneous variety is a $G$-representation $I(X) \subseteq \Sym( V^*)$. Kostant's Theorem characterizes such ideal:
\begin{theorem}
 \label{reptheory-theorem-kostant}
% Author: Fulvio Gesmundo
Let $G$ be a semisimple algebraic group and let $V_\lambda$ be the irreducible $G$-representation of highest weight $\lambda$. Let $X \subseteq \bbP V_\lambda$ be the rational homogeneous variety in $\bbP V_\lambda$. Then, for every $d \geq 1$, 
\[
I(X)_d = V_{d\lambda}^\perp \subseteq S^d V_\lambda^*.
\]
Moreover, $I(X)_2$ generates the ideal $I(X)$.
\end{theorem}
We refer to \cite[Ch.16]{Lan12}

\section{The Foulkes map and the ideal of the Chow variety}
\label{RepTheory-section-foulkes}
% Author: Fulvio Gesmundo

Let $V$ be a vector space. The Chow variety of completely reducible forms of degree $d$ is 
\[
\Ch^{d}(V) = \{ \ell_1 \cdots \ell_d : \ell_i \in V \} \subseteq \bbP S^d V.
\]
It is not hard to see that $\Ch^d(V)$ is an Zariski-closed. It is evidently closed under the action of $\GL(V)$, hence it is a $\GL(V)$-variety. If $d \leq \dim V -1$, then 
\[
\Ch^d(V) = \overline{\GL(V) \cdot (x_1 \cdots x_d)}
\]
where $x_1 \vvirg x_d$ are $d$ vectors of $V$ in general linear position; in particular if $d\leq \dim V$ they are linearly independent. The Chow variety is the image of the Segre variety $\Seg(\bbP V \ttimes \bbP V) \subseteq \bbP V^{\otimes d}$ under the projection onto the fully symmetric component $S^d V$.

Since $\Ch^d(V)$ is a $\GL(V)$-variety, by \ref{repTheory-lemma-GactionOnIdeal}, the homogeneous components $I(\Ch^d(V))_e \subseteq S^e S^d V^*$ of the ideal of the Chow variety are subrepresentations of $ S^e S^d V^*$. These representations are hard to understand and they are related to long standing problems in representation theory and combinatorics. In particular, the components $I(\Ch^d(V))_e$ can be described in terms of the kernel of a natural map:
\begin{definition}
 \label{RepTheory-definition-foulkesmap}
% Author: Fulvio Gesmundo
 Let $d,e$ be positive integers. Index the tensor factors of $V^{\otimes de}$ with pairs $(i,j) \in [d]\times [e]$. Let $\pi_{d,e}$ be the composition
 \[
\bigotimes_{(i,j)} V_{(i,j)} \to (S^e V_{(1,\ast)}) \ootimes (S^eV_{(d,\ast)}) \to S^d S^e V  
 \]
 of the $\frakS_e$-symmetrization on every set of $e$ spaces with common first index, followed by the symmetrization on the $d$ resulting compies of $S^e V$. Let $\pi_{e,d}$ be the analogous composition obtained by reversing the two symmetrizations and write $S^eS^d V$ for its image. The Foulkes-Howe map is the composition
 \[
 \rmFH_{d,e} : S^e S^d V \to V^{\otimes de} \xto{\pi_{d,e}} S^d S^e V.
 \]
 of the embedding of $S^e S^d V = \image(\pi_{e,d})$ into $V^{\otimes de}$ followed by the projection $\pi_{d,e}$.
 \end{definition}
This map was introduced by Hermite in the case $\dim V = 2$ \cite{Her56}, and it was observed that in that case it is an isomorphism; this result is called Hermite reciprocity today. Hadamard studied it in general and proved that it completely controls the ideal of the Chow varieties \cite{Had97}.
\begin{theorem}
 \label{RepTheory-theorem-foulkeskerChow}
% Author: Fulvio Gesmundo
% Contributor: Christian Ikenmeyer
Let $V$ be a vector space. Let $d,e$ be nonnegative integers. Then 
\[
I(\Ch^d (V)) _ e = \ker \rmFH_{d,e}.
\]
\end{theorem}
\begin{proof}
 We refer to \cite[Prop. 8.6.1.2]{Lan12}.
\end{proof}


In particular, one recovers Hermite's result:
\begin{corollary}[Hermite reciprocity]
\label{RepTheory-corollary-hermiteReciprocity}
% Author: Fulvio Gesmundo
If $\dim V = 2$, then the Foulkes-Howe map $\rmFH_{d,e}$ is an isomorphism.
\end{corollary}
\begin{proof}
By the Fundamental Theorem of Algebra, every binary form splits as product of linear forms. Therefore $\Ch^d (V) = \bbP S^d V$ if $\dim V = 2$, and $I(\Ch^d(V))_e = 0$ for every $e$. By \ref{RepTheory-theorem-foulkeskerChow}, we deduce that $\rmFH_{d,e}$ is injective for every $d,e$. Since domain and codomain have the same dimension, we conclude that it is an isomorphism.
\end{proof}

Hermite reciprocity for arbitrary fields is explained in \cite[Sec.~3.4]{AFPRW19} and \cite{RS21,MW22}.


Another immediate consequence of \ref{RepTheory-theorem-foulkeskerChow} is the following:
\begin{corollary}
\label{RepTheory-corollary-foulkesplethysmbound}
% Author: Fulvio Gesmundo
Let $V$ be a vector space with $\dim V =n$. Let $d,e$ be nonnegative integers and let $\lambda$ be a partition of $ed$ with $\ell(\lambda) \leq n$. If $a_\lambda(e,d) > a_\lambda(d,e)$ then $I(\Ch_d(V))_e \neq 0$.
\end{corollary}
A special case of \ref{RepTheory-corollary-foulkesplethysmbound} is the one where $\ell(\lambda) > d$. In this case, a consequence of Pieri's rule (see \ref{introduction-proposition-Pieri}) is that $a_\lambda(e,d) = 0$. Therefore the entire isotypic component of type $\lambda$ in $S^e S^d V^*$ is contained in $I(\Ch_d(V))_e$. In fact, this is also a consequence of the fact that, by construction, $\Ch_d(V)$, regarded as a subvariety of $V^{\otimes d}$, is contained in the subspace variety of tensors whose multilinear ranks are $d$, hence all modules in $S^eS^d V^*$ of type $\lambda$ with $\ell(\lambda) \geq d$ vanish on $\Ch_d(V)$.

\subsection{Inequalities between plethysm coefficients}
\label{RepTheory-subsection-plethysmInequalities}
% Author: Fulvio Gesmundo
Results on inequalities between different structure coefficients in representation theory are of interest in algebraic combinatorics. For plethysm coefficients, several results are known in the asymptotic setting. A fundamental conjecture is due to Foulkes:
\begin{conjecture}
 \label{RepTheory-conjecture-Foulkes}
% Author: Fulvio Gesmundo
 Let $e \geq d$ be nonnegative integers and let $\lambda$ be a partition of $ed$. Then $a_\lambda (e,d) \geq a_\lambda(d,e)$.
\end{conjecture}
A stronger conjecture was posed by Hadamard, predicting that $\rmFH_{d,e}$ is injective for $e \leq d$ and it is known by the Foulkes-Howe conejcture. Proving this stronger conjecture of a specific pair $(e,d)$ clearly proves \ref{RepTheory-conjecture-Foulkes} for the same pair. However, in \cite{MN05}, it was shown that Hadamard's conjecture is false: $\rmFH_{5,5}$ has nontrivial kernel. The situation is far from understood in general and a better understanding would shed light on structural properties of the ideal of Chow varieties.


