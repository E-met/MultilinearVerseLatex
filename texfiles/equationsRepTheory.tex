In a number of settings, algebraic varieties of interest are invariant under the action of an algebraic group. In these cases, representation theory, and in particular \ref{repTheory-lemma-Schur} can be of great help in the study of equations for the variety of interest.

\begin{definition}
 \label{repTheory-definition-Gvariety}
 Let $G$ be a group and let $V$ be a $G$-representation. We say that a variety $X \subseteq \bbP V$ is a $G$-variety if $X$ is invariant under the action of $G$, that is $G \cdot X = X$. 
\end{definition}
We point out that the points of a $G$-variety $X$ are not in general fixed by the action of $G$. If this is the case, we say that $X$ is point-wise invariant by the action of $G$.

The Segre-Veronese varieties $\nu_{d_1 \vvirg d_k} ( \bbP V_1 \ttimes \bbP V_k)$ of \ref{classicalAG-definition-SegreVeronese}, their secant varieties (see \ref{secantVarieties-definition-secantvariety}), tangential varieties, joins of such, and more generally varieties constructed \emph{functorially} from them are $G$-varieties, with $G = \GL(V_1) \ttimes \GL(V_d)$. Orbit-closures for the action of a group $G$, in the sense of \ref{introduction-definition-orbitsdegenerations}, are $G$-variaties as well.

The action of a group $G$ on a space $V$ induces an action of the polynomial ring $\bbC[V]$ via pull-back: given $F \in \bbC[V]$, $g \in G$, one has 
\[
g \cdot F = F \circ g^{-1}.
\]
If the action is linear, it restricts to the homogeneous components of $\bbC[V] = \bigoplus_{d \geq 0} S^d V^*$. If $X$ is a $G$-variety, then the homogeneous components of its ideal are $G$-representations:
\begin{lemma}
\label{repTheory-lemma-GactionOnIdeal}
 Let $X \subseteq \bbP V$ be a $G$-variety. Let $I(X) \subseteq \bbC[V]$ be the ideal of $X$. Then for every $g \in G$ and every $F \in \bbC[V]$, we have that $F \in I(X)$ if and only if $g \cdot F \in I(X)$. In particular, the homogeneous components $I(X)_d \subseteq  S^d V$ are subrepresentations of $G$.
\end{lemma}
Because of this result, every equation $F$ of a $G$-variety yields a \emph{module} of equations, that is the smallest $G$-subrepresentation containing $F$. The term equation is often used loosely, in the general sense of \emph{Zariski closed condition}.

