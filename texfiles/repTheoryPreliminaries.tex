Let $\GL(V)$ denote the general linear group of a vector space $V$ and $\SL(V)$ denote the special linear group. Let $\frakS_d$ denote the symmetric group on $d$ elements. 

A representation of a group $G$ is a group homomorphism $\rho : G \to \GL(V)$ for some vector space $V$. We use the word \emph{representation} to refer to the space $V$ itself. For $g \in G$ and $v \in V$, we often write $g \cdot v$ of $gv$ for the image of $v$ via $\rho(g)$. 

Given $V,W$ representations for a group $G$, and a linear map $\phi : V \to W$, we say that $\phi$ is $G$-equivariant if it commutes with the action of $G$, that is $\phi( g \cdot v) = g \cdot \phi(v)$ for every $v \in V$, $g \in G$. The subset of $\Hom(V,W)$ consisting of $G$-equivariant maps is a linear subspace, and it is denoted by $\Hom_G(V,W)$. 

A fundamental result which will prove useful numerous times is Schur's Lemma. 
\begin{lemma}[Schur's Lemma]
\label{repTheory-lemma-Schur}
Let $G$ be a group, and let $\phi : V \to W$ be an equivariant map between two $G$-representations. Then $\ker(\phi)$ and $\image(\phi)$ are $G$-representations. In particular, if $V$ is irreducible, then $\phi = 0$ or $\phi$ is injective. If $V= W$ then $\phi = \lambda \Id_V$ for some $\lambda \in \bbC$, that is $\dim \Hom _G(V,W) = 1$.
\end{lemma}

 If $H$ is a subgroup of a group $G$, $V$ is a representation of $H$ and $W$ is a representation of $G$, write 
\[
\Ind_H^G ( V) \qquad \Res_H^G (W)
\]
for the $G$-representation \emph{induced} by $V$ and the \emph{restriction} of $W$ to $H$. We refer to \cite[Sec. 4.4.1]{GW09} for details on these constructions. Frobenius reciprocity links these two constructions, see \cite[Thm. 4.4.1]{GW09}:
\begin{lemma}
 \label{RepTheory-lemma-Frobeniusreciprocity}
 Let $G$ be a group with a subgroup $H$. Let $V$ be a representation of $H$ and let $W$ be a representation of $G$. There is a canonical isomorphism
 \[
\Hom_G(W,\Ind_H^G ( V)) \simeq \Hom_H (\Res_H^G (W) , V).
 \]
\end{lemma}
If $V$ is a representation of a group $G$, write $V^G$ for the subspace of $G$-invariants. 

We are particularly interested in representations of the general linear group and of the symmetric group. These are described in terms of combinatorics of \emph{partitions}. A partition $\lambda = (\lambda_1 \vvirg \lambda_n)$ is a non-increasing sequence of positive integers. We say that $\lambda$ is a partition of $d$ if $\sum_i \lambda_i = d$, and we write $|\lambda| = d$; we say that $n$ is the length of $\lambda$ and we write $n = \ell(\lambda)$. Write $\lambda \partinto_n d$ to mean that $\lambda$ is a partition of $d$ of length at most $n$. Repeated numbers in partitions are usually expressed as exponents: for instance $\lambda= (3,3,1,1,1)$ can be written as $\lambda = (3^2,1^3)$. Partitions are represented by Young diagrams, which are top-left justified diagrams of boxes, having $\lambda_j$ boxes in their $j$-th row. Given a partition $\lambda$, denote by $\lambda^{\bft}$ the \emph{transpose} (or conjugate) partition of $\lambda$, obtained by transposing its Young diagram; for instance if $\lambda = (3^2,1^3)$ then $\lambda^\bft = (4,2,2)$. 

The (polynomial) irreducible representations of the general linear group of a vector space of dimension $n$ are indexed by partitions of length at most $n$. Let $\bbS_\lambda V$ be the irreducible representation associated to the partition $n$; this is the Schur module of $\GL(V)$ associated to $\lambda$. If $\lambda = (d)$ consists of a single row then $\bbS_\lambda V = S^d V$ is the representation of symmetric tensors of order $d$; if $\lambda = (1^d)$ consists of a single column then $\bbS_\lambda V = \Lambda^d V$ is the representation of skew-symmetric tensors.

The irreducible representations of the symmetric group $\frakS_d$ are indexed by partitions of $d$. Let $[\lambda]$ be the irreducible representation associated to the partition $\lambda$; this is the Specht module of $\frakS_d$ of type $\lambda$. 

The vector space $V^{\otimes d}$ is acted on naturally by $\frakS_d$, which permutes the tensor factors, and $\GL(V)$ which acts diagonally on all tensor factors; these two actions commute and the Schur-Weyl decomposition theorem expresses the spaces as a direct sum of irreducible representations for $\frakS_d \times \GL(V)$. Such decomposition is as follows:
\[
V^{\otimes d} = \bigoplus_{\lambda \partinto_n d} [\lambda] \otimes \bbS_{\lambda} V.
\]
The projections $V^{\otimes d} \to [\lambda] \otimes \bbS_\lambda V$ are canonical and they can be described combinatorially via Young symmetrizers, see \cite[Lecture 4]{FH91}.

By \ref{repTheory-lemma-Schur}, the Schur-Weyl decomposition shows $\dim \Hom_{\GL(V)} ( \bbS_\lambda V, V^{\otimes d}) = \dim [\lambda]$. More generally, a fundamental problem in representation theory and algebraic combinatorics consists in determining the decomposition of the tensor product of two (or more) irreducible representations. It is convenient to give these definitions in terms of representations of the symmetric group, so that they do not depend on the dimension of the vector space $V$.

\begin{definition}[Littlewood-Richardson coefficients]
 \label{RepTheory-definition-LRcoefficient}
 Let $\lambda,\mu,\nu$ be three partitions of three integers $k,m, k+m$ respectively. The {\it Littlewood-Richardson coefficient} associated to $(\lambda,\mu;\nu)$ is 
 \[
c^\nu_{\lambda,\mu} = 
\dim \Hom_{\frakS_{k+m}} ([\nu], \Ind_{\frakS_k \times \frakS_m}^{\frakS_{k+m}} ([\lambda] \otimes [\mu]) = 
\dim \Hom_{\frakS_{k} \times \frakS_m} ([\lambda] \otimes [\mu] , \Res_{\frakS_k \times \frakS_m}^{\frakS_{k+m}} [\nu]).
\]
\end{definition}
Note that the two dimensions in \ref{RepTheory-definition-LRcoefficient} are equal by \ref{RepTheory-lemma-Frobeniusreciprocity}. From the definition, it is clear that $c^\nu_{\lambda,\mu} = c^\nu_{\mu,\lambda}$. 

We have the following result:
\begin{lemma}
 \label{RepTheory-lemma-LRcoefficientGL}
  Let $\lambda,\mu,\nu$ be three partitions. If $V$ is a vector space with $\dim V \geq \ell(\nu)$, then 
 \[
c^\nu_{\lambda,\mu} = \dim \Hom_{\GL(V)} (\bbS_\nu V,  \bbS_\lambda V \otimes \bbS_\mu V ).
\]
\end{lemma}
Via \ref{repTheory-lemma-Schur}, the Littlewood-Richardson coefficient $c^\nu_{\lambda,\mu} $ coincides with the multiplicity of $\bbS_\nu V$ as a subrepresentation of $ \bbS_\lambda V \otimes \bbS_\mu V $, that is 
\[
 \bbS_\lambda V \otimes \bbS_\mu V  = \bigoplus_{\nu } (\bbS_{\nu} V )^{\oplus c^\nu_{\lambda,\mu} }.
\]

Littlewood-Richardson coefficients are hard to compute \cite{Nar06} even though there are polynomial-time algorithms to decide whether they are zero or nonzero as observed in \cite{MNS12} following \cite{KT99}. However, there are several special cases for which they are easy to determine. For instance, when $\mu$ is a partition consisting of a single row of a single column, we have the following classical result.
\begin{proposition}[Pieri's rule]
\label{RepTheory-proposition-Pieri}
 Let $\lambda$ be a partition and let $d \geq 0$. Then 
 \[
 \bbS_{\lambda} V \otimes S^d V = \bigoplus_{\nu \in R} \bbS_{\nu} V, \qquad  \bbS_{\lambda} V \otimes \Lambda^d V = \bigoplus_{\nu \in C} \bbS_{\nu} V.
 \]
Here $R$ (resp. $C$) is the set of partitions of $|\lambda|+d$ obtained from $\lambda$ by adding $d$ boxes, no two of them on the same column (resp. row).
\end{proposition}
There are variants of \ref{RepTheory-proposition-Pieri} for other groups, such as the orthogonal group or the symplectic group.

We introduce other two structure coefficients which appear in the literature on algebraic combinatorics and play an important role in the study of equations for varieties of tensors. For integers $d,e$, denote by $\frakS_d \wr \frakS_e$ the wreath product of $\frakS_e$ with $\frakS_d$, that is the semidirect product
\[
\frakS_d \wr \frakS_e = (\frakS_d ^{\times e}) \rtimes \frakS_e
\]
where $\frakS_e$ acts by permuting the $e$ copies of $\frakS_d$. The group $\frakS_d \wr \frakS_e $ is naturally a subgroup of $\frakS_{de}$, acting via permutation on pairs $(i,j) \in [d] \times [e]$.
\begin{definition}
 \label{RepTheory-definition-plethysm}
 Let $d,e$ be integers and let $\lambda$ be partitions of $de$. The {\it plethysm coefficient} associated to $(\lambda;d,e)$ is
 \[
a_\lambda(e,d) = \dim [\lambda]^{\frakS_d \wr \frakS_e}.
\]
\end{definition}
We have the following result:
\begin{lemma}
 \label{RepTheory-lemma-plethysmGL}
 Let $d,e$ be integers and let $\lambda$ be a partition of $de$. Let $V$ be a vector space with $\dim V \geq \ell(\lambda)$. Then
 \[
 a_\lambda(e,d) = \dim \Hom_{\GL(V)} ( \bbS_\lambda V , S^e S^d V).
 \]
\end{lemma}
Sometimes the word \emph{plethysm} is used more generally for the \emph{composition} of two representations, and the expression \emph{plethysm coefficient} is used for the corresponding multiplicities.


\begin{definition}
 \label{RepTheory-definition-kroncoefficient}
 Let $\lambda,\mu,\nu$ be partitions of an integer $d$. The {\it Kronecker coefficient} associated to $(\lambda,\mu,\nu)$ is 
 \[
 g_{\lambda,\mu,\nu} = \dim ([\lambda] \otimes [\mu] \otimes [\nu])^{\frakS_d}
 \]
\end{definition}
We have the following result:
\begin{lemma}
 \label{RepTheory-lemma-kroncoefficientGL}
 Let $\lambda,\mu,\nu$ be partitions of an integer $d$. Let $V,W$ be vector spaces with $\dim V \geq \ell(\lambda)$, $\dim W \geq \ell(\mu)$. Then 
 \[
 g_{\lambda,\mu,\nu} = \dim \Hom_{\GL(V)} ( \bbS_\lambda V \otimes \bbS_\mu W , \bbS_\nu(V \otimes W)).
 \]
\end{lemma}



