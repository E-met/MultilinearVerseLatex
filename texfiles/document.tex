\documentclass[oneside]{book}
\usepackage{amsmath}
\usepackage{amssymb}

 \usepackage{verbatim}
 
 \usepackage[all]{xy}
 
 \xyoption{2cell}
 \UseAllTwocells
 
 
 \usepackage{lmodern}
 \usepackage[T1]{fontenc}
 

 \usepackage{hyperref}
 \usepackage{enumerate} 


\usepackage{amsthm}
 
 \theoremstyle{plain}
 \newtheorem{theorem}{Theorem}[section]
 \newtheorem{proposition}[theorem]{Proposition}
 \newtheorem{lemma}[theorem]{Lemma}
 \newtheorem{corollary}[theorem]{Corollary} 
 
 \theoremstyle{definition} 
 \newtheorem{definition}[theorem]{Definition}
 \newtheorem{example}[theorem]{Example}
  \newtheorem{remark}[theorem]{Remark}
 
 \numberwithin{equation}{subsection}


 \usepackage{amsthm}

 \newcommand{\xto}[1]{\xrightarrow{\phantom{a}{#1}{\phantom{a}}}}
\newcommand{\xbackto}[1]{\xleftarrow{\phantom{a}{#1}{\phantom{a}}}}

\newcommand{\vvirg}{ , \ldots , }
\newcommand{\ootimes}{ \otimes \cdots \otimes }
\newcommand{\ooplus}{ \oplus \cdots \oplus }
\newcommand{\ttimes}{ \times \cdots \times }
\newcommand{\bboxtimes}{ \boxtimes \cdots \boxtimes }
\newcommand{\wwedge}{ \wedge \dots \wedge}

\newcommand{\contract}{\rotatebox[origin=c]{180}{ \reflectbox{$\neg$} }}

\newcommand{\biglangle}{\bigl\langle}
\newcommand{\bigrangle}{\bigr\rangle}

\newcommand{\Wedge}{\bigwedge}

\newcommand{\textbigotimes}{{\textstyle \bigotimes}}
\newcommand{\textbigtimes}{{\textstyle \bigtimes}}
\newcommand{\textbinom}[2]{{\textstyle \binom{#1}{#2}}}

\newcommand{\textfrac}[2]{{\textstyle \frac{#1}{#2}}}

\newcommand{\bigboxtimes}{{\scalebox{1.2}{$\boxtimes$}}}

%Alphabet Shortcuts

%capital boldface
\newcommand{\bfA}{\mathbf{A}}
\newcommand{\bfB}{\mathbf{B}}
\newcommand{\bfC}{\mathbf{C}}
\newcommand{\bfD}{\mathbf{D}}
\newcommand{\bfE}{\mathbf{E}}
\newcommand{\bfF}{\mathbf{F}}
\newcommand{\bfG}{\mathbf{G}}
\newcommand{\bfH}{\mathbf{H}}
\newcommand{\bfI}{\mathbf{I}}
\newcommand{\bfJ}{\mathbf{J}}
\newcommand{\bfK}{\mathbf{K}}
\newcommand{\bfL}{\mathbf{L}}
\newcommand{\bfM}{\mathbf{M}}
\newcommand{\bfN}{\mathbf{N}}
\newcommand{\bfO}{\mathbf{O}}
\newcommand{\bfP}{\mathbf{P}}
\newcommand{\bfQ}{\mathbf{Q}}
\newcommand{\bfR}{\mathbf{R}}
\newcommand{\bfS}{\mathbf{S}}
\newcommand{\bfT}{\mathbf{T}}
\newcommand{\bfU}{\mathbf{U}}
\newcommand{\bfV}{\mathbf{V}}
\newcommand{\bfW}{\mathbf{W}}
\newcommand{\bfX}{\mathbf{X}}
\newcommand{\bfY}{\mathbf{Y}}
\newcommand{\bfZ}{\mathbf{Z}}

%small boldface
\newcommand{\bfa}{\mathbf{a}}
\newcommand{\bfb}{\mathbf{b}}
\newcommand{\bfc}{\mathbf{c}}
\newcommand{\bfd}{\mathbf{d}}
\newcommand{\bfe}{\mathbf{e}}
\newcommand{\bff}{\mathbf{f}}
\newcommand{\bfg}{\mathbf{g}}
\newcommand{\bfh}{\mathbf{h}}
\newcommand{\bfi}{\mathbf{i}}
\newcommand{\bfj}{\mathbf{j}}
\newcommand{\bfk}{\mathbf{k}}
\newcommand{\bfl}{\mathbf{l}}
\newcommand{\bfm}{\mathbf{m}}
\newcommand{\bfn}{\mathbf{n}}
\newcommand{\bfo}{\mathbf{o}}
\newcommand{\bfp}{\mathbf{p}}
\newcommand{\bfq}{\mathbf{q}}
\newcommand{\bfr}{\mathbf{r}}
\newcommand{\bfs}{\mathbf{s}}
\newcommand{\bft}{\mathbf{t}}
\newcommand{\bfu}{\mathbf{u}}
\newcommand{\bfv}{\mathbf{v}}
\newcommand{\bfw}{\mathbf{w}}
\newcommand{\bfx}{\mathbf{x}}
\newcommand{\bfy}{\mathbf{y}}
\newcommand{\bfz}{\mathbf{z}}

%capital mathcal
\newcommand{\calA}{\mathcal{A}}
\newcommand{\calB}{\mathcal{B}}
\newcommand{\calC}{\mathcal{C}}
\newcommand{\calD}{\mathcal{D}}
\newcommand{\calE}{\mathcal{E}}
\newcommand{\calF}{\mathcal{F}}
\newcommand{\calG}{\mathcal{G}}
\newcommand{\calH}{\mathcal{H}}
\newcommand{\calI}{\mathcal{I}}
\newcommand{\calJ}{\mathcal{J}}
\newcommand{\calK}{\mathcal{K}}
\newcommand{\calL}{\mathcal{L}}
\newcommand{\calM}{\mathcal{M}}
\newcommand{\calN}{\mathcal{N}}
\newcommand{\calO}{\mathcal{O}}
\newcommand{\calP}{\mathcal{P}}
\newcommand{\calQ}{\mathcal{Q}}
\newcommand{\calR}{\mathcal{R}}
\newcommand{\calS}{\mathcal{S}}
\newcommand{\calT}{\mathcal{T}}
\newcommand{\calU}{\mathcal{U}}
\newcommand{\calV}{\mathcal{V}}
\newcommand{\calW}{\mathcal{W}}
\newcommand{\calX}{\mathcal{X}}
\newcommand{\calY}{\mathcal{Y}}
\newcommand{\calZ}{\mathcal{Z}}

%capital mathscr
\newcommand{\scrA}{\mathscr{A}}
\newcommand{\scrB}{\mathscr{B}}
\newcommand{\scrC}{\mathscr{C}}
\newcommand{\scrD}{\mathscr{D}}
\newcommand{\scrE}{\mathscr{E}}
\newcommand{\scrF}{\mathscr{F}}
\newcommand{\scrG}{\mathscr{G}}
\newcommand{\scrH}{\mathscr{H}}
\newcommand{\scrI}{\mathscr{I}}
\newcommand{\scrJ}{\mathscr{J}}
\newcommand{\scrK}{\mathscr{K}}
\newcommand{\scrL}{\mathscr{L}}
\newcommand{\scrM}{\mathscr{M}}
\newcommand{\scrN}{\mathscr{N}}
\newcommand{\scrO}{\mathscr{O}}
\newcommand{\scrP}{\mathscr{P}}
\newcommand{\scrQ}{\mathscr{Q}}
\newcommand{\scrR}{\mathscr{R}}
\newcommand{\scrS}{\mathscr{S}}
\newcommand{\scrT}{\mathscr{T}}
\newcommand{\scrU}{\mathscr{U}}
\newcommand{\scrV}{\mathscr{V}}
\newcommand{\scrW}{\mathscr{W}}
\newcommand{\scrX}{\mathscr{X}}
\newcommand{\scrY}{\mathscr{Y}}
\newcommand{\scrZ}{\mathscr{Z}}

%capital mathbb
\newcommand{\bbA}{\mathbb{A}}
\newcommand{\bbB}{\mathbb{B}}
\newcommand{\bbC}{\mathbb{C}}
\newcommand{\bbD}{\mathbb{D}}
\newcommand{\bbE}{\mathbb{E}}
\newcommand{\bbF}{\mathbb{F}}
\newcommand{\bbG}{\mathbb{G}}
\newcommand{\bbH}{\mathbb{H}}
\newcommand{\bbI}{\mathbb{I}}
\newcommand{\bbJ}{\mathbb{J}}
\newcommand{\bbK}{\mathbb{K}}
\newcommand{\bbL}{\mathbb{L}}
\newcommand{\bbM}{\mathbb{M}}
\newcommand{\bbN}{\mathbb{N}}
\newcommand{\bbO}{\mathbb{O}}
\newcommand{\bbP}{\mathbb{P}}
\newcommand{\bbQ}{\mathbb{Q}}
\newcommand{\bbR}{\mathbb{R}}
\newcommand{\bbS}{\mathbb{S}}
\newcommand{\bbT}{\mathbb{T}}
\newcommand{\bbU}{\mathbb{U}}
\newcommand{\bbV}{\mathbb{V}}
\newcommand{\bbW}{\mathbb{W}}
\newcommand{\bbX}{\mathbb{X}}
\newcommand{\bbY}{\mathbb{Y}}
\newcommand{\bbZ}{\mathbb{Z}}

%small mathbb
\newcommand{\bba}{\mathbb{a}}
\newcommand{\bbb}{\mathbb{b}}
\newcommand{\bbc}{\mathbb{c}}
\newcommand{\bbd}{\mathbb{d}}
\newcommand{\bbe}{\mathbb{e}}
\newcommand{\bbf}{\mathbb{f}}
\newcommand{\bbg}{\mathbb{g}}
\newcommand{\bbh}{\mathbb{h}}
\newcommand{\bbi}{\mathbb{i}}
\newcommand{\bbj}{\mathbb{j}}
\newcommand{\bbk}{\mathbb{k}}
\newcommand{\bbl}{\mathbb{l}}
\newcommand{\bbm}{\mathbb{m}}
\newcommand{\bbn}{\mathbb{n}}
\newcommand{\bbo}{\mathbb{o}}
\newcommand{\bbp}{\mathbb{p}}
\newcommand{\bbq}{\mathbb{q}}
\newcommand{\bbr}{\mathbb{r}}
\newcommand{\bbs}{\mathbb{s}}
\newcommand{\bbt}{\mathbb{t}}
\newcommand{\bbu}{\mathbb{u}}
\newcommand{\bbv}{\mathbb{v}}
\newcommand{\bbw}{\mathbb{w}}
\newcommand{\bbx}{\mathbb{x}}
\newcommand{\bby}{\mathbb{y}}
\newcommand{\bbz}{\mathbb{z}}

%capital mathfrak
\newcommand{\frakA}{\mathfrak{A}}
\newcommand{\frakB}{\mathfrak{B}}
\newcommand{\frakC}{\mathfrak{C}}
\newcommand{\frakD}{\mathfrak{D}}
\newcommand{\frakE}{\mathfrak{E}}
\newcommand{\frakF}{\mathfrak{F}}
\newcommand{\frakG}{\mathfrak{G}}
\newcommand{\frakH}{\mathfrak{H}}
\newcommand{\frakI}{\mathfrak{I}}
\newcommand{\frakJ}{\mathfrak{J}}
\newcommand{\frakK}{\mathfrak{K}}
\newcommand{\frakL}{\mathfrak{L}}
\newcommand{\frakM}{\mathfrak{M}}
\newcommand{\frakN}{\mathfrak{N}}
\newcommand{\frakO}{\mathfrak{O}}
\newcommand{\frakP}{\mathfrak{P}}
\newcommand{\frakQ}{\mathfrak{Q}}
\newcommand{\frakR}{\mathfrak{R}}
\newcommand{\frakS}{\mathfrak{S}}
\newcommand{\frakT}{\mathfrak{T}}
\newcommand{\frakU}{\mathfrak{U}}
\newcommand{\frakV}{\mathfrak{V}}
\newcommand{\frakW}{\mathfrak{W}}
\newcommand{\frakX}{\mathfrak{X}}
\newcommand{\frakY}{\mathfrak{Y}}
\newcommand{\frakZ}{\mathfrak{Z}}

%small mathfrak
\newcommand{\fraka}{\mathfrak{a}}
\newcommand{\frakb}{\mathfrak{b}}
\newcommand{\frakc}{\mathfrak{c}}
\newcommand{\frakd}{\mathfrak{d}}
\newcommand{\frake}{\mathfrak{e}}
\newcommand{\frakf}{\mathfrak{f}}
\newcommand{\frakg}{\mathfrak{g}}
\newcommand{\frakh}{\mathfrak{h}}
\newcommand{\fraki}{\mathfrak{i}}
\newcommand{\frakj}{\mathfrak{j}}
\newcommand{\frakk}{\mathfrak{k}}
\newcommand{\frakl}{\mathfrak{l}}
\newcommand{\frakm}{\mathfrak{m}}
\newcommand{\frakn}{\mathfrak{n}}
\newcommand{\frako}{\mathfrak{o}}
\newcommand{\frakp}{\mathfrak{p}}
\newcommand{\frakq}{\mathfrak{q}}
\newcommand{\frakr}{\mathfrak{r}}
\newcommand{\fraks}{\mathfrak{s}}
\newcommand{\frakt}{\mathfrak{t}}
\newcommand{\fraku}{\mathfrak{u}}
\newcommand{\frakv}{\mathfrak{v}}
\newcommand{\frakw}{\mathfrak{w}}
\newcommand{\frakx}{\mathfrak{x}}
\newcommand{\fraky}{\mathfrak{y}}
\newcommand{\frakz}{\mathfrak{z}}

%capital roman
\newcommand{\rmA}{\mathrm{A}}
\newcommand{\rmB}{\mathrm{B}}
\newcommand{\rmC}{\mathrm{C}}
\newcommand{\rmD}{\mathrm{D}}
\newcommand{\rmE}{\mathrm{E}}
\newcommand{\rmF}{\mathrm{F}}
\newcommand{\rmG}{\mathrm{G}}
\newcommand{\rmH}{\mathrm{H}}
\newcommand{\rmI}{\mathrm{I}}
\newcommand{\rmJ}{\mathrm{J}}
\newcommand{\rmK}{\mathrm{K}}
\newcommand{\rmL}{\mathrm{L}}
\newcommand{\rmM}{\mathrm{M}}
\newcommand{\rmN}{\mathrm{N}}
\newcommand{\rmO}{\mathrm{O}}
\newcommand{\rmP}{\mathrm{P}}
\newcommand{\rmQ}{\mathrm{Q}}
\newcommand{\rmR}{\mathrm{R}}
\newcommand{\rmS}{\mathrm{S}}
\newcommand{\rmT}{\mathrm{T}}
\newcommand{\rmU}{\mathrm{U}}
\newcommand{\rmV}{\mathrm{V}}
\newcommand{\rmW}{\mathrm{W}}
\newcommand{\rmX}{\mathrm{X}}
\newcommand{\rmY}{\mathrm{Y}}
\newcommand{\rmZ}{\mathrm{Z}}

%small roman
\newcommand{\rma}{\mathrm{a}}
\newcommand{\rmb}{\mathrm{b}}
\newcommand{\rmc}{\mathrm{c}}
\newcommand{\rmd}{\mathrm{d}}
\newcommand{\rme}{\mathrm{e}}
\newcommand{\rmf}{\mathrm{f}}
\newcommand{\rmg}{\mathrm{g}}
\newcommand{\rmh}{\mathrm{h}}
\newcommand{\rmi}{\mathrm{i}}
\newcommand{\rmj}{\mathrm{j}}
\newcommand{\rmk}{\mathrm{k}}
\newcommand{\rml}{\mathrm{l}}
\newcommand{\rmm}{\mathrm{m}}
\newcommand{\rmn}{\mathrm{n}}
\newcommand{\rmo}{\mathrm{o}}
\newcommand{\rmp}{\mathrm{p}}
\newcommand{\rmq}{\mathrm{q}}
\newcommand{\rmr}{\mathrm{r}}
\newcommand{\rms}{\mathrm{s}}
\newcommand{\rmt}{\mathrm{t}}
\newcommand{\rmu}{\mathrm{u}}
\newcommand{\rmv}{\mathrm{v}}
\newcommand{\rmw}{\mathrm{w}}
\newcommand{\rmx}{\mathrm{x}}
\newcommand{\rmy}{\mathrm{y}}
\newcommand{\rmz}{\mathrm{z}}

%small greek boldface
\newcommand{\bfalpha}{\boldsymbol{\alpha}}
\newcommand{\bfbeta}{\boldsymbol{\beta}}
\newcommand{\bfgamma}{\boldsymbol{\gamma}}
\newcommand{\bfdelta}{\boldsymbol{\delta}}
\newcommand{\bfepsilon}{\boldsymbol{\epsilon}}
\newcommand{\bfzeta}{\boldsymbol{\zeta}}
\newcommand{\bfeta}{\boldsymbol{\eta}}
\newcommand{\bftheta}{\boldsymbol{\theta}}
\newcommand{\bfiota}{\boldsymbol{\iota}}
\newcommand{\bfkappa}{\boldsymbol{\kappa}}
\newcommand{\bflambda}{\boldsymbol{\lambda}}
\newcommand{\bfmu}{\boldsymbol{\mu}}
\newcommand{\bfnu}{\boldsymbol{\nu}}
\newcommand{\bfxi}{\boldsymbol{\xi}}
\newcommand{\bfpi}{\boldsymbol{\pi}}
\newcommand{\bfrho}{\boldsymbol{\rho}}
\newcommand{\bfsigma}{\boldsymbol{\sigma}}
\newcommand{\bftau}{\boldsymbol{\tau}}
\newcommand{\bfphi}{\boldsymbol{\varphi}}
\newcommand{\bfchi}{\boldsymbol{\chi}}
\newcommand{\bfpsi}{\boldsymbol{\psi}}
\newcommand{\bfomega}{\boldsymbol{\omega}}

%capital greek boldface
\newcommand{\bfGamma}{\boldsymbol{\Gamma}}
\newcommand{\bfDelta}{\boldsymbol{\Delta}}
\newcommand{\bfTheta}{\boldsymbol{\Theta}}
\newcommand{\bfLambda}{\boldsymbol{\Lambda}}
\newcommand{\bfXi}{\boldsymbol{\Xi}}
\newcommand{\bfPi}{\boldsymbol{\Pi}}
\newcommand{\bfSigma}{\boldsymbol{\Sigma}}
\newcommand{\bfPhi}{\boldsymbol{\Phi}}
\newcommand{\bfPsi}{\boldsymbol{\Psi}}
\newcommand{\bfOmega}{\boldsymbol{\Omega}}

% Preferences
\newcommand{\eps}{\varepsilon}

\newcommand{\injects}{\hookrightarrow}
\newcommand{\surjects}{\twoheadrightarrow}
\newcommand{\backto}{\longleftarrow}
\newcommand{\dashto}{\dashrightarrow}


\newcommand{\pder}{\mathop{}\!\mathrm{\partial}}
\newcommand{\pbder}{\mathop{}\!\mathrm{\overline{\partial}}}
\newcommand{\diff}{\mathop{}\!\mathrm{d}}
\newcommand{\Lap}{\mathop{}\!\Delta}

\newcommand{\id}{\mathrm{id}}
\newcommand{\Id}{\mathrm{Id}}
%\newcommand{\wreath}{\scalebox{1}[.75]{\textbf{\descnode}}} %nice wreath product symbol

\newcommand{\Ad}{\mathrm{Ad}}
\newcommand{\ad}{\mathrm{ad}}

\newcommand{\corank}{\mathrm{corank}}
\newcommand{\rank}{\mathrm{rank}}
\newcommand{\gen}{\mathrm{gen}}
% Algebra
\newcommand{\image}{\mathrm{image}}  
\DeclareMathOperator{\Cat}{Cat}
\DeclareMathOperator{\codim}{codim}
\DeclareMathOperator{\coker}{coker}
\DeclareMathOperator{\Hom}{Hom}
\newcommand{\calHom}{\mathcal{H}om}
\DeclareMathOperator{\Supp}{Supp}
\DeclareMathOperator{\Tr}{Tr}
\DeclareMathOperator{\trace}{trace}
\DeclareMathOperator{\diag}{diag}
\DeclareMathOperator{\rk}{rk}
\DeclareMathOperator{\End}{End}
\DeclareMathOperator{\Pf}{Pf}
\DeclareMathOperator{\Sym}{Sym}
\DeclareMathOperator{\Aut}{Aut}
\DeclareMathOperator{\Inn}{Inn}
\DeclareMathOperator{\Out}{Out}
\DeclareMathOperator{\Spec}{Spec}
\DeclareMathOperator{\Proj}{Proj}
\DeclareMathOperator{\sgn}{sgn}
\DeclareMathOperator{\charact}{char}
\DeclareMathOperator{\Gal}{Gal}
\DeclareMathOperator{\Div}{Div}
\DeclareMathOperator{\ord}{ord}
\DeclareMathOperator{\Cl}{Cl}
\DeclareMathOperator{\Cox}{Cox}
\DeclareMathOperator{\Pic}{Pic}
\DeclareMathOperator{\Ann}{Ann}
\DeclareMathOperator{\rad}{rad}
\DeclareMathOperator{\soc}{soc}
\DeclareMathOperator{\tr}{tr}
\DeclareMathOperator{\socdeg}{socdeg}
\DeclareMathOperator{\socle}{socle}

\newcommand{\SO}{\mathrm{SO}}
\newcommand{\GL}{\mathrm{GL}}
\newcommand{\SL}{\mathrm{SL}}
\newcommand{\PGL}{\mathrm{PGL}}
\newcommand{\fraksl}{\mathfrak{sl}}
\newcommand{\frakgl}{\mathfrak{gl}}

\newcommand{\Bl}{\mathrm{Bl}}
\newcommand{\Flat}{\mathrm{Flat}}
\newcommand{\YFlat}{\mathrm{YFlat}}

\newcommand{\Seg}{\mathrm{Seg}}
\newcommand{\Gr}{\mathrm{Gr}}

\newcommand{\Hilb}{\mathrm{Hilb}}
\newcommand{\Gor}{\mathrm{Gor}}
\newcommand{\Ch}{\mathrm{Ch}}

\newcommand{\vol}{\mathrm{vol}}
% Categories

\DeclareMathOperator{\Mor}{Mor}
\newcommand{\frakMod}{\frakM\frako\frakd}
\newcommand{\frakTop}{\frakT\frako\frakp}
\newcommand{\frakSch}{\frakS\frakc\frakh}
\newcommand{\frakAb}{\frakA\frakb}
\newcommand{\frakGp}{\mathfrak{Gp}}
\newcommand{\frakSet}{\mathfrak{Set}}

% complexity classes
\newcommand{\bfVNP}{\mathbf{VNP}}
\newcommand{\bfVP}{\mathbf{VP}}
\newcommand{\bfNP}{\mathbf{NP}}

\newcommand{\bfVQP}{\mathbf{VQP}}

\DeclareMathOperator{\diam}{diam}
\DeclareMathOperator{\dist}{dist}
\DeclareMathOperator{\mult}{mult}

\newcommand{\rmFH}{\mathrm{FH}}
\newcommand{\rmHF}{\mathrm{HF}}
\newcommand{\rmHP}{\mathrm{HP}}


\newcommand{\supp}{\mathrm{supp}}
\newcommand{\slrk}{\mathrm{slrk}}
\newcommand{\aslrk}{\uwave{\mathrm{slrk}}}
\newcommand{\uslrk}{\underline{\mathrm{slrk}}}
\newcommand{\uR}{\underline{\rmR}}
\newcommand{\uQ}{\underline{\rmQ}}

\newcommand{\aQ}{\uwave{\rmQ}}
\newcommand{\cR}{\mathrm{cR}}
\newcommand{\ucR}{\underline{\cR}}



\newcommand{\perm}{\mathrm{perm}}
\newcommand{\MaMu}{\mathbf{MaMu}}
\newcommand{\Pow}{\mathrm{Pow}}
\newcommand{\IMM}{\mathrm{IMM}}

\DeclareMathOperator{\Res}{Res}
\DeclareMathOperator{\Ind}{Ind}
\DeclareMathOperator{\Lie}{Lie}


\newcommand{\Mat}{\mathrm{Mat}}
\newcommand{\bOmega}{\overline{\Omega}}
\newcommand{\bond}{\mathrm{bond}}
\newcommand{\bbond}{\underline{\mathrm{bond}}}

\newcommand{\deggeq}{\unrhd}

\DeclareMathOperator{\Stab}{Stab}
\newcommand{\MPS}{\mathcal{MPS}}
\newcommand{\TNS}{\mathcal{T\!N\!S}}


\newcommand{\partinto}{\vdash}




%%% Last Update: Feb 15, 2024


\begin{document}

\part{Introduction}
\label{part-introduction}
Tensors manifest in various forms across mathematics and numerous other fields. The roots of the study of tensors and their properties goes back by more than a century: for example, in 1850s, Sylvester was already considering {\it decompositions of binary forms as sums of powers} \cite{Syl51}; or, since the late XIX century, classical algebraic geometers were already studying {\it secant varieties} \cite{Sco08,Ter11}; or, in 1920s, Hitchcock already introduced {\it polyadic decompositions} of tensors \cite{Hit27}.

Considering the long story spanning multiple domains, there is are many beautiful and rich resources available for both beginners and experts and written in different languages with varying jargon. In particular, in recent years, we have witnessed a remarkable growth in literature pertaining to what can broadly be called \emph{geometry of tensors}. Scholars from diverse fields such as classical geometry, theoretical computer science, data science, and quantum physics gave their contribution, actively engaging in each other's open problems and questions. However, due to the varied languages used in these works, gaining a comprehensive overview of the field's (or better, the fields') current state presents a significant challenge. Just to mention some of them, we refer to \cite{Zak93,Ger96,IK99,Hac12,Lan12,CGO14,Rus16,BCCGO18,BBCMM19}. 

It was from this realization that the concept of a unifying portal was born. The \textbf{MultilinearVerse} wants to be an online platform conceived with the aim of serving as such a unifying resource with the possibility of growing in the future to cover all different facets of the subject and dynamically follow the advancement of state-of-the-art research.

The development of this portal started in early 2024, drawing considerable inspiration from The Stacks project \cite{Stacks}. The portal is built on the Gerby project, an online tag-based view for large \LaTeX~documents. The tag-based system ensures that every page, statement, and \emph{label} within the document is assigned its own unique tag, facilitating easy referencing. MultilinearVerse is not a book and it does not intend to replace classical resources. It has chapter, sections, and a linear numbering of its statements. Yet, its design encourages a tree-based approach to navigation, based on the mathematical connections between different topics, which are made readily apparent through various interlinks.

We do not plan that the portal will cover background material, to the level of a Master's degree in Mathematics. We will include more precise references on the background in the single sections of the portal. We acknowledge that the definition of \emph{background material} will be mostly arbitrary. After all, MultilinearVerse is a product of its developers, and necessarily reflects their mathematical strengths and weaknesses. Most of the geometry is limited to the complex setting, with a focus on results and methods coming from classical algebraic geometry. We expect and we hope that \emph{all} aspects of the multifaceted world of tensors will be covered eventually with the contribution of our rich community. 

\chapter{Preliminaries}
\label{introduction-chapter-intropreliminaries}
In this chapter, we introduce all basic notions, terminologies and notations that will be used through all the MultilinearVerse. 

\section{Representing tensors}
\label{introduction-section-representingtensors}

\begin{definition}[Tensor product]\label{introduction-definition-tensor_product}
Let $V,W$ be finite dimensional $\Bbbk$-vector spaces.
\begin{itemize}
    \item Let $\Hom(V,W)$ denote the vector space of linear functions from $V$ to $W$.
    \item Let $V^* = \Hom(V,\Bbbk)$ denote the \emph{dual space} of $V$.
    \item Let $V \otimes W$ denote the tensor product of $V$ and $W$: it can be equivalently identified with the space of linear maps $\Hom(V^*,W) \simeq \Hom(W^*,V)$ or with the space of bilinear map $V \times W \to \bbC$.
\end{itemize}
Given $\Bbbk$-vector spaces $V_1 \vvirg V_d$, let $V_1 \ootimes V_d$ denote their tensor product, which is identified with the space of multilinear maps $V_1^* \ttimes V_d^* \to \bbC$. If $V_1 = \cdots = V_d = V$, then we write for short $V^{\otimes d}$. 

The elements of a tensor product of $d$ vector spaces are called \emph{tensors} of \emph{order $d$}.

In coordinates, if $\{v_0,\ldots,v_m\}$ is a basis of $V$ and $\{w_0,\ldots,w_n\}$ is a basis of $W$, then $\{v_i \otimes w_j ~:~ i \in \{0,\ldots,m\}, j \in \{0,\ldots,n\}$ is a basis of $V\otimes W$. Therefore, any element $T \in V\otimes W$ can be written uniquely as 
\[
    T = \sum_{i,j} t_{i,j}v_i\otimes w_j;
\]
therefore, the choice of basis gives an identification of the space $V\otimes W$ with the space of $m\times n$ \emph{matrices} with coefficients in $\Bbbk$. In general, if $\{v_0^{(i)},\ldots,v_{n_i}^{(i)}\}$ is a basis of $V_i$, then any element $T \in V_1 \ootimes V_d$ can be written uniquely as 
\[
    T = \sum_{i_1,\ldots,i_d} t_{i_1,\ldots,i_d}v_{i_1}^{(1)}\ootimes v_{i_d}^{(d)};
\]
therefore, the choice of basis gives an identification of the space $V_1 \ootimes V_d$ with the space of $d$-dimensional arrays of size $n_1 \times\cdots\times n_d$. 
\end{definition}

\begin{definition}[Flattenings]\label{introduction-definition-flattenings}
For every $K \subseteq \{1 \vvirg d\}$, let $K^c = \{1\vvirg d\} \smallsetminus K$. Then, a tensor $T \in V_1 \ootimes V_d$ defines a linear map 
\[
    \Flat_K(T): \bigotimes_{k \in K} V_{k}^* \to  \bigotimes_{k' \in K^c} V_{k'}
\]
called \emph{flattening} of $T$. It is easy to see that $\Flat_K(T) = (\Flat_{K^c}(T))^\bft$. 
\end{definition}

\begin{definition}[Concise tensors]\label{introduction-definition-concise}
    A tensor $T$ is \emph{concise} if the flattenings $T_k : V_k^* \to \bigotimes_{k' \neq k} V_{k'}$ are injective. 
\end{definition}

A tensor $T \in V_1 \ootimes V_d$ can be identified with the image of its flattening $T : V_1^* \to V_2 \ootimes V_d$, up to a change of coordinates on the space $V_1$. Here, we consider the action of $\GL(V_1)$ on $V_1 \ootimes V_d$: for more about group actions, see \ref{introduction-section-groupactions}. In other words, if $T,T'$ are two tensors such that the image of their first flattening is the same subspace of $V_2 \ootimes V_d$, then there exists an element $g \in \GL(V_1)$ such that $T = g\cdot T'$. Similar statements hold with respect to the other simple flattenings.

\subsection{Symmetric Tensors}
\label{introduction-subsection-symmetric_tensors}

On the space of tensors $V^{\otimes d}$, we consider the action of the \emph{symmetric group} $\frakS_d$ which acts by permuting the tensor factors: if $T = (t_{i_1,\ldots,i_d}) \in V^{\otimes d}$, then, for any $\sigma \in \frakS_d$, then \[(\sigma \cdot T)_{i_1,\ldots,i_d} = t_{i_{\sigma(1)},\ldots,i_{\sigma(d)}}\] for any multi-index $(i_1,\ldots,i_d)$. 
\begin{definition}[Symmetric tensors]
    \label{introduction-definition-symmetric_tensors}

    A tensor $T \in V^{\otimes d}$ which is invariant under the action of $\frakS_d$ is said to be \emph{symmetric}. Let $S^d V \subseteq V^{\otimes d}$ denote the space of symmetric tensors. 

    If the base field has characteristic $0$, then $S^d V$ can be identified with the space of \emph{homogeneous polynomials} of degree $d$ on $V^*$. For example, the monomial $xy^2$ is identified with the symmetric tensor $\frac{1}{3}\left(x\otimes y \otimes y + y\otimes x \otimes y + y\otimes y \otimes x\right)$.
\end{definition}

\subsection{Partially Symmetric Tensors}
\label{introduction-subsection-partially_symmetric_tensors}

On the space of tensors $V_1^{\otimes d_1}\ootimes V_m^{\otimes d_m}$, we consider the action of the \emph{symmetric group} $\frakS_{d_1} \ttimes \frakS_{d_m}$ which acts by permuting the tensor factors accordingly to the partition: namely, if $T = T_1 \ootimes T_m \in V_1^{\otimes d_1}\ootimes V_m^{\otimes d_m}$, with $T_i \in V_i^{\otimes d_i}$, then, for any $\sigma = (\sigma_1,\ldots,\sigma_m)\in \frakS_{d_1} \ttimes \frakS_{d_m}$, then \[\sigma \cdot T = (\sigma_1\cdot T_1) \ootimes (\sigma_m\cdot T_m) \in V_1^{\otimes d_1}\ootimes V_m^{\otimes d_m},\] where each $\sigma_i$ acts permuting the factors of $T_i$ as explained in \ref{introduction-definition-symmetric_tensors}.
\begin{definition}[Partially symmetric tensors]
    \label{introduction-definition-partially_symmetric_tensors}
    
    A tensor $T \in V_1^{\otimes d_1}\ootimes V_m^{\otimes d_m}$ which is invariant under the action of $\frakS_{d_1} \ttimes \frakS_{d_m}$ is said to be \emph{partially symmetric}. Let $S^{d_1,\ldots,d_m}(V_1,\ldots,V_m) = S^{d_1} V_1 \ootimes S^{d_m}V_m \subseteq V_1^{\otimes d_1}\ootimes V_m^{\otimes d_m}$ denote the space of partially symmetric tensors. 

    If the base field has characteristic $0$, then $S^{d_1} V_1 \ootimes S^{d_m}V_m \subseteq V_1^{\otimes d_1}\ootimes V_m^{\otimes d_m}$ can be identified with the space of \emph{multi-homogeneous polynomials} of multi-degree $(d_1,\ldots,d_m)$ on $V^*_1 \ttimes V^*_m$. 
\end{definition}

Inside the space of tensors $V^{\otimes d}$, we have all possible spaces of partially symmetric tensors $S^{\underline{d}}(V) = S^{d_1} V \ootimes S^{d_m}V$ for all partitions $\underline{d} = (d_1,\ldots,d_m)\vdash d$. In particular, $S^{\underline{d}}(V) \subset S^{\underline{d'}}(V)$ if $\underline{d}'$ is a refinement of $\underline{d}$.

Fixed a basis $\{w_0,\ldots,w_m\}$ of $W$, any tensor $T \in W \otimes S^{d-1} V$ can be written as $T = \sum_{i=1}^m w_i \otimes T_i$ and then identified with the $m$-tuple $\{T_1,\ldots,T_m\}$ of elements of $S^{d-1} V$. This is also a set of generators for the image of the first flattening $\Flat_1(T)$, see \ref{introduction-definition-flattenings}.

For example, if $F \in S^d V$ is a symmetric tensor, with $\{v_0 \vvirg v_n\}$ basis of $V$, then $F$ is regarded as the element in $S^{(1,d-1)}(V)$ given by 
\[
    \sum_{j=0}^n v_j \otimes \textstyle\frac{\partial }{\partial v_j} F
\]
where the corresponding $(n+1)$-tuple of elements of $S^{d-1} V$ can be identified with the gradient of the homogeneous polynomial $F$.
\subsection{Skew-symmetric Tensors}\label{introduction-subsection-skew_symmetric_tensors}
On the space of tensors $V^{\otimes d}$, we consider the skew-symmetric action of the symmetric group $\frakS_d$: if $T = (t_{i_1,\ldots,i_d}) \in V^{\otimes d}$, then, for any $\sigma \in \frakS_d$, then $(\sigma \cdot T)_{i_1,\ldots,i_d} = \sgn(\sigma)t_{i_{\sigma(1)},\ldots,i_{\sigma(d)}}$ for any multi-index $(i_1,\ldots,i_d)$.
\begin{definition}[Skew symmetric tensors]\label{introduction-definition-skew_symmetric_tensors}
    A tensor $T \in V^{\otimes d}$ which is invariant under the skew-symmetric action of $\frakS_d$ is said to be \emph{skew symmetric}. Let $\Wedge^d V \subseteq V^{\otimes d}$ denote the space of skew-symmetric tensors. 
\end{definition}

\section{Group actions, restrictions and degenerations}
\label{introduction-section-groupactions}

\begin{definition}
\label{introduction-definition-orbitsdegenerations}
Let $G$ be a group acting on a space $V$ and let $v \in V$. The $G$-orbit of $v$ is 
\[
\Omega_v = \{ g(v) : g \in G \}.
\]
The $G$-orbit-closure $\bOmega_v$ is the closure of $\Omega_v$. We say that $w$ is $G$-isomorphic to $v$ if $w \in \Omega_v$. We say that $w$ is a degeneration of $v$ if $w \in \bOmega_v$.

Typically $V$ is a vector space, but the definition applies more generally with $V$ any topological space.
\end{definition}

If $G$ is a complex algebraic group acting algebraically on an algebraic variety $V$ (typically an affine space) then the closure defining $\bOmega_v$ can be taken equivalently in the Zariski or the Euclidean topology, see, e.g., \cite[Thm. 2.33]{Mum76}.

One can give an apparently different definition of degeneration, in terms of formal power series. Let $\bbC[[\eps]]$ be the ring of formal power series in one variable $\eps$, and let $\bbC((\eps))$ denote its quotient field, that is the field of Laurent series in $\eps$. For a vector space $V$ over $\bbC$, let $V^{[\eps]} = V \otimes_{\bbC} \bbC((\eps))$, which is a $\bbC((\eps))$-vector space. If $G$ acts on $V$, then the action extends to the action of a group $G^{[\eps]}$ on $V^{[\eps]}$: when $G$ is algebraic, then $G^{[\eps]}$ is the group of the $\bbC((\eps))$-points of $G$. 
\begin{definition}
\label{introduction-definition-formaldegenerations}

Let $G$ be a linear algebraic group acting linearly on a vector space $V$. Let $v,w \in V$. We say that $w$ is a \emph{formal $G$-degeneration} of $v$ if there exists an element $g_\eps \in G^{[\eps]}$ such that
\[
g_\eps \cdot v = w + \eps w_1 + \cdots .
\]
\end{definition}
In contrast with \ref{introduction-definition-formaldegenerations}, the degeneration defined in \ref{introduction-definition-orbitsdegenerations} is sometimes called \emph{topological degeneration}. It turns out that the two definitions are equivalent.
\begin{theorem}
\label{introduction-theorem-degenerationsequivalence}

Let $G$ be a linear algebraic group acting linearly on a space $V$. Let $v,w \in V$. The following are equivalent:
\begin{itemize}
 \item $w$ is a (topological) $G$-degeneration of $v$;
 \item $w$ is a formal $G$-degeneration of $v$.
\end{itemize}
\end{theorem}
\begin{proof}
The proof is given in \cite[Sec.20.6]{BCS97} in the case of a product of general linear groups acting on a tensor space. In \cite[Sec.2.3]{Kra84}, the proof is given for the action of $\GL(V)$ on an arbitrary vector space. A sketch of the general proof is given in \cite[Rmk.4.4]{CGZ23}.
\end{proof}

\section{Decomposable tensors and classical algebraic varieties}
\label{introduction-section-decomposable_tensors}

\begin{definition}[Decomposable tensor]\label{introduction-definition-decomposable_tensor}
    In $V_1\ootimes V_d$, tensors that are products of vectors, e.g., $T = v_1 \ootimes v_d$, are called \emph{decomposable tensors} or \emph{rank-one tensors}.
\end{definition}
Decomposable tensors parametrize classical algebraic varieties. Since the property of being decomposable is clearly invariant under scalar multiples, these algebraic varieties are naturally defined inside \emph{projective spaces}.

We follow the usual notation $[v] \in \bbP V$ for the projective point corresponding to a vector $v \in V \smallsetminus 0$.

\begin{definition}[Segre variety]
\label{introduction-definition-Segre}

Let $V_1 \vvirg V_d$ be vector spaces. The \emph{Segre embedding} is the algebraic map defined by 
\begin{align*}
\Seg: \bbP V_1 \ttimes \bbP V_d &\to \bbP (V_1 \ootimes V_d) \\
([v_1] \vvirg [v_d]) &\mapsto [v_1 \ootimes v_d].
\end{align*}
The image of the Segre embedding is the \emph{Segre variety} of rank-one tensors:
\[
\Seg( \bbP V_1 \ttimes \bbP V_d) =\{ [v_1 \ootimes v_d] : v_j \in V_j\}.
\]
\end{definition}
When $d = 2$, the Segre variety is the variety of rank-one matrices in $\bbP (V_1 \otimes V_2)$.

\begin{definition}[Veronese variety]
\label{introduction-definition-Veronese}

Let $V$ be vector spaces. The \emph{$d$-th Veronese embedding} is the algebraic map defined by 
\begin{align*}
    \nu_d : \bbP V &\to \bbP S^d V \\
    [v] &\mapsto [v^{\otimes d}].
\end{align*}
The image of the Veronese embedding is the \emph{Veronese variety}:
\[
    \nu_d( \bbP V ) =\{ [v^{\otimes d}] :  v \in V\}.
\]
Recall that the symmetric tensors can be identified with homogeneous polynomials. Hence, the Veronese variety corresponds to the variety parametrized by $d$-th powers of linear forms in the projective space of degree-$d$ polynomials. 

When $\dim V = 2$, then $\bbP V = \bbP^1$ and $\nu_d( \bbP^1)$ is the \emph{rational degree-$d$ normal curve} in $\bbP S^d V = \bbP^d$.
\end{definition}

\begin{definition}[Segre-Veronese variety]
\label{introduction-definition-SegreVeronese}

Let $V_1 \vvirg V_m$ be vector spaces. The \emph{Segre-Veronese embedding} of multidegree $\underline{d} = (d_1 \vvirg d_m)$ is the algebraic map defined by 
\begin{align*}
\nu_{\underline{d}} : \bbP V_1 \ttimes \bbP V_m &\to \bbP S^{\underline{d}}(V_1,\ldots,V_m) \\
([v_1] \vvirg [v_m]) &\mapsto [v_1^{\otimes d_1} \ootimes v_m^{\otimes d_m}].
\end{align*}
The image of the Segre-Veronese embedding is the \emph{Segre-Veronese variety}:
\[
\nu_{\underline{d}}( \bbP V_1 \ttimes \bbP V_k ) = \{ [v_1^{\otimes d_1} \ootimes v_k^{\otimes d_k}] :  v_j \in V_j\}.
\]
\end{definition}
Note that the Segre variety is a Segre-Veronese variety of multidegree $(1 \vvirg 1)$, while the Veronese variety is a Segre-Veronese variety of multidegree $(d)$.

\begin{definition}[Grassmannian]\label{introduction-definition-Grassmannian}
    Let $V$ be a vector space. The \emph{Grassmannian} $\Gr(d,V)$ is the set of $d$-dimensional subspaces of $V$: this is identified with the set of decomposable skew-symmetric tensors in $\Wedge^dV$. The Grassmannian can be embedded as an algebraic subvariety of $\bbP \Wedge^dV$ via the \emph{Pl\"ucker embedding} given by 
    \begin{align*}
    \wedge_{d} : \bbP V \ttimes \bbP V &\to \bbP \Wedge^{d}V \\
    ([v_1] \vvirg [v_d]) &\mapsto [v_1 \wwedge v_d].
    \end{align*}
\end{definition}    

%%% AleO: to be added: these are orbit closures



\part{Geometry of Secant Varieties}
\label{part-geometry_secants}
We assume some familiarity with algebraic geometry at the level of a first graduate course. We refer to \cite{Sha94} for the properties of the ideal-varieties correspondence, to \cite{Har92} for the definitions of dimension and degree of an algebraic variety, and to \cite{CLO07} for more computational aspects. Occasionally, more advanced results from \cite{Mum76} and \cite{Vak24} might be used. We follow mainly \cite{Lan12} for the elements of the theory of tensors and \cite{BCCGO18}, \cite{CGO14} for the classical results on secant varieties and apolarity theory. We draw inspiration from \cite{Rus16} and \cite{IK99} for the geometry in projective space and the geometry of $0$-dimensional schemes, respectively.

Given vector spaces $V_1 \vvirg V_d$, a tensor is an element $T \in V_1 \ootimes V_d$. If $\dim V_j = n_j +1 $, then we say that $T$ is a tensor of order $d$ and format $(n_1+1 \vvirg n_d+1)$. 

\chapter{Classical varieties in the space of tensors}
\label{classicalAG-chapter-VarietiesTensorSpaces}
We start with some definitions of classical algebraic varieties in the space of tensors.
\begin{definition}
\label{classicalAG-definition-Segre}
Let $V_1 \vvirg V_d$ be vector spaces. The Segre embedding is the algebraic map defined by 
\begin{align*}
\Seg: \bbP V_1 \ttimes \bbP V_d &\to \bbP (V_1 \ootimes V_d) \\
(v_1 \vvirg v_d) &\mapsto v_1 \ootimes v_d.
\end{align*}
The image of the Segre embedding is the \emph{Segre variety} of rank one tensors:
\[
\Seg( \bbP V_1 \ttimes \bbP V_d) =\{ v_1 \ootimes v_d : v_j \in \bbP V_j\}.
\]
\end{definition}
When $d = 2$, the Segre variety is the variety of rank one matrices in $\bbP (V_1 \otimes V_2) = \bbP ( \Hom(V_1^* , V_2))$.

\begin{definition}
\label{classicalAG-definition-Veronese}
Let $V$ be vector spaces. The $d$-th Veronese embedding is the algebraic map defined by 
\begin{align*}
\nu_d : \bbP V &\to \bbP S^d V \\
v &\mapsto v^{\otimes d}.
\end{align*}
The image of the Veronese embedding is the \emph{Veronese variety}:
\[
\nu_d( \bbP V ) =\{ v^d :  v \in \bbP V\}.
\]
\end{definition}
When $\dim V= 2$, then $\bbP V = \bbP^1$. In this case $\nu_d( \bbP^1)$ is a curve in $\bbP S^d V = \bbP^d$: it coincides with the rational normal curve of degree $d$.

\begin{definition}
\label{classicalAG-definition-SegreVeronese}
Let $V_1 \vvirg V_k$ be vector spaces. The Segre-Veronese embedding of multidegree $(d_1 \vvirg d_k)$ s the algebraic map defined by 
\begin{align*}
\nu_{d_1 \vvirg d_k} : \bbP V_1 \ttimes \bbP V_k &\to \bbP (S^{d_1} V_1 \ootimes S^{d_k} V_k) \\
(v_1 \vvirg v_k) &\mapsto v_1^{\otimes d_1} \ootimes v_k^{\otimes d_k}.
\end{align*}
The image of the Segre-Veronese embedding is the \emph{Segre-Veronese variety}:
\[
\nu_{d_1 \vvirg d_k}( \bbP V_1 \ttimes \bbP V_k ) =\{ v_1^{\otimes d_1} \ootimes v_k^{\otimes d_k} :  v_j \in \bbP V_j\}.
\]
\end{definition}

Note that the Segre variety is a Segre-Veronese variety of multidegree $(1 \vvirg 1)$. The Veronese variety is a Segre-Veronese variety of multidegree $(d)$.



\chapter{Secant varieties}
\label{classicalAG-chapter-SecantVarieties}
\begin{definition}[Join of varieties]
\label{classicalAG-definition-join}
Let $X,Y \subseteq \mathbb{P}^N$ be two algebraic varieties. The {\it (geometric) join} of $X$ and $Y$ is 
\[
\bbJ(X,Y) =  \overline{ \{ p \in \mathbb{P}^N : p \in \langle x , y \rangle \text{ for some $x \in X, y \in Y$}\}}
\]
where the overline denotes the topological closure, equivalently in the Zariski or the Euclidean topology.

Given varieties $X_1 \vvirg X_s$, one can define their geometric join recursively
\[
\bbJ( X_1 \vvirg X_s) = \bbJ( \bbJ(X_1 \vvirg X_{s-1}), X_s).
\]
\end{definition}
There is a standard construction which realizes the join of varieties as the image of an \emph{incidence correspondence}. Given varieties $X_1 \vvirg X_s \subseteq \bbP^N$, consider the set
\[
A\bbJ^\circ(X_1 \vvirg X_s) = \{ (x_1 \vvirg x_s, p) \in X_1 \ttimes X_s \times \bbP^N : p \in \langle x_1 \vvirg x_s\rangle\}
\]
and let $A\bbJ(X_1 \vvirg X_s) = \overline{A\bbJ^\circ(X_1 \vvirg X_s)}$ be its closure in $X_1 \ttimes X_s \times \bbP^N$. The variety $A\bbJ(X_1 \vvirg X_s)$ is called the \emph{abstract join} of $X_1 \vvirg X_s$. The natural projection $A\bbJ(X_1 \vvirg X_s)  \to \bbP^N$ surjects onto $\bbJ(X_1 \vvirg X_s)$. On the other hand, the projection $A\bbJ(X_1 \vvirg X_s) \to X_1 \ttimes X_s$ realizes, locally, $A\bbJ^\circ(X_1 \vvirg X_s)$ as a $\bbP^{s-1}$-bundle over (and open subset of) $X_1 \ttimes X_s$. In particular, this shows that if $X_1 \vvirg X_s$ are irreducible, then $\bbJ(X_1 \vvirg X_s)$ is irreducible as well.


\begin{definition}[Secant variety]
\label{classicalAG-definition-secantvariety}
Let $X \subseteq \mathbb{P}^N$ be an algebraic variety. The $r$-th {\it secant variety} of $X$ is 
\[
\sigma_r(X) = \overline{\{ p \in \mathbb{P}^N : p \in \langle x_1 , \ldots ,  x_r \rangle \text{ for some $x_1 , \ldots , x_r \in X$}
\}}
\]
where the overline denotes the topological closure, equivalently in the Zariski or the Euclidean topology. Equivalently, $\sigma_r(X) = \bbJ(\underbrace{X \vvirg X}_{r \text{ times}})$.
\end{definition}

Similarly to the case of join, one can define an incidence correspondence realizing $\sigma_r(X)$ as the image of a projective morphism. Clearly, this can be done considering the abstract join $A\bbJ( X \vvirg X)$. However, it is often convenient to consider a symmetrized version of it. Let $X^{(\cdot r)} = X^{\times r} /\frakS_r$ be the symmetrized product of $r$ copies of $X$. Define 
\[
A\sigma^\circ_r(X) = \{ (Z , p  ) \in X^{(\cdot r)} \times \bbP^N : p \in \langle Z \rangle \text{ for some } Z \in X^{\cdot r}\}
\]
and let $A\sigma_r(X) = \overline{ A \sigma^\circ_r(X)}$. As in the case of joins, the projection of $A\sigma_r(X) \to \bbP^N$ surjects onto $\sigma_r(X)$ whereas the projection $A\sigma_r(X) \to X^{(\cdot r)}$ is, on an open set, a $\bbP^{r-1}$-bundle. In particular, if $X$ is irreducible, then $A\sigma_r(X)$ and $\sigma_r(X)$ are irreducible as well.

\begin{lemma}
\label{classicalAG-lemma-expecteddimension}
Let $X_1,X_2$ be two varieties in $\bbP^N$ with $\dim X_1 = n_1, \dim X_2 = n_2$. Then 
\[
\dim \bbJ(X_1,X_2) \leq \min\{ N, n_1 + n_2 + 1\}.
\]
In particular, if $X \subseteq \bbP^N$ is a variety with $\dim X = n$, then 
\[
\dim \sigma_r(X) \leq \min\{ N , rn + r-1\}.
\]
\end{lemma}
\begin{proof}
It is easy to verify that $\dim A \bbJ(X_1,X_2) =  n_1 + n_2 + 1$, which immediately yields the result.
\end{proof}
Given varieties $X_1,X_2 \subseteq \bbP^N$ with $\dim X_i = n_i$, we say that the join $\bbJ(X_1,X_2)$ has the expected dimension if $\dim \bbJ(X_1,X_2) =  \min\{ N, n_1 + n_2 + 1\}$ and we say that it is defective otherwise. Similar terminology is used in the case of secant varieties: we say that $\sigma_r(X)$ has the expected dimension if $\dim \sigma_r(X) = \min\{ N , r \dim X + r-1\}$; we say that $\sigma_r(X)$ is defective, or that $X$ is $r$-defective, if it does not have the expected dimension. 

\section{Dimension of secant varieties}
\label{classicalAG-section-dimension}

Throughout more than a century, a great deal of research was devoted in the study of dimension of secant varieties, and in the classification of defective varieties. A very first result in this direction is \ref{classicalAG-lemma-palatini} which appears already in \cite{Pal09}. 

\begin{lemma}[Palatini's Lemma]
\label{classicalAG-lemma-palatini}
Let $X \subseteq \bbP^N$ be an algebraic variety with $\codim X \geq 2$ and let $C \subseteq \bbP^N$ be a linearly non-degenerate algebraic curve. Then either $\dim \bbJ(X,C) = \dim X + 2$. In particular, all secant varieties of algebraic curves have the expected dimension.
\end{lemma}


A fundamental tool in the study of dimension of secant varieties is Terracini's Lemma, dating back to \cite{Ter11}.
\begin{lemma}[Terracini's Lemma]
\label{classicalAG-lemma-terracini}
Let $X, Y \subseteq \mathbb{P}^N$ be algebraic varieties and let $\bbJ(X,Y)$ be their join. Let $z \in \bbJ(X,Y)$ be a smooth point such that $z \in \langle x , y \rangle$ for smooth points $x \in X$ and $y \in Y$. Then
\[
T_{z} \bbJ(X,Y) = \langle T_x X , T_y Y\rangle.
\]
In particular 
\[
\dim \sigma_r(X) = \dim \langle T_{x_1} X \vvirg T_{x_r} X\rangle
\]
for generic points $x_1 \vvirg x_r \in X$.
\end{lemma}
This result offers a \emph{dual} point of view to the problem of determining the dimension of secant varieties via interpolation of fat points. Let $X$ be an algebraic variety and let $\calL$ be a (very ample) line bundle on $X$. Let $\phi_\calL : X \to \bbP H^0(X, \calL)^\vee$ be the embedding defined by $\calL$. 
\begin{proposition}
 \label{classicalAG-proposition-interpolationfatpoints}
 Let $X$ be an irreducible algebraic variety, $\calL$ a very ample line bundle on $X$. Then
 \[
 \dim \sigma_r( \phi_\calL(X)) = h^0(\calL)-1 - h^0( \calI_{Z} \otimes \calL)
 \]
 where $Z$ is the union of $r$ double points supported at generic points of $X$.
\end{proposition}







% \chapter{Defectivity results for secant varieties}
% \label{classicalAG-chapter-defectivity}
% 
% \chapter{Weak-defectivity results for secant varieties}
% \label{classicalAG-chapter-weakdefectivity}
% 
% \chapter{Apolarity theory}
% \label{classicalAG-chapter-apolarity}
% 
% \chapter{Alexander-Hirschowitz theorem}
% \label{classicalAG-chapter-AlexanderHirschowitz}
% 
% \chapter{AH type theorems for other varieties}
% \label{classicalAG-chapter-AHlike}
% 
% \chapter{Other notions of rank}
% \label{classicalAG-chapter-otherranks}


% \chapter{Strassen's additivity conjecture}
% \label{classicalAG-chapter-Strassenadditivity}
% 
% \chapter{Comon's conjecture}
% \label{classicalAG-chapter-Comon}



\part{Representation Theory}
\label{part-RepTheory}
We assume some background on representation theory, to the level of \cite[Ch.6]{Lan12}; for a detailed introduction, we refer to \cite{FH91}. We introduce some notation and important facts.

Let $\GL(V)$ denote the general linear group of a vector space $V$ and $\SL(V)$ denote the special linear group. Let $\frakS_d$ denote the symmetric group on $d$ elements. 

A representation of a group $G$ is a group homomorphism $\rho : G \to \GL(V)$ for some vector space $V$. We use the word \emph{representation} to refer to the space $V$ itself. 

A partition $\lambda = (\lambda_1 \vvirg \lambda_n)$ is a non-increasing sequence of positive integers. We say that $\lambda$ is a partition of $d$ if $\sum_i \lambda_i = d$, and we write $|\lambda| = d$; we say that $n$ is the length of $\lambda$ and we write $n = \ell(\lambda)$. Write $\lambda \partinto_n d$ to mean that $\lambda$ is a partition of $d$ of length at most $n$.

The (polynomial) irreducible representations of the general linear group of a vector space of dimension $n$ are indexed by partitions of length at most $n$. Let $\bbS_\lambda V$ be the irreducible representation associated to the partition $n$; this is the Schur module of $\GL(V)$ associated to $\lambda$.

The irreducible representations of the symmetric group $\frakS_d$ are indexed by partitions of $d$. Let $[\lambda]$ be the irreducible representation associated to the partition $\lambda$; this is the Specht module of $\frakS_d$ of type $\lambda$. 

The vector space $V^{\otimes d}$ is acted on naturally by $\frakS_d$, which permutes the tensor factors, and $\GL(V)$ which acts diagonally on all tensor factors; these two actions commute and the Schur-Weyl decomposition theorem expresses the spaces as a direct sum of irreducible representations for $\frakS_d \times \GL(V)$. Such decomposition is as follows:
\[
V^{\otimes d} = \bigoplus_{\lambda \partinto_n d} [\lambda] \otimes \bbS_{\lambda} V.
\]
A fundamental result which will prove useful numerous times is Schur's Lemma. Given $V,W$ representations for an algebraic group $G$, and a linear map $\phi : V \to W$, we say that $\phi$ is $G$-equivariant if it commutes with the action of $G$, that is $\phi( g \cdot v) = g \cdot \phi(v)$ for every $v \in V$, $g \in G$. The subset of $\Hom(V,W)$ consisting of $G$-equivariant maps is a linear subspace, and it is denoted by $\Hom_G(V,W)$. 
\begin{lemma}[Schur's Lemma]
\label{repTheory-lemma-Schur}
Let $G$ be a group, and let $\phi : V \to W$ be an equivariant map between two $G$-representations. Then $\ker(\phi)$ and $\image(\phi)$ are $G$-representations. In particular, if $V$ is irreducible, then $\phi = 0$ or $\phi$ is injective. If $V= W$ then $\phi = \lambda \Id_V$ for some $\lambda \in \bbC$, that is $\dim \Hom _G(V,W) = 1$.
\end{lemma}

By \ref{repTheory-lemma-Schur}, the Schur-Weyl decomposition shows $\dim \Hom_{\GL(V)} ( \bbS_\lambda V, V^{\otimes d}) = \dim [\lambda]$. More generally, a fundamental problem in representation theory and algebraic combinatorics consists in determining the decomposition of the tensor product of two (or more) irreducible representations. For representations of the general linear group we have the following:
\begin{definition}[Littlewood-Richardson coefficients]
 \label{RepTheory-definition-LRcoefficient}
 Let $\lambda,\mu,\nu$ be three partitions. The {\it Littlewood-Richardson coefficient} associated to $(\lambda,\mu;\nu)$ is
 \[
c^\nu_{\lambda,\mu} = \dim \Hom_{\GL(V)} (\bbS_\nu V,  \bbS_\lambda V \otimes \bbS_\mu V ).
\]
\end{definition}
Via \ref{repTheory-lemma-Schur}, the Littlewood-Richardson coefficient $c^\nu_{\lambda,\mu} $ coincides with the multiplicity of $\bbS_\nu V$ as a subrepresentation of $ \bbS_\lambda V \otimes \bbS_\mu V $, that is 
\[
 \bbS_\lambda V \otimes \bbS_\mu V  = \bigoplus_{\nu } (\bbS_{\nu} V )^{\oplus c^\nu_{\lambda,\mu} }.
\]
In particular, if $c^\nu_{\lambda,\mu} \neq 0$, then $|\nu| = |\lambda|+|\mu|$.

Littlewood-Richardson coefficients are hard to compute \cite{Nar06} even though there are polynomial-time algorithms to decide whether they are zero or nonzero as observed in \cite{MNS12} following \cite{KT99}. However, there are several special cases for which they are easy to determine. For instance, when $\mu$ is a partition consisting of a single row of a single column, we have the following classical result.
\begin{proposition}[Pieri's rule]
 Let $\lambda$ be a partition and let $d \geq 0$. Then 
 \[
 \bbS_{\lambda} V \otimes S^d V = \bigoplus_{\nu \in R} \bbS_{\nu} V, \qquad  \bbS_{\lambda} V \otimes \Lambda^d V = \bigoplus_{\nu \in C} \bbS_{\nu} V.
 \]
Here $R$ (resp. $C$) is the set of partitions of $|\lambda|+d$ obtained from $\lambda$ by adding $d$ boxes, no two of them on the same column (resp. row).
\end{proposition}

\chapter{Equations via representation theory}
\label{RepTheory-chapter-equations}
\section{$G$-varieties and equations}
 \label{repTheory-section-Gvarieties}
% Author: Fulvio Gesmundo
In a number of settings, algebraic varieties of interest are invariant under the action of an algebraic group. In these cases, representation theory, and in particular \ref{introduction-lemma-Schur} can be of great help in the study of equations for the variety of interest.

\begin{definition}
 \label{repTheory-definition-Gvariety}
% Author: Fulvio Gesmundo
 Let $G$ be a group and let $V$ be a $G$-representation. We say that a variety $X \subseteq \bbP V$ is a $G$-variety if $X$ is invariant under the action of $G$, that is $G \cdot X = X$. 
\end{definition}
We point out that the points of a $G$-variety $X$ are not in general fixed by the action of $G$. If this is the case, we say that $X$ is point-wise invariant by the action of $G$.

The Segre-Veronese varieties $\nu_{d_1 \vvirg d_k} ( \bbP V_1 \ttimes \bbP V_k)$ of \ref{introduction-definition-SegreVeronese}, their secant varieties (see \ref{geometrySecants-chapter-BasicDefinitions}), tangential varieties, joins of such, and more generally varieties constructed \emph{functorially} from them are $G$-varieties, with $G = \GL(V_1) \ttimes \GL(V_d)$. Orbit-closures for the action of a group $G$, in the sense of \ref{introduction-definition-orbitsdegenerations}, are $G$-variaties as well.

The action of a group $G$ on a space $V$ induces an action of the polynomial ring $\bbC[V]$ via pull-back: given $F \in \bbC[V]$, $g \in G$, one has 
\[
g \cdot F = F \circ g^{-1}.
\]
If the action is linear, it restricts to the homogeneous components of $\bbC[V] = \bigoplus_{d \geq 0} S^d V^*$. If $X$ is a $G$-variety, then the homogeneous components of its ideal are $G$-representations as well.
\begin{lemma}
\label{repTheory-lemma-GactionOnIdeal}
% Author: Fulvio Gesmundo
 Let $X \subseteq \bbP V$ be a $G$-variety. Let $I(X) \subseteq \bbC[V]$ be the ideal of $X$. Then for every $g \in G$ and every $F \in \bbC[V]$, we have that $F \in I(X)$ if and only if $g \cdot F \in I(X)$. In particular, the homogeneous components $I(X)_d \subseteq  S^d V^*$ are subrepresentations of $G$.
\end{lemma}
Because of this result, every equation $F$ of a $G$-variety yields a \emph{module} of equations, that is the smallest $G$-subrepresentation containing $F$. The term equation is often used loosely, in the general sense of \emph{Zariski closed condition}.

In fact, it is often possible to realize the homogeneous components of the ideals of a $G$-variety as kernels (or images) of $G$-equivariant maps defined naturally.  A typical example is the one of equations arising from flattenings, discussed in \ref{RepTheory-chapter-flattenings}, which are particularly effective to find equations of secant varieties. We discuss here two other examples:
\begin{itemize}
 \item Kostant's Theorem, which determines equations for rational homogeneous varieties in purely representation theoretic terms;
 \item the Foulkes map, which appears in algebraic combinatorics and whose kernel describes the ideal of the Chow variety of completely reducible forms.
\end{itemize}

\section{Kostant's Theorem}
\label{RepTheory-section-kostant}
% Author: Fulvio Gesmundo
We say that a $G$-variety $X$ is \emph{homogeneous} for the action of $G$ if the group $G$ acts transitively on $X$. Important examples of homogeneous varieties include the varieties of rank one elements introduced in \ref{introduction-section-decomposable_tensors}. In fact, those varieties are examples of rational homogeneous varieties.
\begin{definition}
 \label{reptheory-definition-rationalhomogeneousvariety}
 Let $X \subseteq \bbP V$ be a $G$-variety, with $V$ irreducible representation of a semisimple algebraic group $G$. We say that $X$ is a {\it rational homogeneous variety} if $X$ is the $G$-orbit of an element $p \in \bbP V$.
\end{definition}
A rational homogeneous variety $X \subseteq \bbP V$ is uniquely characterized by the highest weight $\lambda$ of the $G$-representation $V$. Indeed, if $v \in V$ is a highest weight vector, then it is easy to see $X = G \cdot [v]$. 

\begin{example}
 \label{reptheory-example-RHV}
% Author: Fulvio Gesmundo
 The Segre-Veronese variety $X = \nu_{d_1 \vvirg d_k} (\bbP V_1 \ttimes \bbP V_k) \subseteq \bbP (S^{d_1} V_1 \ootimes S^{d_k}V_k)$ is a rational homogeneous variety under the action of $G = \GL(V_1) \ttimes \GL(V_k)$. Indeed, up to the choice of a torus in $\GL(V_i)$ any vector $v_i \in V_i$ can be regarded as a highest weight vector: then $X = G \cdot (v_1^{d_1} \ootimes v_k^{d_k})$. 
\end{example}

By \ref{repTheory-lemma-GactionOnIdeal}, the ideal of a rational homogeneous variety is a $G$-representation $I(X) \subseteq \Sym( V^*)$. Kostant's Theorem characterizes such ideal:
\begin{theorem}
 \label{reptheory-theorem-kostant}
% Author: Fulvio Gesmundo
Let $G$ be a semisimple algebraic group and let $V_\lambda$ be the irreducible $G$-representation of highest weight $\lambda$. Let $X \subseteq \bbP V_\lambda$ be the rational homogeneous variety in $\bbP V_\lambda$. Then, for every $d \geq 1$, 
\[
I(X)_d = V_{d\lambda}^\perp \subseteq S^d V_\lambda^*.
\]
Moreover, $I(X)_2$ generates the ideal $I(X)$.
\end{theorem}
We refer to \cite[Ch.16]{Lan12}

\section{The Foulkes map and the ideal of the Chow variety}
\label{RepTheory-section-foulkes}
% Author: Fulvio Gesmundo

Let $V$ be a vector space. The Chow variety of completely reducible forms of degree $d$ is 
\[
\Ch^{d}(V) = \{ \ell_1 \cdots \ell_d : \ell_i \in V \} \subseteq \bbP S^d V.
\]
It is not hard to see that $\Ch^d(V)$ is an Zariski-closed. It is evidently closed under the action of $\GL(V)$, hence it is a $\GL(V)$-variety. If $d \leq \dim V -1$, then 
\[
\Ch^d(V) = \overline{\GL(V) \cdot (x_1 \cdots x_d)}
\]
where $x_1 \vvirg x_d$ are $d$ vectors of $V$ in general linear position; in particular if $d\leq \dim V$ they are linearly independent. The Chow variety is the image of the Segre variety $\Seg(\bbP V \ttimes \bbP V) \subseteq \bbP V^{\otimes d}$ under the projection onto the fully symmetric component $S^d V$.

Since $\Ch^d(V)$ is a $\GL(V)$-variety, by \ref{repTheory-lemma-GactionOnIdeal}, the homogeneous components $I(\Ch^d(V))_e \subseteq S^e S^d V^*$ of the ideal of the Chow variety are subrepresentations of $ S^e S^d V^*$. These representations are hard to understand and they are related to long standing problems in representation theory and combinatorics. In particular, the components $I(\Ch^d(V))_e$ can be described in terms of the kernel of a natural map:
\begin{definition}
 \label{RepTheory-definition-foulkesmap}
% Author: Fulvio Gesmundo
 Let $d,e$ be positive integers. Index the tensor factors of $V^{\otimes de}$ with pairs $(i,j) \in [d]\times [e]$. Let $\pi_{d,e}$ be the composition
 \[
\bigotimes_{(i,j)} V_{(i,j)} \to (S^e V_{(1,\ast)}) \ootimes (S^eV_{(d,\ast)}) \to S^d S^e V  
 \]
 of the $\frakS_e$-symmetrization on every set of $e$ spaces with common first index, followed by the symmetrization on the $d$ resulting compies of $S^e V$. Let $\pi_{e,d}$ be the analogous composition obtained by reversing the two symmetrizations and write $S^eS^d V$ for its image. The Foulkes-Howe map is the composition
 \[
 \rmFH_{d,e} : S^e S^d V \to V^{\otimes de} \xto{\pi_{d,e}} S^d S^e V.
 \]
 of the embedding of $S^e S^d V = \image(\pi_{e,d})$ into $V^{\otimes de}$ followed by the projection $\pi_{d,e}$.
 \end{definition}
This map was introduced by Hermite in the case $\dim V = 2$ \cite{Her56}, and it was observed that in that case it is an isomorphism; this result is called Hermite reciprocity today. Hadamard studied it in general and proved that it completely controls the ideal of the Chow varieties \cite{Had97}.
\begin{theorem}
 \label{RepTheory-theorem-foulkeskerChow}
% Author: Fulvio Gesmundo
Let $V$ be a vector space. Let $d,e$ be nonnegative integers. Then 
\[
I(\Ch^d (V)) _ e = \ker \rmFH_{d,e}.
\]
\end{theorem}
\begin{proof}
 We refer to \cite[Prop. 8.6.1.2]{Lan12}.
\end{proof}


In particular, one recovers Hermite's result:
\begin{corollary}[Hermite reciprocity]
\label{RepTheory-corollary-hermiteReciprocity}
% Author: Fulvio Gesmundo
If $\dim V = 2$, then the Foulkes-Howe map $\rmFH_{d,e}$ is an isomorphism.
\end{corollary}
\begin{proof}
By the Fundamental Theorem of Algebra, every binary form splits as product of linear forms. Therefore $\Ch^d (V) = \bbP S^d V$ if $\dim V = 2$, and $I(\Ch^d(V))_e = 0$ for every $e$. By \ref{RepTheory-theorem-foulkeskerChow}, we deduce that $\rmFH_{d,e}$ is injective for every $d,e$. Since domain and codomain have the same dimension, we conclude that it is an isomorphism.
\end{proof}

Hermite reciprocity for arbitrary fields is explained in \cite[Sec.~3.4]{AFPRW19} and \cite{RS21,MW22}.


Another immediate consequence of \ref{RepTheory-theorem-foulkeskerChow} is the following:
\begin{corollary}
\label{RepTheory-corollary-foulkesplethysmbound}
% Author: Fulvio Gesmundo
Let $V$ be a vector space with $\dim V =n$. Let $d,e$ be nonnegative integers and let $\lambda$ be a partition of $ed$ with $\ell(\lambda) \leq n$. If $a_\lambda(e,d) > a_\lambda(d,e)$ then $I(\Ch_d(V))_e \neq 0$.
\end{corollary}
A special case of \ref{RepTheory-corollary-foulkesplethysmbound} is the one where $\ell(\lambda) > d$. In this case, a consequence of Pieri's rule (see \ref{introduction-proposition-Pieri}) is that $a_\lambda(e,d) = 0$. Therefore the entire isotypic component of type $\lambda$ in $S^e S^d V^*$ is contained in $I(\Ch_d(V))_e$. In fact, this is also a consequence of the fact that, by construction, $\Ch_d(V)$, regarded as a subvariety of $V^{\otimes d}$, is contained in the subspace variety of tensors whose multilinear ranks are $d$, hence all modules in $S^eS^d V^*$ of type $\lambda$ with $\ell(\lambda) \geq d$ vanish on $\Ch_d(V)$.

\subsection{Inequalities between plethysm coefficients}
\label{RepTheory-subsection-plethysmInequalities}
% Author: Fulvio Gesmundo
Results on inequalities between different structure coefficients in representation theory are of interest in algebraic combinatorics. For plethysm coefficients, several results are known in the asymptotic setting. A fundamental conjecture is due to Foulkes:
\begin{conjecture}
 \label{RepTheory-conjecture-Foulkes}
% Author: Fulvio Gesmundo
 Let $e \geq d$ be nonnegative integers and let $\lambda$ be a partition of $ed$. Then $a_\lambda (e,d) \geq a_\lambda(d,e)$.
\end{conjecture}
A stronger conjecture was posed by Hadamard, predicting that $\rmFH_{d,e}$ is injective for $e \leq d$ and it is known by the Foulkes-Howe conejcture. Proving this stronger conjecture of a specific pair $(e,d)$ clearly proves \ref{RepTheory-conjecture-Foulkes} for the same pair. However, in \cite{MN05}, it was shown that Hadamard's conjecture is false: $\rmFH_{5,5}$ has nontrivial kernel. The situation is far from understood in general and a better understanding would shed light on structural properties of the ideal of Chow varieties.




\chapter{Flattening methods}
\label{RepTheory-chapter-flattenings}
A typical Zariski closed condition is the boundedness of the rank of a matrix. We refer to \emph{flattening method} for any method which yields modules of equations via rank condition on a matrix. More precisely, a \emph{generalized flattening} of a space $V$ is any (polynomial) map 
\[
\Flat : V \to \Hom ( E,F)
\]
for two vector spaces $E,F$. More generally, one can consider a section of a bundle $\calHom(\calE,\calF)$ where $\calE,\calF$ are two vector bundles over $\bbP V$, see e.g. \cite{EH88}.

We say that a module of equations for a variety $X$ arises from a flattening if, for every $x \in X$, $\Flat(x)$ is a matrix of rank at most $r_X$. The equations can be described explicitly as the minors of size $r_X+1$ of $\Flat(v)$, which are polynomials in the element $v \in V$.

Tautologically, the variety $\sigma_r( \bbP E \times \bbP F)$ of matrices of rank at most $r$ has (ideal-theoretic) equations arising from a flattening. Similarly, the standard flattenings introduced in \ref{introduction-section-representingtensors} are examples of flattening maps; it is a classical fact that they yield equations for Segre-Veronese varieties:
\begin{proposition}
% Author: Fulvio Gesmundo
\label{RepTheory-proposition-standardFlatrank1}
 Let $T \in \bbP (V_1 \ootimes V_k)$ be a tensor. Then $T \in \bbP V_1 \ttimes \bbP V_k$ belongs to the Segre variety if and only if all flattenings $T_i : V_i^* \to \bigotimes _{j \neq i} V_j$ have rank $1$.
\end{proposition}
A similar statement holds for Segre-Veronese varieties and it is discussed in \ref{RepTheory-section-flatteningsSecants} together with an extension to its secant varieties.

\section{Flattenings for secant varieties}
\label{RepTheory-section-flatteningsSecants}
% Author: Fulvio Gesmundo

When the flattening map $\Flat : V \to \Hom(E,F)$ is linear, then rank conditions for a variety $X$ yield rank conditions for the secant varieties of $X$. We state this formally following \cite[Prop. 4.1.1]{LO13}.
\begin{lemma}
 \label{RepTheory-lemma-LandsbergOttaviani411}
% Author: Fulvio Gesmundo
 Let $X \subseteq \bbP V$ be a variety and let $\Flat : V \to \Hom(E,F)$ be a linear map. Let 
 \[
 r_X = \max \{ r : \rank ( \Flat(x)) = r \text{ for some $x \in X$}\}.
 \]
If $p \in \sigma_r(X)$, then $\rank(\Flat(p)) \leq r \cdot r_X$. In particular, if $\rank(\Flat(p)) = s$, then $p \notin \sigma_r (X)$ for $r = \lceil s / r_X \rceil$.
\end{lemma}

\ref{RepTheory-proposition-standardFlatrank1} is a particular case of \ref{RepTheory-lemma-LandsbergOttaviani411}. We discuss it here in the case of Segre-Veronese varieties. In this setting, the standard flattenings defined in \ref{introduction-definition-flattenings} can be written as follows. Fix spaces $V_1 \vvirg V_k$ and integers $d_1 \vvirg d_k, e_1 \vvirg e_k$ with $e_i \leq d_i$; then there is a natural flattening map
\[
\begin{aligned}
\Flat_{(d_1 \vvirg d_k)}^{(e_1 \vvirg e_k)} : S^{d_1} V_1 \ootimes S^{d_k}V_k &\to \Hom ( S^{e_1} V_1^* \ootimes S^{e_k}V_k^* , S^{d_1 - e_1} V_1 \ootimes S^{d_k- e_k}V_k ) \\ 
f_1 \ootimes f_k &\mapsto \left[ \eta_1 \ootimes \eta_k \mapsto  \eta_1(f_1) \ootimes \eta_k(f_k) \right] .
\end{aligned}
\]
In the symmetric setting, the map $\Flat_e : S^e V^* \to S^{d-e} V$ is called \emph{catalecticant map} of $f$.

Then the following holds:
\begin{lemma}
\label{RepTheory-lemma-catalecticantSegreVeronese}
% Author: Fulvio Gesmundo
 Let $T \in  S^{d_1} V_1 \ootimes S^{d_k}V_k$. The following are equivalent:
 \begin{enumerate}[(i)]
  \item $T$ is an element of the Segre-Veronese variety $\nu_{d_1 \vvirg d_k} ( \bbP V_1 \ttimes \bbP V_k)$;
  \item for every $(e_1 \vvirg e_k)$, $\Flat_{(d_1 \vvirg d_k)}^{(e_1 \vvirg e_k)}(T)$ has rank one, as a linear map. 
 \end{enumerate}
In particular, the $2 \times 2$ minors of the standard flattenings give set-theoretic equations for $\nu_{d_1 \vvirg d_k} ( \bbP V_1 \ttimes \bbP V_k)$. Moreover, if $T \in \sigma_r(\nu_{d_1 \vvirg d_k} ( \bbP V_1 \ttimes \bbP V_k))$ then $\rk(\Flat_{(d_1 \vvirg d_k)}^{(e_1 \vvirg e_k)}(T)) \leq r$.
\end{lemma}
The converse of \ref{RepTheory-lemma-catalecticantSegreVeronese} in the case of higher secant varieties is false, except in particular cases. An important case is the one of the second secant variety: in this case the converse was proved in \cite[Theorem 5.1]{LM04} in the case of Segre variety, whereas in the setting of Veronese varieties it is classical; the results of \cite{BL14} and in particular the classification results for elements of the second secant variety allow one to extend it to every Segre-Veronese variety.
\begin{theorem}
 \label{RepTheory-theorem-LandsbergManivelSigma2}
% Author: Fulvio Gesmundo
 Let $T \in  S^{d_1} V_1 \ootimes S^{d_k}V_k$. The following are equivalent:
 \begin{enumerate}[(i)]
  \item $T \in \sigma_2(\nu_{d_1 \vvirg d_k} ( \bbP V_1 \ttimes \bbP V_k))$;
  \item for every $(e_1 \vvirg e_k)$, $\rank( \Flat_{(d_1 \vvirg d_k)}^{(e_1 \vvirg e_k)}(T) ) \leq 2$. 
 \end{enumerate} 
 In particular, $3 \times 3$ minors of the standard flattenings give set-theoretic equations for $ \sigma_2(\nu_{d_1 \vvirg d_k} ( \bbP V_1 \ttimes \bbP V_k))$.
\end{theorem}
Another particular case is the one of the rational normal curve, that is the case where $k = 1$ and $\dim V = 2$. The result is classical, see e.g. \cite[Chapter 7]{IK99}.
\begin{theorem}
 \label{RepTheory-theorem-rationalnormalcurves}
% Author: Fulvio Gesmundo
 Let $f \in  S^{d} V$ with $\dim V = 2$. The following are equivalent:
 \begin{enumerate}[(i)]
  \item $f \in \sigma_r(\nu_{d} \bbP V)$;
  \item for every $e$ with $0 \leq e \leq d$, $\rank( \Flat_d^e(f))  \leq r$. 
 \end{enumerate} 
 In particular, $(r+1) \times (r+1)$ minors of the standard flattenings give set-theoretic equations for $\sigma_r(\nu_d(\bbP^1))$.
\end{theorem}
More generally, only partial results are known. However, in \cite{LO13}, equations for secant varieties of Segre varieties are constructed via \ref{RepTheory-lemma-LandsbergOttaviani411} in a representation theoretic setting, with the introduction of more interesting flattening maps. We present the construction here in slightly broader generality. Let $G$ be a group and let $V$ be a $G$-representation. Suppose $E,F$ are $G$-representations with the property that $E^* \otimes F$ contains $V$ as a subrepresentation. The Young flattening of $V$, associated to the pair $(E,F)$ is the linear map $\YFlat^V_{E,F} : V \to \Hom( E,F)$ mapping the element $v \in V$ to the composition
\begin{align*}
E \xto{\otimes v} E \otimes V \to E \otimes E^* \otimes F \xto{\lrcorner \Id_E} F
\end{align*}
where the first map is tensorising by $v$, the second is the inclusion $V \subseteq E^* \otimes F$ and the third map is contraction by the identigy element $\Id_E \in E \otimes E^*$.

We record explicitly the construction of Koszul-Young flattenings for tensors of order three. In several settings, classical equations for secant varieties can be reinterpreted in terms of this construction: this is the case of the classical Aronhold invariant of ternaty cubics \cite[Prop. 4.4.7]{Stu93} and for Strassen's equations for higher secant varieties \cite{Str83,LO13}.

Let $V_1,V_2,V_3$ be vector spaces with $\dim V_i = n_i+1$. Fix $p \leq n_1$. For $T \in V_1 \otimes V_2 \otimes V_3$, let $T_1^{\wedge p} $ be the composition
\[
\Lambda^p V_1 \otimes V_2^*  \xto{\id_{\Lambda^p V_1} \boxtimes T} \Lambda^p V_1 \otimes V_1 \otimes V_3 \to \Lambda^{p+1} V_1 \otimes V_3.
\]
In this case, one has the following result.
\begin{proposition}
 \label{RepTheory-proposition-KoszulFlat}
% Author: Fulvio Gesmundo
 Let $T \in V_1 \otimes V_2 \otimes V_3$ be a tensor. If $T \in \bbP V_1 \times \bbP V_2 \times \bbP V_3$, then $\rank( T_1^{\wedge p}) = \binom{n}{p}$. In particular, if $\rank ( T_1^{\wedge p} ) \geq R$, then 
 \[
T \notin \sigma_r ( \bbP V_1 \times \bbP V_2 \times \bbP V_3) 
 \]
for $r = \Bigl\lfloor R / \binom{n}{p} \Bigr\rfloor$. In other words $\uR(T) \geq r+1$.
\end{proposition}



Over time, strong \emph{barriers} for flattening methods for secant varieties have been proved. This means that they can only provide equations for secant varieties up to a certain value $r$, much smaller than the value $r_{\gen}$ of the generic rank. These barriers are discussed in \cite{EGOW18} from the point of view of complexity theory. A geometric reason is provided in \cite{Gal17}, in conjunction with results on the value of generic and maximum cactus rank \cite{BR13,BBG19}. 




\section{Equations for Chow varieties}
\label{RepTheory-section-chowvarieties}
% Author: Fulvio Gesmundo

Let $V$ be a vector space. The Chow variety of completely reducible forms of degree $d$ is
\[
\Ch_d(V) = \{ \ell_1 \cdots \ell_d : \ell_i \in \bbP V\} \subseteq \bbP S^d V.
\]
It is not hard to see that $\Ch_d(V)$ is an algebraic variety. 

Note that if $\dim V = 2$, then $\Ch_d(V) = \bbP S^d V$. Brill \cite{Bri98} and Gordan \cite{Gor94} determined set-theoretic equations for $\Ch_d(V)$ which are now known as Brill's equations in terms of a flattening map. In this section, we describe Brill's equations following \cite{Lan12,Gua18}. In particular, they are based on the construction of a flattening map $S^d V \to \Hom(S^{d(d-1)} V^*, \bbS_{(d,d)} V )$, which is identically $0$ on $\Ch_d$. 

In order to describe Brill's map, we introduce the following notation. Let $\dim V = n+1$.
\begin{itemize}
 \item $\pi_{(d,d)}: S^d V \otimes S^d V \to \bbS_{(d,d)} V$ is the equivariant projection onto the component $ \bbS_{(d,d)}$ of $S^d V \otimes S^d V$ which is unique by \ref{introduction-proposition-Pieri}.
 \item For every $k$, and every $e$, define
 \[
 \begin{aligned}
 E_k : S^e V &\to S^{k} V \otimes S^{k(e-1)}V  \\
 f &\mapsto \sum_{\alpha \in \bbN^{n+1}, |\alpha| = k} \binom{k}{\alpha}^{-1} \bfx^\alpha \otimes \left( f^{k-1}\textstyle \frac{\partial^k}{\partial \bfx^\alpha} f \right)
\end{aligned}
 \]
 which is a polynomial map. Elements in the image of $E_k$ can be multiplied according to the natural product 
 \[
( S^a V \otimes S^{a(e-1)} ) \times ( S^b V \otimes S^{b(e-1)} )  \to ( S^{a+b} V \otimes S^{(a+b)(e-1)} ).
\]
 \item Let $\calP_d ( e_1 \vvirg e_d)$ be the polynomial expression of the power sum polynomial of degree $d$ in terms of the elementary symmetric functions $e_1 \vvirg e_d$. 
 \item Let 
 \[
 \begin{aligned}
 \calQ_{d} : S^d V \to S^d V \otimes S^{d(d-1)}V \\
 f \mapsto \calP_d ( E_1(f) \vvirg E_d(f)).
 \end{aligned}
 \]
\end{itemize}
Define the Brill map $\calB(f) \in \Hom ( S^{d(d-1)}V^* , \bbS_{(d,d)} V)$ associated to $f \in S^d V$ to be the composition
\[
\begin{array}{rcccl}
S^{d(d-1)}V^* &\to & S^{d(d-1)}V^* \otimes S^d V \otimes S^d V \otimes S^{d(d-1)} V & \to & \bbS_{(d,d)} V \\
\Delta &\mapsto &\Delta \otimes f \otimes Q_d(f)  & &\\
& & \Delta \otimes g \otimes h \otimes H &\mapsto & D(H) \cdot \pi_{(d,d)} (g \otimes h)
\end{array}
 \]






\chapter{Classification results}
\label{RepTheory-chapter-classifications}
This chapter collects orbit classification results for the action of algebraic groups on spaces of tensors. The classification is usually possible only when the representation structure is either \emph{finite} or \emph{tame}: respectively, this means that there are finitely many orbits, or the orbits are in one-to-one correspondence with finitely many families in finitely many parameters. These definitions can be made more precise following \cite{Vin79} and \cite{Gab72}.

The only tensor spaces with finitely many orbits are subspaces of $\bbC^2 \otimes \bbC^3 \otimes \bbC^6$ or its permutations. The list of the orbits, and a description of the degeneration poset, is given in \ref{RepTheory-orbitclassification-section-236}.

There are, instead, a few cases where the structure is tame:
\begin{itemize}
 \item $\GL_2 \times \GL_m \times \GL_n$ acting on $\bbC^2 \otimes \bbC^m \otimes \bbC^n$;
 \item $\GL_3 \times \GL_3 \times \GL_3$ acting on $\bbC^3 \otimes \bbC^3 \otimes \bbC^3$;
 \item $\GL_2 \times \GL_2 \times \GL_2 \times \GL_2$ acting on $\bbC^2 \otimes \bbC^2 \otimes \bbC^2 \otimes \bbC^2$.
\end{itemize}
The case of format $(2,m,n)$ dates back to Kronecker, see, e.g., \cite[Ch.XIII]{Gan59}. In general, orbit classification results can be achieved using classical representation theoretic methods \cite{Vin79}. More recent references include \cite{Nur00} for the case of format $(3,3,3)$ and \cite{CD12} for the case of format $(2,2,2,2)$.

% In the setting of symmetric tensors, there are two cases with finite representation type:
% \begin{itemize}
%  \item $\GL_n$ acting on $S^2 \bbC^n$;
%  \item $\GL_2$ acting on $S^3 \bbC^n$;
% \end{itemize}
% moreover, there are two cases with tame representation type:
% \begin{itemize}
%  \item $\GL_2$ acting on $S^4 \bbC^2$;
%  \item $\GL_3$ acting on $S^3 \bbC^3$.
% \end{itemize}
% 
% Invariant theoretic aspects are discussed in \ref{RepTheory-chapter-invariantTheory}.

\section{Normal forms for format $(2,3,6)$}
\label{RepTheory-orbitclassification-section-236}
% Author: Fulvio Gesmundo

We follow the presentation of \cite[Sec. 10.3]{Lan12}. Fix bases 
\begin{align*}
&a_0,a_1 \text { of } \bbC^2, \\
&b_0\vvirg b_2 \text{ of } \bbC^3, \\ 
&c_0\vvirg c_5 \text{ of } \bbC^6. \\ 
\end{align*}
The action of $\GL_2 \times \GL_3 \times \GL_6$ on $\bbC^2 \otimes \bbC^3 \otimes \bbC^6$ has $27$ orbits. We list representatives for these orbits highlighting their multilinear. 

\begin{itemize}
 \item Multilinear ranks $(0,0,0)$: 
 \[
  T_0 = 0;
 \]
\item Multilinear ranks $(1,1,1)$:
\[
T_1 = a_0 \otimes b_0 \otimes c_0;
\]
\item Multilinear ranks $(1,2,2)$:
\[
T_2 = a_0 \otimes b_0 \otimes c_0 + a_0 \otimes b_1 \otimes c_1;
\]
\item Multilinear ranks $(2,1,2)$:
\[
T_3 = a_0 \otimes b_0 \otimes c_0 + a_1 \otimes b_0 \otimes c_1;
\]
\item Multilinear ranks $(2,2,1)$:
\[
T_4 = a_0 \otimes b_0 \otimes c_0 + a_1 \otimes b_1 \otimes c_0;
\]
\item Multilinear ranks $(2,2,2)$:
\begin{align*}
T_5 &= a_0 \otimes b_0 \otimes c_1 + a_0 \otimes b_1 \otimes c_0 + a_1 \otimes b_0 \otimes c_0, \\
T_6 &= a_0 \otimes b_0 \otimes c_0 + a_1 \otimes b_1 \otimes c_1;
\end{align*}
\item Multilinear ranks $(2,2,3)$:
\begin{align*}
T_7 &= a_0 \otimes b_0 \otimes c_0 + a_0 \otimes b_1 \otimes c_1 + a_1 \otimes b_0 \otimes c_2, \\
T_8 &= a_0 \otimes b_0 \otimes c_0 + a_1 \otimes b_1 \otimes c_1 + (a_0 + a_1) \otimes (b_0 + b_1) \otimes c_2;
\end{align*}
\item Multilinear ranks $(2,2,4)$:
\[
T_9 = a_0 \otimes b_0 \otimes c_0 + a_0 \otimes b_1 \otimes c_1 + a_1 \otimes b_0 \otimes c_2 + a_1 \otimes b_1 \otimes c_3;
\]
\item Multilinear ranks $(1,3,3)$:
\[
T_{10} = a_0 \otimes (b_0 \otimes c_0 + b_1 \otimes c_1 + b_2 \otimes c_2)
\]
\item Multilinear ranks $(2,3,2)$:
\begin{align*}
T_{11} &= a_0 \otimes b_0 \otimes c_0 + a_0 \otimes b_1 \otimes c_1 + a_1 \otimes b_2 \otimes c_0, \\
T_{12} &= a_0 \otimes b_0 \otimes c_0 + a_1 \otimes b_1 \otimes c_1 + (a_0 + a_1) \otimes b_2 \otimes (c_0 + c_1);
\end{align*}
\item Multilinear ranks $(2,3,3)$:
\begin{align*}
T_{13} &= a_0 \otimes (b_0 \otimes c_1 + b_1 \otimes c_0) + a_1 \otimes (b_0 \otimes c_2 + b_2 \otimes c_0), \\
T_{14} &= a_0 \otimes (b_0 \otimes c_0 +  b_1 \otimes c_1) + a_1 \otimes b_2 \otimes c_2,\\
T_{15} &= a_0 \otimes (b_0 \otimes c_0 +  b_1 \otimes c_1 + b_2 \otimes c_2) + a_1 \otimes b_0 \otimes c_1,\\
T_{16} &= a_0 \otimes (b_0 \otimes c_0 +  b_1 \otimes c_1 + b_2 \otimes c_2) + a_1 \otimes (b_0 \otimes c_1 + b_1 \otimes c_2),\\
T_{17} &= a_0 \otimes (b_0 \otimes c_0 +  b_1 \otimes c_1) + a_1 \otimes (b_2 \otimes c_2 + b_0 \otimes c_1 ),\\
T_{18} &= a_0 \otimes b_0 \otimes c_0 +  (a_0+a_1) \otimes b_1 \otimes c_1 + a_1 \otimes (b_2 \otimes c_2);
\end{align*}
\item Multilinear ranks $(2,3,4)$:
\begin{align*}
T_{19} &= a_0 \otimes (b_0 \otimes c_0 + b_1 \otimes c_1 + b_2 \otimes c_3) + a_1 \otimes (b_0 \otimes c_1 + b_1 \otimes c_2), \\
T_{20} &= a_0 \otimes (b_0 \otimes c_0 +  b_1 \otimes c_2 + b_2 \otimes c_3) + a_1 \otimes b_0 \otimes c_1,\\
T_{21} &= a_0 \otimes (b_0 \otimes c_0 +  b_1 \otimes c_2 + b_2 \otimes c_3) + a_1 \otimes (b_0 \otimes c_1 +b_1 \otimes c_3),\\
T_{22} &= a_0 \otimes (b_0 \otimes c_0 +  b_1 \otimes c_2) + a_1 \otimes (b_0 \otimes c_1 +b_2 \otimes c_3),\\
T_{23} &= a_0 \otimes (b_0 \otimes c_0 +  b_1 \otimes c_1 + b_2 \otimes c_2) + a_1 \otimes (b_0 \otimes c_1 + b_1 \otimes c_2 + b_2 \otimes c_3);\\
\end{align*}
\item Multilinear ranks $(2,3,5)$:
\begin{align*}
T_{24} &= a_0 \otimes (b_0 \otimes c_0 + b_1 \otimes c_1 + b_2 \otimes c_2) + a_1 \otimes (b_1 \otimes c_3 + b_2 \otimes c_4), \\
T_{25} &= a_0 \otimes (b_0 \otimes c_0 +  b_1 \otimes c_1 + b_2 \otimes c_2) + a_1 \otimes (b_0 \otimes c_2 + b_1 \otimes c_3 + b_2 \otimes c_4);\\
\end{align*}
\item Multilinear ranks $(2,3,6)$:
\[
T_{26} = a_0 \otimes (b_0 \otimes c_0 +  b_1 \otimes c_1 + b_2 \otimes c_2) + a_1 \otimes (b_0 \otimes c_3 + b_1 \otimes c_4 + b_2 \otimes c_5).
\]
\end{itemize}

% $$
% \xymatrix{
%  & & & 26 & & & & \\
%  & & 25  & & & & & \\
% } 
% $$
% 
% The degeneration poset is described in [Figure to be inserted]. The proof of the relevant restrictions and degenerations is given by describing the explicit restrictions and degenerations between the tensors of the classification. The linear maps are given in bases; vectors for which the image is not specified are mapped to themselves.
% \begin{itemize}
%  \item $T_{26}$ restricts to $T_{25}$ via
%  \[
%  \begin{array}{l}
%   c_3 \mapsto c_{2}\\
%   c_4 \mapsto c_{3}\\
%   c_5 \mapsto c_{4}
%  \end{array}
% \] 
%  \item $T_{25}$ degenerates to $T_{24}$ via
%  \[
%  \begin{array}{lll}
%   a_1 \mapsto \eps a_1 & ~ & c_3 \mapsto \eps^{-1} c_3 \\
%   & & c_4 \mapsto \eps^{-1} c_4
%  \end{array}
% \] 
% \item $T_{25}$ restricts to $T_{23}$ via 
%  \[
%  \begin{array}{lll}
% b_1 \mapsto b_2 & ~&  c_1 \mapsto c_{2} \\
% b_2 \mapsto b_1 & ~& c_2 \mapsto c_{1} \\
% & & c_4 \mapsto c_{2}
%   \end{array}
% \] 
% \end{itemize}

\subsection{Proof of the classification}


Let $T \in V_1 \otimes V_2 \otimes V_3$ with $\dim V_1 =2, \dim V_2 = 3$ and $\dim V_3 = 6$. Let $(r_1,r_2,r_3)$ be the multilinear ranks of $T$. By \ref{preliminaries-lemma-multilinearranksineq}, we have the three inequalities
\[
r_1 \leq r_2r_3, \quad r_2 \leq r_1r_3, \quad r_3 \leq r_1r_2
\]
and $T$ is concise in $\bbC^{r_1} \otimes \bbC^{r_2} \otimes \bbC^{r_3} \subseteq V_1 \otimes V_2 \otimes V_3$. We deduce that the only possible multilinear ranks are the ones listed in the classification.

Clearly, if $r_i = 0$ for any $i$, then $r_i = 0$ for all $i$ and $T = 0$.

Suppose one of the multilinear ranks is $r_i = 1$. If $r_1 = 1$, then the inequalities guarantee $r_2 = r_3$. Using the action of $\GL(V_1)$, we obtain $T = a_0 \otimes M$ for some $M \in V_2 \otimes V_3$. Since $r_2 = r_3$, we may use the action of $\GL(V_2) \times \GL(V_3)$ to normalize $M = \sum_{0}^{r_2-1} b_i \otimes c_i$. For $r_2 = r_3 = 1,2,3$, these are cases $T_1,T_2,T_{10}$ of the classification.  Similarly, if $r_2 = 1$ (and $r_1 = r_3 = 2$) or $r_3  = 1$ (and $r_1 = r_2 = 2$), we obtain cases $T_3$ and $T_4$ of the classification, respectively.

If $r_1 = 2$, let $L_T = \bbP \Im( T: V_1^* \to V_2 \otimes V_3) \subseteq \bbP (V_2 \otimes V_3)$ be the projective line image of the first flattening. The rest of the proof is based on the position of $L_T$ with respect to the varieties of matrices of bounded rank in $\bbP (V_2 \otimes V_3)$.

{\it (to be completed)}

% \chapter{The GCT approach}
% \label{RepTheory-chapter-GCT}
% 
\chapter{Invariant theory}
\label{RepTheory-chapter-invariantTheory}



\part{Optimization Theory}
\label{part-optimization}

\chapter{Introduction}
\label{optimization-chapter-intro}

\part{Complexity Theory}
\label{part-complexitytheory}

\chapter{Introduction}
\label{complexitytheory-chapter-intro}

\part{Quantum Physics}
\label{part-quantumph}

\chapter{Introduction}
\label{quantumph-chapter-intro}

\part{Data Science}
\label{part-datascience}

\chapter{Introduction}
\label{datascience-chapter-intro}

\bibliographystyle{amsalpha}
\bibliography{tensor}
\end{document}
