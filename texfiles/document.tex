\documentclass[oneside]{book}
\usepackage{amsmath}
\usepackage{amssymb}

 \usepackage{verbatim}
 
 \usepackage[all]{xy}
 
 \xyoption{2cell}
 \UseAllTwocells
 
 
 \usepackage{lmodern}
 \usepackage[T1]{fontenc}
 

 \usepackage{hyperref}
 \usepackage{enumerate} 


 \usepackage{amsthm}
 
 \theoremstyle{plain}
 \newtheorem{theorem}{Theorem}[section]
 \newtheorem{corollary}[theorem]{Corollary} 
 \newtheorem{conjecture}[theorem]{Conjecture} 
 \newtheorem{proposition}[theorem]{Proposition}
 \newtheorem{lemma}[theorem]{Lemma}
 
 \theoremstyle{definition} 
 \newtheorem{definition}[theorem]{Definition}
 \newtheorem{example}[theorem]{Example}
 \newtheorem{remark}[theorem]{Remark}
 \newtheorem{question}[theorem]{Question}
 \newtheorem{algorithm}[theorem]{Algorithm}
 
 \numberwithin{equation}{subsection}

 \usepackage{amsthm}

 \newcommand{\xto}[1]{\xrightarrow{\phantom{a}{#1}{\phantom{a}}}}
\newcommand{\xbackto}[1]{\xleftarrow{\phantom{a}{#1}{\phantom{a}}}}

\newcommand{\vvirg}{ , \ldots , }
\newcommand{\ootimes}{ \otimes \cdots \otimes }
\newcommand{\ooplus}{ \oplus \cdots \oplus }
\newcommand{\ttimes}{ \times \cdots \times }
\newcommand{\bboxtimes}{ \boxtimes \cdots \boxtimes }
\newcommand{\wwedge}{ \wedge \dots \wedge}

\newcommand{\contract}{\rotatebox[origin=c]{180}{ \reflectbox{$\neg$} }}

\newcommand{\biglangle}{\bigl\langle}
\newcommand{\bigrangle}{\bigr\rangle}

\newcommand{\Wedge}{\bigwedge}

\newcommand{\textbigotimes}{{\textstyle \bigotimes}}
\newcommand{\textbigtimes}{{\textstyle \bigtimes}}
\newcommand{\textbinom}[2]{{\textstyle \binom{#1}{#2}}}

\newcommand{\textfrac}[2]{{\textstyle \frac{#1}{#2}}}

\newcommand{\bigboxtimes}{{\scalebox{1.2}{$\boxtimes$}}}

%Alphabet Shortcuts

%capital boldface
\newcommand{\bfA}{\mathbf{A}}
\newcommand{\bfB}{\mathbf{B}}
\newcommand{\bfC}{\mathbf{C}}
\newcommand{\bfD}{\mathbf{D}}
\newcommand{\bfE}{\mathbf{E}}
\newcommand{\bfF}{\mathbf{F}}
\newcommand{\bfG}{\mathbf{G}}
\newcommand{\bfH}{\mathbf{H}}
\newcommand{\bfI}{\mathbf{I}}
\newcommand{\bfJ}{\mathbf{J}}
\newcommand{\bfK}{\mathbf{K}}
\newcommand{\bfL}{\mathbf{L}}
\newcommand{\bfM}{\mathbf{M}}
\newcommand{\bfN}{\mathbf{N}}
\newcommand{\bfO}{\mathbf{O}}
\newcommand{\bfP}{\mathbf{P}}
\newcommand{\bfQ}{\mathbf{Q}}
\newcommand{\bfR}{\mathbf{R}}
\newcommand{\bfS}{\mathbf{S}}
\newcommand{\bfT}{\mathbf{T}}
\newcommand{\bfU}{\mathbf{U}}
\newcommand{\bfV}{\mathbf{V}}
\newcommand{\bfW}{\mathbf{W}}
\newcommand{\bfX}{\mathbf{X}}
\newcommand{\bfY}{\mathbf{Y}}
\newcommand{\bfZ}{\mathbf{Z}}

%small boldface
\newcommand{\bfa}{\mathbf{a}}
\newcommand{\bfb}{\mathbf{b}}
\newcommand{\bfc}{\mathbf{c}}
\newcommand{\bfd}{\mathbf{d}}
\newcommand{\bfe}{\mathbf{e}}
\newcommand{\bff}{\mathbf{f}}
\newcommand{\bfg}{\mathbf{g}}
\newcommand{\bfh}{\mathbf{h}}
\newcommand{\bfi}{\mathbf{i}}
\newcommand{\bfj}{\mathbf{j}}
\newcommand{\bfk}{\mathbf{k}}
\newcommand{\bfl}{\mathbf{l}}
\newcommand{\bfm}{\mathbf{m}}
\newcommand{\bfn}{\mathbf{n}}
\newcommand{\bfo}{\mathbf{o}}
\newcommand{\bfp}{\mathbf{p}}
\newcommand{\bfq}{\mathbf{q}}
\newcommand{\bfr}{\mathbf{r}}
\newcommand{\bfs}{\mathbf{s}}
\newcommand{\bft}{\mathbf{t}}
\newcommand{\bfu}{\mathbf{u}}
\newcommand{\bfv}{\mathbf{v}}
\newcommand{\bfw}{\mathbf{w}}
\newcommand{\bfx}{\mathbf{x}}
\newcommand{\bfy}{\mathbf{y}}
\newcommand{\bfz}{\mathbf{z}}

%capital mathcal
\newcommand{\calA}{\mathcal{A}}
\newcommand{\calB}{\mathcal{B}}
\newcommand{\calC}{\mathcal{C}}
\newcommand{\calD}{\mathcal{D}}
\newcommand{\calE}{\mathcal{E}}
\newcommand{\calF}{\mathcal{F}}
\newcommand{\calG}{\mathcal{G}}
\newcommand{\calH}{\mathcal{H}}
\newcommand{\calI}{\mathcal{I}}
\newcommand{\calJ}{\mathcal{J}}
\newcommand{\calK}{\mathcal{K}}
\newcommand{\calL}{\mathcal{L}}
\newcommand{\calM}{\mathcal{M}}
\newcommand{\calN}{\mathcal{N}}
\newcommand{\calO}{\mathcal{O}}
\newcommand{\calP}{\mathcal{P}}
\newcommand{\calQ}{\mathcal{Q}}
\newcommand{\calR}{\mathcal{R}}
\newcommand{\calS}{\mathcal{S}}
\newcommand{\calT}{\mathcal{T}}
\newcommand{\calU}{\mathcal{U}}
\newcommand{\calV}{\mathcal{V}}
\newcommand{\calW}{\mathcal{W}}
\newcommand{\calX}{\mathcal{X}}
\newcommand{\calY}{\mathcal{Y}}
\newcommand{\calZ}{\mathcal{Z}}

%capital mathscr
\newcommand{\scrA}{\mathscr{A}}
\newcommand{\scrB}{\mathscr{B}}
\newcommand{\scrC}{\mathscr{C}}
\newcommand{\scrD}{\mathscr{D}}
\newcommand{\scrE}{\mathscr{E}}
\newcommand{\scrF}{\mathscr{F}}
\newcommand{\scrG}{\mathscr{G}}
\newcommand{\scrH}{\mathscr{H}}
\newcommand{\scrI}{\mathscr{I}}
\newcommand{\scrJ}{\mathscr{J}}
\newcommand{\scrK}{\mathscr{K}}
\newcommand{\scrL}{\mathscr{L}}
\newcommand{\scrM}{\mathscr{M}}
\newcommand{\scrN}{\mathscr{N}}
\newcommand{\scrO}{\mathscr{O}}
\newcommand{\scrP}{\mathscr{P}}
\newcommand{\scrQ}{\mathscr{Q}}
\newcommand{\scrR}{\mathscr{R}}
\newcommand{\scrS}{\mathscr{S}}
\newcommand{\scrT}{\mathscr{T}}
\newcommand{\scrU}{\mathscr{U}}
\newcommand{\scrV}{\mathscr{V}}
\newcommand{\scrW}{\mathscr{W}}
\newcommand{\scrX}{\mathscr{X}}
\newcommand{\scrY}{\mathscr{Y}}
\newcommand{\scrZ}{\mathscr{Z}}

%capital mathbb
\newcommand{\bbA}{\mathbb{A}}
\newcommand{\bbB}{\mathbb{B}}
\newcommand{\bbC}{\mathbb{C}}
\newcommand{\bbD}{\mathbb{D}}
\newcommand{\bbE}{\mathbb{E}}
\newcommand{\bbF}{\mathbb{F}}
\newcommand{\bbG}{\mathbb{G}}
\newcommand{\bbH}{\mathbb{H}}
\newcommand{\bbI}{\mathbb{I}}
\newcommand{\bbJ}{\mathbb{J}}
\newcommand{\bbK}{\mathbb{K}}
\newcommand{\bbL}{\mathbb{L}}
\newcommand{\bbM}{\mathbb{M}}
\newcommand{\bbN}{\mathbb{N}}
\newcommand{\bbO}{\mathbb{O}}
\newcommand{\bbP}{\mathbb{P}}
\newcommand{\bbQ}{\mathbb{Q}}
\newcommand{\bbR}{\mathbb{R}}
\newcommand{\bbS}{\mathbb{S}}
\newcommand{\bbT}{\mathbb{T}}
\newcommand{\bbU}{\mathbb{U}}
\newcommand{\bbV}{\mathbb{V}}
\newcommand{\bbW}{\mathbb{W}}
\newcommand{\bbX}{\mathbb{X}}
\newcommand{\bbY}{\mathbb{Y}}
\newcommand{\bbZ}{\mathbb{Z}}

%small mathbb
\newcommand{\bba}{\mathbb{a}}
\newcommand{\bbb}{\mathbb{b}}
\newcommand{\bbc}{\mathbb{c}}
\newcommand{\bbd}{\mathbb{d}}
\newcommand{\bbe}{\mathbb{e}}
\newcommand{\bbf}{\mathbb{f}}
\newcommand{\bbg}{\mathbb{g}}
\newcommand{\bbh}{\mathbb{h}}
\newcommand{\bbi}{\mathbb{i}}
\newcommand{\bbj}{\mathbb{j}}
\newcommand{\bbk}{\mathbb{k}}
\newcommand{\bbl}{\mathbb{l}}
\newcommand{\bbm}{\mathbb{m}}
\newcommand{\bbn}{\mathbb{n}}
\newcommand{\bbo}{\mathbb{o}}
\newcommand{\bbp}{\mathbb{p}}
\newcommand{\bbq}{\mathbb{q}}
\newcommand{\bbr}{\mathbb{r}}
\newcommand{\bbs}{\mathbb{s}}
\newcommand{\bbt}{\mathbb{t}}
\newcommand{\bbu}{\mathbb{u}}
\newcommand{\bbv}{\mathbb{v}}
\newcommand{\bbw}{\mathbb{w}}
\newcommand{\bbx}{\mathbb{x}}
\newcommand{\bby}{\mathbb{y}}
\newcommand{\bbz}{\mathbb{z}}

%capital mathfrak
\newcommand{\frakA}{\mathfrak{A}}
\newcommand{\frakB}{\mathfrak{B}}
\newcommand{\frakC}{\mathfrak{C}}
\newcommand{\frakD}{\mathfrak{D}}
\newcommand{\frakE}{\mathfrak{E}}
\newcommand{\frakF}{\mathfrak{F}}
\newcommand{\frakG}{\mathfrak{G}}
\newcommand{\frakH}{\mathfrak{H}}
\newcommand{\frakI}{\mathfrak{I}}
\newcommand{\frakJ}{\mathfrak{J}}
\newcommand{\frakK}{\mathfrak{K}}
\newcommand{\frakL}{\mathfrak{L}}
\newcommand{\frakM}{\mathfrak{M}}
\newcommand{\frakN}{\mathfrak{N}}
\newcommand{\frakO}{\mathfrak{O}}
\newcommand{\frakP}{\mathfrak{P}}
\newcommand{\frakQ}{\mathfrak{Q}}
\newcommand{\frakR}{\mathfrak{R}}
\newcommand{\frakS}{\mathfrak{S}}
\newcommand{\frakT}{\mathfrak{T}}
\newcommand{\frakU}{\mathfrak{U}}
\newcommand{\frakV}{\mathfrak{V}}
\newcommand{\frakW}{\mathfrak{W}}
\newcommand{\frakX}{\mathfrak{X}}
\newcommand{\frakY}{\mathfrak{Y}}
\newcommand{\frakZ}{\mathfrak{Z}}

%small mathfrak
\newcommand{\fraka}{\mathfrak{a}}
\newcommand{\frakb}{\mathfrak{b}}
\newcommand{\frakc}{\mathfrak{c}}
\newcommand{\frakd}{\mathfrak{d}}
\newcommand{\frake}{\mathfrak{e}}
\newcommand{\frakf}{\mathfrak{f}}
\newcommand{\frakg}{\mathfrak{g}}
\newcommand{\frakh}{\mathfrak{h}}
\newcommand{\fraki}{\mathfrak{i}}
\newcommand{\frakj}{\mathfrak{j}}
\newcommand{\frakk}{\mathfrak{k}}
\newcommand{\frakl}{\mathfrak{l}}
\newcommand{\frakm}{\mathfrak{m}}
\newcommand{\frakn}{\mathfrak{n}}
\newcommand{\frako}{\mathfrak{o}}
\newcommand{\frakp}{\mathfrak{p}}
\newcommand{\frakq}{\mathfrak{q}}
\newcommand{\frakr}{\mathfrak{r}}
\newcommand{\fraks}{\mathfrak{s}}
\newcommand{\frakt}{\mathfrak{t}}
\newcommand{\fraku}{\mathfrak{u}}
\newcommand{\frakv}{\mathfrak{v}}
\newcommand{\frakw}{\mathfrak{w}}
\newcommand{\frakx}{\mathfrak{x}}
\newcommand{\fraky}{\mathfrak{y}}
\newcommand{\frakz}{\mathfrak{z}}

%capital roman
\newcommand{\rmA}{\mathrm{A}}
\newcommand{\rmB}{\mathrm{B}}
\newcommand{\rmC}{\mathrm{C}}
\newcommand{\rmD}{\mathrm{D}}
\newcommand{\rmE}{\mathrm{E}}
\newcommand{\rmF}{\mathrm{F}}
\newcommand{\rmG}{\mathrm{G}}
\newcommand{\rmH}{\mathrm{H}}
\newcommand{\rmI}{\mathrm{I}}
\newcommand{\rmJ}{\mathrm{J}}
\newcommand{\rmK}{\mathrm{K}}
\newcommand{\rmL}{\mathrm{L}}
\newcommand{\rmM}{\mathrm{M}}
\newcommand{\rmN}{\mathrm{N}}
\newcommand{\rmO}{\mathrm{O}}
\newcommand{\rmP}{\mathrm{P}}
\newcommand{\rmQ}{\mathrm{Q}}
\newcommand{\rmR}{\mathrm{R}}
\newcommand{\rmS}{\mathrm{S}}
\newcommand{\rmT}{\mathrm{T}}
\newcommand{\rmU}{\mathrm{U}}
\newcommand{\rmV}{\mathrm{V}}
\newcommand{\rmW}{\mathrm{W}}
\newcommand{\rmX}{\mathrm{X}}
\newcommand{\rmY}{\mathrm{Y}}
\newcommand{\rmZ}{\mathrm{Z}}

%small roman
\newcommand{\rma}{\mathrm{a}}
\newcommand{\rmb}{\mathrm{b}}
\newcommand{\rmc}{\mathrm{c}}
\newcommand{\rmd}{\mathrm{d}}
\newcommand{\rme}{\mathrm{e}}
\newcommand{\rmf}{\mathrm{f}}
\newcommand{\rmg}{\mathrm{g}}
\newcommand{\rmh}{\mathrm{h}}
\newcommand{\rmi}{\mathrm{i}}
\newcommand{\rmj}{\mathrm{j}}
\newcommand{\rmk}{\mathrm{k}}
\newcommand{\rml}{\mathrm{l}}
\newcommand{\rmm}{\mathrm{m}}
\newcommand{\rmn}{\mathrm{n}}
\newcommand{\rmo}{\mathrm{o}}
\newcommand{\rmp}{\mathrm{p}}
\newcommand{\rmq}{\mathrm{q}}
\newcommand{\rmr}{\mathrm{r}}
\newcommand{\rms}{\mathrm{s}}
\newcommand{\rmt}{\mathrm{t}}
\newcommand{\rmu}{\mathrm{u}}
\newcommand{\rmv}{\mathrm{v}}
\newcommand{\rmw}{\mathrm{w}}
\newcommand{\rmx}{\mathrm{x}}
\newcommand{\rmy}{\mathrm{y}}
\newcommand{\rmz}{\mathrm{z}}

%small greek boldface
\newcommand{\bfalpha}{\boldsymbol{\alpha}}
\newcommand{\bfbeta}{\boldsymbol{\beta}}
\newcommand{\bfgamma}{\boldsymbol{\gamma}}
\newcommand{\bfdelta}{\boldsymbol{\delta}}
\newcommand{\bfepsilon}{\boldsymbol{\epsilon}}
\newcommand{\bfzeta}{\boldsymbol{\zeta}}
\newcommand{\bfeta}{\boldsymbol{\eta}}
\newcommand{\bftheta}{\boldsymbol{\theta}}
\newcommand{\bfiota}{\boldsymbol{\iota}}
\newcommand{\bfkappa}{\boldsymbol{\kappa}}
\newcommand{\bflambda}{\boldsymbol{\lambda}}
\newcommand{\bfmu}{\boldsymbol{\mu}}
\newcommand{\bfnu}{\boldsymbol{\nu}}
\newcommand{\bfxi}{\boldsymbol{\xi}}
\newcommand{\bfpi}{\boldsymbol{\pi}}
\newcommand{\bfrho}{\boldsymbol{\rho}}
\newcommand{\bfsigma}{\boldsymbol{\sigma}}
\newcommand{\bftau}{\boldsymbol{\tau}}
\newcommand{\bfphi}{\boldsymbol{\varphi}}
\newcommand{\bfchi}{\boldsymbol{\chi}}
\newcommand{\bfpsi}{\boldsymbol{\psi}}
\newcommand{\bfomega}{\boldsymbol{\omega}}

%capital greek boldface
\newcommand{\bfGamma}{\boldsymbol{\Gamma}}
\newcommand{\bfDelta}{\boldsymbol{\Delta}}
\newcommand{\bfTheta}{\boldsymbol{\Theta}}
\newcommand{\bfLambda}{\boldsymbol{\Lambda}}
\newcommand{\bfXi}{\boldsymbol{\Xi}}
\newcommand{\bfPi}{\boldsymbol{\Pi}}
\newcommand{\bfSigma}{\boldsymbol{\Sigma}}
\newcommand{\bfPhi}{\boldsymbol{\Phi}}
\newcommand{\bfPsi}{\boldsymbol{\Psi}}
\newcommand{\bfOmega}{\boldsymbol{\Omega}}

% Preferences
\newcommand{\eps}{\varepsilon}

\newcommand{\injects}{\hookrightarrow}
\newcommand{\surjects}{\twoheadrightarrow}
\newcommand{\backto}{\longleftarrow}
\newcommand{\dashto}{\dashrightarrow}


\newcommand{\pder}{\mathop{}\!\mathrm{\partial}}
\newcommand{\pbder}{\mathop{}\!\mathrm{\overline{\partial}}}
\newcommand{\diff}{\mathop{}\!\mathrm{d}}
\newcommand{\Lap}{\mathop{}\!\Delta}

\newcommand{\id}{\mathrm{id}}
\newcommand{\Id}{\mathrm{Id}}
%\newcommand{\wreath}{\scalebox{1}[.75]{\textbf{\descnode}}} %nice wreath product symbol

\newcommand{\Ad}{\mathrm{Ad}}
\newcommand{\ad}{\mathrm{ad}}

\newcommand{\corank}{\mathrm{corank}}
\newcommand{\rank}{\mathrm{rank}}
\newcommand{\gen}{\mathrm{gen}}
% Algebra
\newcommand{\image}{\mathrm{image}}  
\DeclareMathOperator{\Cat}{Cat}
\DeclareMathOperator{\codim}{codim}
\DeclareMathOperator{\coker}{coker}
\DeclareMathOperator{\Hom}{Hom}
\newcommand{\calHom}{\mathcal{H}om}
\DeclareMathOperator{\Supp}{Supp}
\DeclareMathOperator{\Tr}{Tr}
\DeclareMathOperator{\trace}{trace}
\DeclareMathOperator{\diag}{diag}
\DeclareMathOperator{\rk}{rk}
\DeclareMathOperator{\End}{End}
\DeclareMathOperator{\Pf}{Pf}
\DeclareMathOperator{\Sym}{Sym}
\DeclareMathOperator{\Aut}{Aut}
\DeclareMathOperator{\Inn}{Inn}
\DeclareMathOperator{\Out}{Out}
\DeclareMathOperator{\Spec}{Spec}
\DeclareMathOperator{\Proj}{Proj}
\DeclareMathOperator{\sgn}{sgn}
\DeclareMathOperator{\charact}{char}
\DeclareMathOperator{\Gal}{Gal}
\DeclareMathOperator{\Div}{Div}
\DeclareMathOperator{\ord}{ord}
\DeclareMathOperator{\Cl}{Cl}
\DeclareMathOperator{\Cox}{Cox}
\DeclareMathOperator{\Pic}{Pic}
\DeclareMathOperator{\Ann}{Ann}
\DeclareMathOperator{\rad}{rad}
\DeclareMathOperator{\soc}{soc}
\DeclareMathOperator{\tr}{tr}
\DeclareMathOperator{\socdeg}{socdeg}
\DeclareMathOperator{\socle}{socle}

\newcommand{\SO}{\mathrm{SO}}
\newcommand{\GL}{\mathrm{GL}}
\newcommand{\SL}{\mathrm{SL}}
\newcommand{\PGL}{\mathrm{PGL}}
\newcommand{\fraksl}{\mathfrak{sl}}
\newcommand{\frakgl}{\mathfrak{gl}}

\newcommand{\Bl}{\mathrm{Bl}}
\newcommand{\Flat}{\mathrm{Flat}}
\newcommand{\YFlat}{\mathrm{YFlat}}

\newcommand{\Seg}{\mathrm{Seg}}
\newcommand{\Gr}{\mathrm{Gr}}

\newcommand{\Hilb}{\mathrm{Hilb}}
\newcommand{\Gor}{\mathrm{Gor}}
\newcommand{\Ch}{\mathrm{Ch}}

\newcommand{\vol}{\mathrm{vol}}
% Categories

\DeclareMathOperator{\Mor}{Mor}
\newcommand{\frakMod}{\frakM\frako\frakd}
\newcommand{\frakTop}{\frakT\frako\frakp}
\newcommand{\frakSch}{\frakS\frakc\frakh}
\newcommand{\frakAb}{\frakA\frakb}
\newcommand{\frakGp}{\mathfrak{Gp}}
\newcommand{\frakSet}{\mathfrak{Set}}

% complexity classes
\newcommand{\bfVNP}{\mathbf{VNP}}
\newcommand{\bfVP}{\mathbf{VP}}
\newcommand{\bfNP}{\mathbf{NP}}

\newcommand{\bfVQP}{\mathbf{VQP}}

\DeclareMathOperator{\diam}{diam}
\DeclareMathOperator{\dist}{dist}
\DeclareMathOperator{\mult}{mult}

\newcommand{\rmFH}{\mathrm{FH}}
\newcommand{\rmHF}{\mathrm{HF}}
\newcommand{\rmHP}{\mathrm{HP}}


\newcommand{\supp}{\mathrm{supp}}
\newcommand{\slrk}{\mathrm{slrk}}
\newcommand{\aslrk}{\uwave{\mathrm{slrk}}}
\newcommand{\uslrk}{\underline{\mathrm{slrk}}}
\newcommand{\uR}{\underline{\rmR}}
\newcommand{\uQ}{\underline{\rmQ}}

\newcommand{\aQ}{\uwave{\rmQ}}
\newcommand{\cR}{\mathrm{cR}}
\newcommand{\ucR}{\underline{\cR}}



\newcommand{\perm}{\mathrm{perm}}
\newcommand{\MaMu}{\mathbf{MaMu}}
\newcommand{\Pow}{\mathrm{Pow}}
\newcommand{\IMM}{\mathrm{IMM}}

\DeclareMathOperator{\Res}{Res}
\DeclareMathOperator{\Ind}{Ind}
\DeclareMathOperator{\Lie}{Lie}


\newcommand{\Mat}{\mathrm{Mat}}
\newcommand{\bOmega}{\overline{\Omega}}
\newcommand{\bond}{\mathrm{bond}}
\newcommand{\bbond}{\underline{\mathrm{bond}}}

\newcommand{\deggeq}{\unrhd}

\DeclareMathOperator{\Stab}{Stab}
\newcommand{\MPS}{\mathcal{MPS}}
\newcommand{\TNS}{\mathcal{T\!N\!S}}


\newcommand{\partinto}{\vdash}




%%% Last Update: Feb 15, 2024


\begin{document}

\part{Introduction}
\label{part-introduction}
Tensors manifest in various forms across mathematics and numerous other fields. The roots of the study of tensors and their properties goes back by more than a century: for example, in 1850s, Sylvester was already considering {\it decompositions of binary forms as sums of powers} \cite{Syl51}; or, since the late XIX century, classical algebraic geometers were already studying {\it secant varieties} \cite{Sco08,Ter11}; or, in 1920s, Hitchcock already introduced {\it polyadic decompositions} of tensors \cite{Hit27}.

Considering the long story spanning multiple domains, there is are many beautiful and rich resources available for both beginners and experts and written in different languages with varying jargon. In particular, in recent years, we have witnessed a remarkable growth in literature pertaining to what can broadly be called \emph{geometry of tensors}. Scholars from diverse fields such as classical geometry, theoretical computer science, data science, and quantum physics gave their contribution, actively engaging in each other's open problems and questions. However, due to the varied languages used in these works, gaining a comprehensive overview of the field's (or better, the fields') current state presents a significant challenge. Just to mention some of them, we refer to \cite{Zak93,Ger96,IK99,Hac12,Lan12,CGO14,Rus16,BCCGO18,BBCMM19}. 

It was from this realization that the concept of a unifying portal was born. The \textbf{MultilinearVerse} wants to be an online platform conceived with the aim of serving as such a unifying resource with the possibility of growing in the future to cover all different facets of the subject and dynamically follow the advancement of state-of-the-art research.

The development of this portal started in early 2024, drawing considerable inspiration from The Stacks project \cite{Stacks}. The portal is built on the Gerby project, an online tag-based view for large \LaTeX~documents. The tag-based system ensures that every page, statement, and \emph{label} within the document is assigned its own unique tag, facilitating easy referencing. MultilinearVerse is not a book and it does not intend to replace classical resources. It has chapter, sections, and a linear numbering of its statements. Yet, its design encourages a tree-based approach to navigation, based on the mathematical connections between different topics, which are made readily apparent through various interlinks.

We do not plan that the portal will cover background material, to the level of a Master's degree in Mathematics. We will include more precise references on the background in the single sections of the portal. We acknowledge that the definition of \emph{background material} will be mostly arbitrary. After all, MultilinearVerse is a product of its developers, and necessarily reflects their mathematical strengths and weaknesses. Most of the geometry is limited to the complex setting, with a focus on results and methods coming from classical algebraic geometry. We expect and we hope that \emph{all} aspects of the multifaceted world of tensors will be covered eventually with the contribution of our rich community. 

\chapter{Preliminaries}
\label{introduction-chapter-intropreliminaries}
In this chapter, we introduce all basic notions, terminologies and notations that will be used through all the MultilinearVerse. 

\section{Representing tensors}
\label{introduction-section-representingtensors}

\begin{definition}[Tensor product]\label{introduction-definition-tensor_product}
Let $V,W$ be finite dimensional $\Bbbk$-vector spaces.
\begin{itemize}
    \item Let $\Hom(V,W)$ denote the vector space of linear functions from $V$ to $W$.
    \item Let $V^* = \Hom(V,\Bbbk)$ denote the \emph{dual space} of $V$.
    \item Let $V \otimes W$ denote the tensor product of $V$ and $W$: it can be equivalently identified with the space of linear maps $\Hom(V^*,W) \simeq \Hom(W^*,V)$ or with the space of bilinear map $V \times W \to \bbC$.
\end{itemize}
Given $\Bbbk$-vector spaces $V_1 \vvirg V_d$, let $V_1 \ootimes V_d$ denote their tensor product, which is identified with the space of multilinear maps $V_1^* \ttimes V_d^* \to \bbC$. If $V_1 = \cdots = V_d = V$, then we write for short $V^{\otimes d}$. 

The elements of a tensor product of $d$ vector spaces are called \emph{tensors} of \emph{order $d$}.

In coordinates, if $\{v_0,\ldots,v_m\}$ is a basis of $V$ and $\{w_0,\ldots,w_n\}$ is a basis of $W$, then $\{v_i \otimes w_j ~:~ i \in \{0,\ldots,m\}, j \in \{0,\ldots,n\}$ is a basis of $V\otimes W$. Therefore, any element $T \in V\otimes W$ can be written uniquely as 
\[
    T = \sum_{i,j} t_{i,j}v_i\otimes w_j;
\]
therefore, the choice of basis gives an identification of the space $V\otimes W$ with the space of $m\times n$ \emph{matrices} with coefficients in $\Bbbk$. In general, if $\{v_0^{(i)},\ldots,v_{n_i}^{(i)}\}$ is a basis of $V_i$, then any element $T \in V_1 \ootimes V_d$ can be written uniquely as 
\[
    T = \sum_{i_1,\ldots,i_d} t_{i_1,\ldots,i_d}v_{i_1}^{(1)}\ootimes v_{i_d}^{(d)};
\]
therefore, the choice of basis gives an identification of the space $V_1 \ootimes V_d$ with the space of $d$-dimensional arrays of size $n_1 \times\cdots\times n_d$. 
\end{definition}

\begin{definition}[Flattenings]\label{introduction-definition-flattenings}
For every $K \subseteq \{1 \vvirg d\}$, let $K^c = \{1\vvirg d\} \smallsetminus K$. Then, a tensor $T \in V_1 \ootimes V_d$ defines a linear map 
\[
    \Flat_K(T): \bigotimes_{k \in K} V_{k}^* \to  \bigotimes_{k' \in K^c} V_{k'}
\]
called \emph{flattening} of $T$. It is easy to see that $\Flat_K(T) = (\Flat_{K^c}(T))^\bft$. 
\end{definition}

\begin{definition}[Concise tensors]\label{introduction-definition-concise}
    A tensor $T$ is \emph{concise} if the flattenings $T_k : V_k^* \to \bigotimes_{k' \neq k} V_{k'}$ are injective. 
\end{definition}

A tensor $T \in V_1 \ootimes V_d$ can be identified with the image of its flattening $T : V_1^* \to V_2 \ootimes V_d$, up to a change of coordinates on the space $V_1$. Here, we consider the action of $\GL(V_1)$ on $V_1 \ootimes V_d$: for more about group actions, see \ref{introduction-section-groupactions}. In other words, if $T,T'$ are two tensors such that the image of their first flattening is the same subspace of $V_2 \ootimes V_d$, then there exists an element $g \in \GL(V_1)$ such that $T = g\cdot T'$. Similar statements hold with respect to the other simple flattenings.

\subsection{Symmetric Tensors}
\label{introduction-subsection-symmetric_tensors}

On the space of tensors $V^{\otimes d}$, we consider the action of the \emph{symmetric group} $\frakS_d$ which acts by permuting the tensor factors: if $T = (t_{i_1,\ldots,i_d}) \in V^{\otimes d}$, then, for any $\sigma \in \frakS_d$, then \[(\sigma \cdot T)_{i_1,\ldots,i_d} = t_{i_{\sigma(1)},\ldots,i_{\sigma(d)}}\] for any multi-index $(i_1,\ldots,i_d)$. 
\begin{definition}[Symmetric tensors]
    \label{introduction-definition-symmetric_tensors}

    A tensor $T \in V^{\otimes d}$ which is invariant under the action of $\frakS_d$ is said to be \emph{symmetric}. Let $S^d V \subseteq V^{\otimes d}$ denote the space of symmetric tensors. 

    If the base field has characteristic $0$, then $S^d V$ can be identified with the space of \emph{homogeneous polynomials} of degree $d$ on $V^*$. For example, the monomial $xy^2$ is identified with the symmetric tensor $\frac{1}{3}\left(x\otimes y \otimes y + y\otimes x \otimes y + y\otimes y \otimes x\right)$.
\end{definition}

\subsection{Partially Symmetric Tensors}
\label{introduction-subsection-partially_symmetric_tensors}

On the space of tensors $V_1^{\otimes d_1}\ootimes V_m^{\otimes d_m}$, we consider the action of the \emph{symmetric group} $\frakS_{d_1} \ttimes \frakS_{d_m}$ which acts by permuting the tensor factors accordingly to the partition: namely, if $T = T_1 \ootimes T_m \in V_1^{\otimes d_1}\ootimes V_m^{\otimes d_m}$, with $T_i \in V_i^{\otimes d_i}$, then, for any $\sigma = (\sigma_1,\ldots,\sigma_m)\in \frakS_{d_1} \ttimes \frakS_{d_m}$, then \[\sigma \cdot T = (\sigma_1\cdot T_1) \ootimes (\sigma_m\cdot T_m) \in V_1^{\otimes d_1}\ootimes V_m^{\otimes d_m},\] where each $\sigma_i$ acts permuting the factors of $T_i$ as explained in \ref{introduction-definition-symmetric_tensors}.
\begin{definition}[Partially symmetric tensors]
    \label{introduction-definition-partially_symmetric_tensors}
    
    A tensor $T \in V_1^{\otimes d_1}\ootimes V_m^{\otimes d_m}$ which is invariant under the action of $\frakS_{d_1} \ttimes \frakS_{d_m}$ is said to be \emph{partially symmetric}. Let $S^{d_1,\ldots,d_m}(V_1,\ldots,V_m) = S^{d_1} V_1 \ootimes S^{d_m}V_m \subseteq V_1^{\otimes d_1}\ootimes V_m^{\otimes d_m}$ denote the space of partially symmetric tensors. 

    If the base field has characteristic $0$, then $S^{d_1} V_1 \ootimes S^{d_m}V_m \subseteq V_1^{\otimes d_1}\ootimes V_m^{\otimes d_m}$ can be identified with the space of \emph{multi-homogeneous polynomials} of multi-degree $(d_1,\ldots,d_m)$ on $V^*_1 \ttimes V^*_m$. 
\end{definition}

Inside the space of tensors $V^{\otimes d}$, we have all possible spaces of partially symmetric tensors $S^{\underline{d}}(V) = S^{d_1} V \ootimes S^{d_m}V$ for all partitions $\underline{d} = (d_1,\ldots,d_m)\vdash d$. In particular, $S^{\underline{d}}(V) \subset S^{\underline{d'}}(V)$ if $\underline{d}'$ is a refinement of $\underline{d}$.

Fixed a basis $\{w_0,\ldots,w_m\}$ of $W$, any tensor $T \in W \otimes S^{d-1} V$ can be written as $T = \sum_{i=1}^m w_i \otimes T_i$ and then identified with the $m$-tuple $\{T_1,\ldots,T_m\}$ of elements of $S^{d-1} V$. This is also a set of generators for the image of the first flattening $\Flat_1(T)$, see \ref{introduction-definition-flattenings}.

For example, if $F \in S^d V$ is a symmetric tensor, with $\{v_0 \vvirg v_n\}$ basis of $V$, then $F$ is regarded as the element in $S^{(1,d-1)}(V)$ given by 
\[
    \sum_{j=0}^n v_j \otimes \textstyle\frac{\partial }{\partial v_j} F
\]
where the corresponding $(n+1)$-tuple of elements of $S^{d-1} V$ can be identified with the gradient of the homogeneous polynomial $F$.
\subsection{Skew-symmetric Tensors}\label{introduction-subsection-skew_symmetric_tensors}
On the space of tensors $V^{\otimes d}$, we consider the skew-symmetric action of the symmetric group $\frakS_d$: if $T = (t_{i_1,\ldots,i_d}) \in V^{\otimes d}$, then, for any $\sigma \in \frakS_d$, then $(\sigma \cdot T)_{i_1,\ldots,i_d} = \sgn(\sigma)t_{i_{\sigma(1)},\ldots,i_{\sigma(d)}}$ for any multi-index $(i_1,\ldots,i_d)$.
\begin{definition}[Skew symmetric tensors]\label{introduction-definition-skew_symmetric_tensors}
    A tensor $T \in V^{\otimes d}$ which is invariant under the skew-symmetric action of $\frakS_d$ is said to be \emph{skew symmetric}. Let $\Wedge^d V \subseteq V^{\otimes d}$ denote the space of skew-symmetric tensors. 
\end{definition}

\section{Group actions, restrictions and degenerations}
\label{introduction-section-groupactions}

\begin{definition}
\label{introduction-definition-orbitsdegenerations}
Let $G$ be a group acting on a space $V$ and let $v \in V$. The $G$-orbit of $v$ is 
\[
\Omega_v = \{ g(v) : g \in G \}.
\]
The $G$-orbit-closure $\bOmega_v$ is the closure of $\Omega_v$. We say that $w$ is $G$-isomorphic to $v$ if $w \in \Omega_v$. We say that $w$ is a degeneration of $v$ if $w \in \bOmega_v$.

Typically $V$ is a vector space, but the definition applies more generally with $V$ any topological space.
\end{definition}

If $G$ is a complex algebraic group acting algebraically on an algebraic variety $V$ (typically an affine space) then the closure defining $\bOmega_v$ can be taken equivalently in the Zariski or the Euclidean topology, see, e.g., \cite[Thm. 2.33]{Mum76}.

One can give an apparently different definition of degeneration, in terms of formal power series. Let $\bbC[[\eps]]$ be the ring of formal power series in one variable $\eps$, and let $\bbC((\eps))$ denote its quotient field, that is the field of Laurent series in $\eps$. For a vector space $V$ over $\bbC$, let $V^{[\eps]} = V \otimes_{\bbC} \bbC((\eps))$, which is a $\bbC((\eps))$-vector space. If $G$ acts on $V$, then the action extends to the action of a group $G^{[\eps]}$ on $V^{[\eps]}$: when $G$ is algebraic, then $G^{[\eps]}$ is the group of the $\bbC((\eps))$-points of $G$. 
\begin{definition}
\label{introduction-definition-formaldegenerations}

Let $G$ be a linear algebraic group acting linearly on a vector space $V$. Let $v,w \in V$. We say that $w$ is a \emph{formal $G$-degeneration} of $v$ if there exists an element $g_\eps \in G^{[\eps]}$ such that
\[
g_\eps \cdot v = w + \eps w_1 + \cdots .
\]
\end{definition}
In contrast with \ref{introduction-definition-formaldegenerations}, the degeneration defined in \ref{introduction-definition-orbitsdegenerations} is sometimes called \emph{topological degeneration}. It turns out that the two definitions are equivalent.
\begin{theorem}
\label{introduction-theorem-degenerationsequivalence}

Let $G$ be a linear algebraic group acting linearly on a space $V$. Let $v,w \in V$. The following are equivalent:
\begin{itemize}
 \item $w$ is a (topological) $G$-degeneration of $v$;
 \item $w$ is a formal $G$-degeneration of $v$.
\end{itemize}
\end{theorem}
\begin{proof}
The proof is given in \cite[Sec.20.6]{BCS97} in the case of a product of general linear groups acting on a tensor space. In \cite[Sec.2.3]{Kra84}, the proof is given for the action of $\GL(V)$ on an arbitrary vector space. A sketch of the general proof is given in \cite[Rmk.4.4]{CGZ23}.
\end{proof}

\section{Decomposable tensors and classical algebraic varieties}
\label{introduction-section-decomposable_tensors}

\begin{definition}[Decomposable tensor]\label{introduction-definition-decomposable_tensor}
    In $V_1\ootimes V_d$, tensors that are products of vectors, e.g., $T = v_1 \ootimes v_d$, are called \emph{decomposable tensors} or \emph{rank-one tensors}.
\end{definition}
Decomposable tensors parametrize classical algebraic varieties. Since the property of being decomposable is clearly invariant under scalar multiples, these algebraic varieties are naturally defined inside \emph{projective spaces}.

We follow the usual notation $[v] \in \bbP V$ for the projective point corresponding to a vector $v \in V \smallsetminus 0$.

\begin{definition}[Segre variety]
\label{introduction-definition-Segre}

Let $V_1 \vvirg V_d$ be vector spaces. The \emph{Segre embedding} is the algebraic map defined by 
\begin{align*}
\Seg: \bbP V_1 \ttimes \bbP V_d &\to \bbP (V_1 \ootimes V_d) \\
([v_1] \vvirg [v_d]) &\mapsto [v_1 \ootimes v_d].
\end{align*}
The image of the Segre embedding is the \emph{Segre variety} of rank-one tensors:
\[
\Seg( \bbP V_1 \ttimes \bbP V_d) =\{ [v_1 \ootimes v_d] : v_j \in V_j\}.
\]
\end{definition}
When $d = 2$, the Segre variety is the variety of rank-one matrices in $\bbP (V_1 \otimes V_2)$.

\begin{definition}[Veronese variety]
\label{introduction-definition-Veronese}

Let $V$ be vector spaces. The \emph{$d$-th Veronese embedding} is the algebraic map defined by 
\begin{align*}
    \nu_d : \bbP V &\to \bbP S^d V \\
    [v] &\mapsto [v^{\otimes d}].
\end{align*}
The image of the Veronese embedding is the \emph{Veronese variety}:
\[
    \nu_d( \bbP V ) =\{ [v^{\otimes d}] :  v \in V\}.
\]
Recall that the symmetric tensors can be identified with homogeneous polynomials. Hence, the Veronese variety corresponds to the variety parametrized by $d$-th powers of linear forms in the projective space of degree-$d$ polynomials. 

When $\dim V = 2$, then $\bbP V = \bbP^1$ and $\nu_d( \bbP^1)$ is the \emph{rational degree-$d$ normal curve} in $\bbP S^d V = \bbP^d$.
\end{definition}

\begin{definition}[Segre-Veronese variety]
\label{introduction-definition-SegreVeronese}

Let $V_1 \vvirg V_m$ be vector spaces. The \emph{Segre-Veronese embedding} of multidegree $\underline{d} = (d_1 \vvirg d_m)$ is the algebraic map defined by 
\begin{align*}
\nu_{\underline{d}} : \bbP V_1 \ttimes \bbP V_m &\to \bbP S^{\underline{d}}(V_1,\ldots,V_m) \\
([v_1] \vvirg [v_m]) &\mapsto [v_1^{\otimes d_1} \ootimes v_m^{\otimes d_m}].
\end{align*}
The image of the Segre-Veronese embedding is the \emph{Segre-Veronese variety}:
\[
\nu_{\underline{d}}( \bbP V_1 \ttimes \bbP V_k ) = \{ [v_1^{\otimes d_1} \ootimes v_k^{\otimes d_k}] :  v_j \in V_j\}.
\]
\end{definition}
Note that the Segre variety is a Segre-Veronese variety of multidegree $(1 \vvirg 1)$, while the Veronese variety is a Segre-Veronese variety of multidegree $(d)$.

\begin{definition}[Grassmannian]\label{introduction-definition-Grassmannian}
    Let $V$ be a vector space. The \emph{Grassmannian} $\Gr(d,V)$ is the set of $d$-dimensional subspaces of $V$: this is identified with the set of decomposable skew-symmetric tensors in $\Wedge^dV$. The Grassmannian can be embedded as an algebraic subvariety of $\bbP \Wedge^dV$ via the \emph{Pl\"ucker embedding} given by 
    \begin{align*}
    \wedge_{d} : \bbP V \ttimes \bbP V &\to \bbP \Wedge^{d}V \\
    ([v_1] \vvirg [v_d]) &\mapsto [v_1 \wwedge v_d].
    \end{align*}
\end{definition}    

%%% AleO: to be added: these are orbit closures



\part{Geometry of Secant Varieties}
\label{part-geometrySecants}
In this part, we approach the theory of tensor decompositions from the perspective of classical algebraic geometry. In particular, we will focus on the geometry of secant varieties. We refer mostly to \cite{Zak93}, \cite{IK99}, \cite{CGO14}, \cite{Rus16} and \cite{BCCGO18} for classical results on secant varieties, on the geometry of $0$-dimensional schemes and apolarity theory. 

We assume some familiarity with algebraic geometry at the level of a first graduate course. We refer to \cite{Sha94} for the properties of the ideal-varieties correspondence, to \cite{Har92} for the definitions of dimension and degree of an algebraic variety, and to \cite{CLO07} for more computational aspects. Occasionally, more advanced results from \cite{Mum76} and \cite{Vak24} might be used. 

\chapter{Secant varieties and Ranks}
\label{geometrySecants-chapter-BasicDefinitions}
 %%%%%%%%%%%%%%%%%%%%%%%%%%%%%%%%%%%%
 In this chapter we introduce the basic notions of secant varieties and ranks. 
 Let's go!
 \section{Secant varieties}
 \label{geometrySecants-section-secants}
 % Author: Alessandro Oneto
 
 \begin{definition}[Join of varieties]
 \label{geometrySecants-definition-join}
 % Author: Alessandro Oneto
 Let $X,Y \subseteq \mathbb{P}^N$ be two algebraic varieties. Consider the incidence correspondence 
 \[
     \itJoin^\circ(X,Y) = \{(x,y,p) ~:~ x \neq y,~p \in \langle x,y \rangle\} \subset X \times Y \times \bbP^N.
 \]
 This is a quasi-projective variety and its Zariski closure is the {\it abstract join} $\itJoin(X,Y) = \overline{\itJoin^\circ(X,Y)}$ of $X$ and $Y$. Consider the projections 
 \[
     \xymatrix{
         X \times Y \times \bbP^N \ar[d]^{\pi_1} \ar[r]^(0.65){\pi_2} & \bbP^N\\
         X \times Y & .
     }
 \]
 The \emph{join} of $X$ and $Y$ is the scheme-theoretic image 
 \[
     j(X,Y) = \pi_2(\itJoin(X,Y))
 \]
 This can also be regarded as the the closure of the union of all possible lines joining a point of $X$ and a point of $Y$, i.e.,
 \[
     j(X,Y) = \overline{ \bigcup_{\substack{x \in X, y \in Y \\ x \neq y}} \langle x,y \rangle} \subset \bbP^N.
     % \{ p \in \mathbb{P}^N : p \in \langle x , y \rangle \text{ for some $x \in X, y \in Y$}\}}
 \]
 The projection $\pi_1$ realizes, locally, $\itJoin(X,Y)$ as a $\bbP^1$-bundle over (an open subset) of $X \times Y$. In particular, if $X$ and $Y$ are irreducible, than $Join(X,Y)$ is irreducible and, consequently, $j(X,Y)$ is irreducible as well.
 
 Given varieties $X_1 \vvirg X_s \subseteq \bbP^N$, one can define their join recursively
 \[
     j( X_1 \vvirg X_s) = j( j(X_1 \vvirg X_{s-1}), X_s);
 \]
 geometrically this is 
 \[
     j( X_1 \vvirg X_s) = \overline{\bigcup_{\substack{x_1 \in X_1,\ldots,x_s \in X_s \\ \{x_1,\ldots,x_s\} \text{ indipendent}}} \langle x_1,\ldots,x_s \rangle} \subset \bbP^N.
 \]
 \end{definition}
 
 Secant varieties are obtained by consecutive joins of a variety with itself.
 \begin{definition}[Secant variety]
 \label{geometrySecants-definition-secantvariety}
 % Author: Alessandro Oneto
 Let $X \subseteq \mathbb{P}^N$ be an projective variety. The {\it $s$-th secant variety} of $X$ is 
 \[
 \sigma_s(X) = \overline{\bigcup_{\substack{x_1,\ldots,x_s \in X \\ \{x_1,\ldots,x_s\} \text{ indipendent}}} \langle x_1,\ldots,x_s \rangle} \subset \bbP^N.
 \]
 Recursively, $\sigma_1(X) = X$ and $\sigma_s(X) = j(X,\sigma_{s-1}(X))$.
 
 The abstract $s$-th secant variety of $X$ can be defined as the abstract join of $s$-copies of $X$ as defined above. However, if is often convenient to consider a symmetrized version. Let $X^{\cdot s} = X^{\times s} / \frakS_s$ be the symmetrized product of $s$ copies of $X$. Then, the {\it abstract $s$-th secant variety} of $X$ is 
 \[
     \Sec_s(X) = \overline{\left\{(\{x_1,\ldots,x_s\},p) ~:~ \substack{\{x_1,\ldots,x_s\} \text{ independent} \\ p \in \langle x_1,\ldots,x_s \rangle}\right\}} \subset X^s \times \bbP^N.
 \]
 Then, $\sigma_s(X) = \pi_2(\Sec_s(X))$, where $\pi_2 : X^s \times \bbP^N \rightarrow \bbP^N$ is the projection on the last factor.

 As explained in \ref{geometrySecants-definition-join} for join varieties, if $X$ is irreducible then all its secant varieties are irreducible.  
 \end{definition}
 
 %%%%%%%%%%%%%%%%%%%%%%%%%%%%%%%%%%%%
 \section{Rank and Border Rank}
 \label{geometrySecants-section-rank}
 % Author: Alessandro Oneto
 
 One of the main motivations to study secant varieties in relation to tensors is because they allow for a geometric definition of {\it rank} with respect to any projective variety and, in particular, to the ones defined in \ref{introduction-section-decomposable_tensors}. 
 
 \begin{definition}
 \label{geometrySecants-definition-Xrank}
 % Author: Alessandro Oneto
     Let $X \subseteq \bbP^N$ be a projective variety and let $p \in \bbP^N$. The \emph{$X$-rank} of $p$ is the smallest number of points of $X$ whose linear span contains $p$. In other words,
     \[
         \rank_X(p) = \min\{s ~:~ \exists x_1,\ldots,x_s \in X, ~ p \in \langle x_1,\ldots,x_s \rangle\}.
     \]
 \end{definition}
 
 It is natural to ask if the $X$-rank is \emph{finite} for any point of the ambient space. The first observation is that $X$ needs to be \emph{non-degenerate}, i.e., not contained in any proper linear subspace.

 \begin{lemma}
    \label{geometrySecants-lemma-secant_of_linear}
    % Author: Alessandro Oneto
    If $L \subset \bbP^N$ is a projective linear space, then $\sigma_s(L) = L$ for any $s$.
 \end{lemma}
 \begin{proof}
    If $L$ is linear, then for any $x_1,\ldots,x_s \in L$ we have $\langle x_1,\ldots,x_s \rangle \subset L$. In particular, $\sigma_s(L) \subset L$. The opposite inclusion is trivial.
 \end{proof}
 \begin{corollary}
    \label{geometrySecants-corollary-secant_of_degenerate_is_degenerate}
    % Author: Alessandro Oneto
    If $X$ is degenerate, i.e., $X \subset L \subsetneq \bbP^N$ for some linear space $L$, then $\sigma_s(X) \subset L$ for any $s$.
 \end{corollary}
 \begin{proof}
    By construction and by \ref{geometrySecants-lemma-secant_of_linear}, $\sigma_s(X) \subset \sigma_s(L) = L$.
 \end{proof}

 It is possible to see that non-degeneracy is also sufficient: i.e., secant varieties of a non-degenerate variety eventually feel the ambient space, see \ref{geometrySecants-lemma-secants_of_non_degenerate}. This is a corollary of the fact that, given a projective algebraic variety $X$, if $\sigma_s(X) = \sigma_{s+1}(X)$ then $\sigma_s(X)$ is a linear space, see \ref{geometrySecants-lemma-palatini_1}. This allows us to well-define the notion of $X$-rank for generic points.
 
 \begin{definition}
     \label{geometrySecants-definition-generic_rank}
     % Author: Alessandro Oneto
     Let $X \subset \bbP^N$ be a non-degenerate algebraic variety over an algebraically closed field. The \emph{generic $X$-rank} is the rank of a generic point of $\bbP^N$, i.e., the rank that occur on a Zariski-dense subset of the ambient space. Equivalently, the generic $X$-rank corresponds to the first secant variety of $X$ filling the ambient space, i.e., 
     \[
         \rank_X^\circ = \min\{s ~:~ \sigma_s(X) = \bbP^N\}.
     \]
     We denote by $\rank_X^{\max}$ the \emph{maximal $X$-rank} among all points $p \in \bbP^N$. 
 \end{definition}

 \begin{remark}
    \label{geometrySecants-remark-generic_rank_strategy}
    % Author: Alessandro Oneto
    A common geometric approach to compute generic ranks is through the study of dimensions of secant varieties. Indeed, if we know all dimensions of secant varieties of a variety $X$ we also know which is the first secant variety filling the ambient space, i.e., the generic $X$-rank. We dedicate \ref{geometrySecants-chapter-DimensionsSecants} to the fascinating story on the study of dimensions of secant varieties which has its roots in the XIX century and still offers many challenges and open questions.
 \end{remark}

 %%%%
 % UNDERLINE THAT OVER THE REALS THERE IS NOT GENERIC BUT TYPICAL RANK + ADD REFERENCE TO CHAPTER ABOUT REAL RANKS
 %%%%

 Now, the fact that, given a non-degenerate variety $X$ defined over an algebraically closed field, the $X$-rank is finite for any point of the ambient space can be deduced by the following easy topological argument coming from \cite{BT15}. 
 
 \begin{theorem}
    \label{geometrySecants-theorem-BT_maxrank_upperbound}
    % Author: Alessandro Oneto
    Let $X \subset \bbP^N$ be a non-degenerate variety. Then, $\rank_X^{\max} \leq 2\rank_X^\circ$. 
    %Moreover, in characteristic zero, if $\sigma_{\rank^\circ_X-1}(X)$ is an hypersurface, then $\rank_X^{\max} \leq 2\rank_X^\circ-1$.
 \end{theorem}
 \begin{proof}
    Let $U$ be the Zariski open subset of $\bbP^N$ such that if $p \in U$ then $\rk_X(p) = \rk^\circ_X$. For any $q \in \bbP^N$ consider the line $\langle p,q \rangle$ and, in particular, consider $p' \in U \cap \langle p,q \rangle$. Then, $\rank_X(p') = \rank_X^\circ$ and \[\rank_X(q) \leq \rank_X(p) + \rank_X(p') = 2\rank_X^\circ.\] 
 \end{proof}

 %%%%
 % ADD REFERENCE TO CHAPTER "MAXIMAL RANKS"
 %%%%

  Clearly, if a point $p \in \bbP^N$ has $X$-rank equal to $r$, then $p \in \sigma_r(X)$. However, the converse is not true in general. The following example is classical. Actually, for most of varieties of tensors defined in \ref{introduction-section-decomposable_tensors} the notion of rank is not lower semicontinuous.
 
 \begin{example}[$X$-rank is not always lower semicontinuous]
   \label{geometrySecants-example-Xrank_semicontinuous}
   % Author: Alessandro Oneto
   Let $X = \nu_3(\bbP^1) \subset \bbP^3$ be the rational normal cubic. If we identify the ambient space with the projective space of degree-$3$ binary homogeneous polynomials $\bbP(S^3\Bbbk^2)$, then $X$ is the variety of cubes of binary linear forms, see \ref{introduction-definition-symmetric_tensors}. Consider $xy^2 \in S^3\Bbbk^2$. Then, 
   \[
      xy^2 = \lim_{\epsilon \to 0} \frac{1}{3\epsilon}\left( x^3 - (x-\epsilon y)^3\right),
 \]
  in particular, $xy^2 \in \sigma_2(X)$ since it is the limit of points lying on secant lines. However, it is an easy exercise to check that $\rank_3(xy^2) = 3$. 
 \end{example}
 
 The existence of examples as the one of \ref{geometrySecants-example-Xrank_semicontinuous} motivate the following notion of rank.

 \begin{definition}[Border rank]
     \label{geometrySecants-definition-border_rank}
     % Author: Alessandro Oneto
     Let $X \subseteq \bbP^N$ be a projective variety and let $p \in \bbP^N$. The \emph{border $X$-rank} of $p$ is the smallest $s$ such that $p$ belongs to the $s$-th secant variety of $X$, that is 
     \[
         \brank_X(p) = \min\{s ~:~ p \in \sigma_s(X)\}.
     \]
 \end{definition}
 
 %% add historical remark on definition of border rank
 %% add reference to Bini example on matrix multiplication 
 
 \subsection{Ranks of tensors}
 \label{geometrySecants-subsection-ranks}
 % Author: Alessandro Oneto
 In \ref{geometrySecants-definition-Xrank} we defined the notion of $X$-rank with respect to any projective variety. When dealing with tensors, the idea is to adapt the variety $X$ to the tensor decomposition that one wants to consider. In particular, the following broadly known and used notions of ranks for tensors come from choosing $X$ as one of the varieties of decomposable tensors defined in \ref{introduction-section-decomposable_tensors}.
 
 \begin{definition}[Tensor rank]
 \label{geometrySecants-definition-tensor_rank}
 % Author: Alessandro Oneto
     The \emph{tensor rank} is the rank of a tensor $T \in V_1 \ootimes V_d$ with respect to the Segre variety $\Seg(V_1,\ldots,V_d)$ (see \ref{introduction-definition-Segre}); i.e., 
     \[
         \rank(T) = \min\left\{s ~:~ T = \sum_{i=1}^s v_{i,1}\ootimes v_{i,d}, ~ v_{i,j} \in V_j\right\}.
     \]
 \end{definition}
\begin{example}
 \label{geometrySecants-example-tensor_rank}
 % Author: Alessandro Oneto
In the case of matrices, that is when $d=2$, the notion of tensor rank coincides with the usual notion of rank of matrices. However, the case of matrices is geometrically very different from the case of higher-order tensors ($d\geq 3$): for example, unlike matrix rank, tensor rank is not lower-semicontinuous in general, as seen in \ref{geometrySecants-example-Xrank_semicontinuous}. 
\end{example}
 
 \begin{definition}[Symmetric tensor rank]
 \label{geometrySecants-definition-symmetric_tensor_rank}
 % Author: Alessandro Oneto
     The \emph{symmetric tensor rank} is the rank of a tensor $T \in S^d(V)$ with respect to the Veronese variety $\nu_d(\bbP V)$ (see \ref{introduction-definition-Veronese}); i.e., 
     \[
         \rank_d(T) = \min\left\{s ~:~ T = \sum_{i=1}^s v_{i}^{\otimes d}, ~ v_{i} \in V\right\}.
     \]
     By interpreting symmetric tensors as homogeneous polynomials, as explained in \ref{introduction-subsection-symmetric_tensors}, the symmetric tensor rank is often called \emph{Waring rank}: this is the smallest possible length of a decompositions of a homogeneous polynomial as sum of powers of linear forms.
 \end{definition}  
 
 \begin{definition}[Partially-symmetric tensor rank]
 \label{geometrySecants-definition-partially_symmetric_tensor_rank}
 % Author: Alessandro Oneto
     Let $\underline{d} = (d_1,\ldots,d_m)$. The \emph{partially-symmetric tensor rank} is the rank of $T \in S^{\underline{d}}(V_1,\ldots,V_m)$ with respect to the Segre-Veronese variety $\nu_{\underline{d}}(V_1,\ldots,V_m)$ (see \ref{introduction-definition-SegreVeronese}); i.e., 
     \[
         \rank_{\underline{d}}(T) = \min\left\{s ~:~ T = \sum_{i=1}^s v_{i,1}^{d_1}\ootimes v_{i,m}^{d_m}, ~ v_{i,j} \in V_j\right\}.
     \]
 \end{definition}
 
 \begin{definition}[Skew-symmetric tensor rank]
 \label{geometrySecants-definition-skewsymmetric_tensor_rank}
 % Author: Alessandro Oneto
     The \emph{skew-symmetric tensor rank} is the rank of $T \in \Lambda^dV$ with respect to the Grassmannian $\Gr_d(V)$ (see \ref{introduction-definition-Grassmannian}) in its Pl\"ucker embedding; i.e., 
     \[
         \rank_{\wedge}(T) = \min\left\{s ~:~ T = \sum_{i=1}^s v_{i,1}\wwedge v_{i,d}, ~ v_{i,j} \in V_j\right\}.
     \]
 \end{definition}


\chapter{Dimensions of Secant Varieties}
\label{geometrySecants-chapter-DimensionsSecants}
%%%%%%%%%%%%%%%%%%%%%%%%%%%%%%%%%%%%

A first question we might ask when studying algebraic varieties is its dimension. 

In the case of secant varieties, we can easily define a notion of expected dimension by a simple parameter count. We will see that this is not always the actual dimension: examples of varieties with secant varieties of dimension lower than the expected are classically known since the XIX century. We will explain how the problem of classifying such special varieties has been approached.

Recall that, knowing dimensions of secant varieties of a non-degenerate algebraic variety $X$ allows us to deduce the generic $X$-rank, see \ref{geometrySecants-section-rank} and in particular \ref{geometrySecants-definition-generic_rank}. This is one reason why the problem of classifying defective Veronese, Segre, Segre-Veronese varieties and Grassmannians received a particular attention. 

\section{Expected Dimension}
\label{geometrySecants-section-expectedDimension}
% Author: Alessandro Oneto

\begin{lemma}
\label{geometrySecants-lemma-expecteddimension}
% Author: Alessandro Oneto
Let $X_1\vvirg X_s \subseteq \bbP^N$ be algebraic varieties. Then 
\[
    \dim j(X_1\vvirg X_s) \leq \min\{ N, \smallsum_{i=1}^s \dim(X_i) + s-1\}.
\]
In particular, for any algebraic variety $X \subseteq \bbP^N$,  
\[
    \dim \sigma_s(X) \leq \min\{ N , s\dim(X) + s - 1\}.
\]
\end{lemma}
\begin{proof}
As they can be obtained by projection, the join $j(X_1,\ldots,X_s)$ and the secant variety $\sigma_s(X)$ have dimension smaller than their abstract analogous.

The dimension of the abstract join $\itJoin(X_1,\ldots,X_s)$, as defined in \ref{geometrySecants-section-secants}, easily follows from the fact that locally it can be interpreted as a $\bbP^{s-1}$-bundle over $X_1 \ttimes X_s$: therefore, 
    \[
        \dim \itJoin(X_1,\ldots,X_s) = \sum_{i=1}^s \dim(X_i) + s-1.
    \]
    In particular, $\dim \Sec_s(X) = s\dim(X) + s-1$.
\end{proof}

One may {\it expect} that the upper bound of \ref{geometrySecants-lemma-expecteddimension} is the actual dimension. This leads the the following definition. 
\begin{definition}
\label{geometrySecants-definition-expecteddimension}
% Author: Alessandro Oneto
    Let $X_1\vvirg X_s \subseteq \bbP^N$ be algebraic varieties. The {\it virtual dimension} of $j(X_1,\ldots,X_s)$ is the dimension of the abstract join. Its {\it expected dimension} is the minimum between the virtual dimension and the dimension of the ambient space. In other words,
    \begin{align*}
        \virdim j(X_1,\ldots,X_s) & = \sum_{i=1}^s \dim(X_i) + s-1; \\ 
        \expdim j(X_1,\ldots,X_s) & = \min \{N,\virdim j(X_1,\ldots,X_s)\}.
    \end{align*}
    In particular, if $X \subseteq \bbP^N$ is an algebraic variety, then
    \begin{align*}
        \virdim \sigma_s(X) & = s\dim(X) + s-1; \\ 
        \expdim \sigma_s(X) & = \min \{N,\virdim \sigma_r(X)\}.
    \end{align*}
By \ref{geometrySecants-lemma-expecteddimension} the expected dimension is always an upper bound for the dimension of a secant variety. We say that $X$ is \emph{$s$-defective} if $\dim \sigma_s(X) < \expdim\sigma_s(X)$. We call the difference $\delta_s(X) = \expdim\sigma_s(X) - \dim \sigma_s(X)$ the \emph{$s$-defect} of $X$. 

%% We will also give an algebraic meaning for the difference $\vdim\sigma_s(X) - \dim \sigma_s(X)$ when we will look at dimensions of secant varieties in terms of interpolation problems. 
\end{definition}

%%%%%%%%%%%%%%%%%%%%%%%%%%%%%%%%%%%%
\section{First lemmas and non-defectiveness of curves}
\label{geometrySecants-section-first_properties}
% Author: Alessandro Oneto
We collect here the first properties on secant varieties and their dimensions. The following lemmas were all classically know, see \cite{Pal09}. For a recent reference, we refer to \cite{Rus16}.

First of all, we observe that linear spaces are the \emph{idempotent} varieties in terms of secant varieties. 

\begin{lemma}
    \label{geometrySecants-lemma-palatini_1}
% Author: Alessandro Oneto
    Let $X\subset \bbP^N$ be an irreducible projective algebraic variety. If $\sigma_{r+1}(X) = \sigma_r(X)$, then $\sigma_r(X)$ is a projective linear space. 
\end{lemma}
\begin{proof}
    We prove this in three steps.

    {\bf Claim 1.} The statement is true for $r = 1$. I.e., if $\sigma_2(X) = X$ then $X$ is a projective linear space.
    \begin{proof}[Proof of Claim 1.]
        Let $P \in \langle X \rangle$. Then, $P = \lambda_1P_1 + \ldots + \lambda_rP_r$ for some $P_i \in X$ and $\lambda_i \in \Bbbk$. By assumption, $\lambda_1P_1 + \lambda_2P_2 \in X$ and then, $(\lambda_1P_1 + \lambda_2P_2) + \lambda_3P_3 \in X$. Inductively, we deduce that $P \in X$. Hence, by generality of $P$, $\langle X \rangle \subset X$ and equality follows. 
    \end{proof}

    {\bf Claim 2.} If $\sigma_{r+1}(X) = \sigma_r(X)$ then $\sigma_{r+h}(X) = \sigma_r(X)$ for all $h \geq 1$. 
    \begin{proof}[Proof of Claim 2.]
        Let $P \in \sigma_{r+2}(X)$, then $P = \lambda_1P_1 + \ldots + \lambda_{r+1}P_{r+1} + \lambda_{r+2}P_{r+2}$ where $P_i \in X$. By assumption, $\lambda_1P_1 + \ldots + \lambda_{r+1}P_{r+1} = \mu_1Q_1 + \ldots + \mu_rQ_r$, where $Q_i \in X$. In particular, $P \in \sigma_{r+1}(X)$. Therefore, $\sigma_{r+2}(X) \subset \sigma_{r+1}(X)$ and equality follows. 
    \end{proof}

    We are now ready to prove the general statement. By Claim 2, we have that $\sigma_{2r}(X) = \sigma_r(X)$. Since $\sigma_2(\sigma_r(X)) \subset \sigma_r(X)$, by irreducibility, we deduce that $\sigma_2(\sigma_r(X)) = \sigma_r(X)$. By Claim 1, $\sigma_r(X)$ is a projective linear space. 
\end{proof}


\begin{lemma}
    \label{geometrySecants-lemma-palatini_2}
% Author: Alessandro Oneto
Let $X\subset \bbP^N$ be a irreducible projective variety. If $\dim\sigma_{r+1}(X) = \dim\sigma_r(X)+1$, then $\sigma_{r+1}(X)$ is a projective linear space. 
\end{lemma}
\begin{proof}
    Let $p \in X$ be a general point. Hence,
    \[
        \sigma_s(X) \subsetneq j(p, \sigma_s(X)) \subset \sigma_{s+1}(X).  
    \]
    By assumption and irreducibility of $\sigma_{s+1}(X)$, $j(p, \sigma_s(X)) = \sigma_{s+1}(X)$. Let $q \in \sigma_{s+1}(X)$ be a generic smooth point: by the latter equality, $q \in \langle p,z \rangle$ for some $z \in \sigma_s(X)$. Being a cone, we also get that $p \in T_q\sigma_{s+1}(X)$. By generality of $p \in X$, we conclude that $\langle X \rangle \subset \sigma_{s+1}(X)$ and then, since the opposite inclusion is trivial, $\sigma_{s+1}(X) = \langle X \rangle$ and $\sigma_s(X)$ is a hypersurface. 
\end{proof}


An immediate but meaningful corollary is the fact that, if $X$ is \emph{non-degenerate}, i.e., it is not contained in any proper linear subspace, then the secant varieties of $X$ eventually fill the ambient space. 

\begin{lemma}
    \label{geometrySecants-lemma-secants_of_non_degenerate}
% Author: Alessandro Oneto
    Let $X\subset \bbP^N$ be a non-degenerate algebraic variety. Then, 
    \[
        X \subsetneq \sigma_2(X) \subsetneq \cdots \subsetneq \sigma_r(X) = \bbP^N.
    \]
\end{lemma}
As already mentioned, this observation allows us to conclude well-define the generic rank with respect to non-degenerate algebraic varieties, see \ref{geometrySecants-definition-generic_rank}.

Another important corollary is that \emph{curves are never defective.}
\begin{proposition}
    \label{geometrySecants-proposition-curves_nondefectiveness}
% Author: Alessandro Oneto
    Let $X\subset \bbP^N$ be a projective curve. Then, $\dim \sigma_s(X) = \min\{N, 2s-1\}$.
\end{proposition}
\begin{proof}
    By \ref{geometrySecants-lemma-expecteddimension},
    \[
        \dim \sigma_s(X) \leq s\dim(X) + s - 1 = 2s - 1.
    \]
    By \ref{geometrySecants-lemma-secants_of_non_degenerate} and the non-degeneracy assumption,
    \[
        \dim \sigma_s(X) \geq \dim \sigma_{s-1}(X) + 2
    \]
    unless $\sigma_{s-1}(X)$ is a hypersurface. Therefore, as long as $\sigma_{s-1}(X)$ is not an hypersurface, we proceed by induction (the base step $\sigma_1(X) = X$ being trivial) to obtain
    \[
        2s-1 \geq \dim \sigma_s(X) \geq \dim \sigma_{s-1}(X) + 2 = 2(s-1) - 1 + 2 = 2s - 1
    \]
    which implies $\dim \sigma_s(X) = 2s-1$. If $\sigma_{s-1}(X)$ is an hypersurface, then we clearly have that $\dim \sigma_s(X) = N = \min\{N,2s-1\}$ where the latter equality follows from being $2(s-1)-1 = 2s-3 = N - 1$. 
\end{proof}

%%%%%%%%%%%%%%%%%%%%%%%%%%%%%%%%%%%%
\section{Terracini's Lemma}
\label{geometrySecants-section-Terracini}
% Author: Alessandro Oneto

A general approach to study dimensions of secant varieties is through their tangent spaces at general points. For this reason, a fundamental tool in the study of dimension of secant varieties is \emph{Terracini's Lemma}, dating back to \cite{Ter11}.

The following statement is a modern approach which can be found in \cite[Corollary 1.10]{Adl87} or in the books \cite[Proposition II.1.10]{Zak93} and \cite[Theorem 1.4.1]{Rus16}.

\begin{lemma}[Terracini's Lemma]
\label{geometrySecants-lemma-terracini}
% Author: Alessandro Oneto
    Let $X, Y \subseteq \mathbb{P}^N$ be irreducible projective varieties over a field $\Bbbk$ of characteristic $0$. Let $x \in X, y \in Y$ be generic points and let $z \in \langle x,y \rangle$ be a generic point. In particular, we may assume that $z$ is a smooth point of $j(X,Y)$. Then
    \[
        T_{z} j(X,Y) = \langle T_x X , T_y Y\rangle.
    \]
\end{lemma}
\begin{proof}
We record the proof over the field of complex numbers, which is the setting originally studied by Terracini and which provides a good intuition on why the result works. See also \cite{BO08}. 

Consider the local parametrization of $j(X,Y)$ at $z$ given by the corresponding linear combination of the parametrization of $X$ and $Y$ at $x$ and $y$, respectively. The statement follows by observing that the matrix obtain by partial derivatives and whose rows span the tangent space of $j(X,Y)$ at $z$ is obtained by elementary operation on the concatenation of the corresponding matrices for $X$ and $Y$. 
\end{proof}

\begin{example}
\label{geometrySecants-example-terracini_positive_char}
% Author: Alessandro Oneto
    In positive characteristics, only the inclusion $T_{z} j(X,Y) \supset \langle T_x X , T_y Y\rangle$ is guaranteed, see e.g. \cite[Proposition II.1.12]{Zak93}. For example, over $\Bbbk$ with ${\rm char}(\Bbbk)= p > 0$ consider the curve $C \subset \bbP^3$ which is the projective closure of the curve parametrized by $t \mapsto (t,t^p,t^{p^2}) \subset \bbA^3$. Note that all tangent lines are parallel, i.e., there exists a point lying on all tangent lines to the curve $C$: such a curve is called {\it strange} in Section IV.3 of \cite{Har77}. It follows that the linear span on two tangent lines is $2$-dimensional since they have non-trivial intersection, while the secant variety of $C$, and then a general tangent space to it, is $3$-dimensional by \ref{geometrySecants-proposition-curves_nondefectiveness}. This contradicts the statement of Terracini's Lemma (\ref{geometrySecants-lemma-terracini}).
\end{example}

\begin{remark}
\label{geometrySecants-remark-terracini}
% Author: Alessandro Oneto
    Terracini's Lemma tells us that the dimension of $j(X,Y)$ is as in \ref{geometrySecants-definition-expecteddimension} if and only if two general tangent spaces $T_x(X)$ and $T_y(Y)$ do not intersect, unless $j(X,Y)$ fills the ambient space. 
\end{remark}

In the case $X = Y$, Terracini's Lemma can be used to study dimensions of secant varieties. As a first example, we can give a first explanation why secant varieties of quadratic Veronese varieties are always defective. In order to do that, we first need to understand what is the tangent space to Veronese varieties. 


\begin{example}[Cones are always defective]
\label{geometrySecants-lemma-cones_defective}
% Author: Alessandro Oneto
    Let $X \subset \bbP V$ be a projective variety. A {\it vertex} of $X$ is a point $p \in X$ such that $j(X,p) = X$, i.e., for any $q \in X$ the line $\langle p,q \rangle \in X$. Let ${\rm Vert}(X)$ the set of vertices of $X$: that is a linear subvariety of $X$. If $X$ is a projective cone, namely ${\rm Vert}(X) \neq \emptyset$, then for every $q \in X \smallsetminus {\rm Vert}(X)$, the tangent space at $q$ contains the vertex space ${\rm Vert}(X)$. In particular, general tangent spaces have non-trivial intersections and, Terracini's Lemma, projective cones are always defective. 
\end{example}

\begin{example}[Quadratic Veronese varieties are always defective]
\label{geometrySecants-lemma-quadratic_Veronese_defective}
% Author: Alessandro Oneto
    Let $V$ be a complex linear space. Let $\nu_d(\bbP V) \subset \bbP(S^d V)$ be the degree-$d$ Veronese embedding of $\bbP^n$ as in \ref{introduction-definition-Veronese}. Let $v \in V$. Then
    \[
        T_{[v]}\nu_d(\bbP V) = \{[v^{d-1}v'] ~:~ v' \in V\}. 
    \]
    Indeed, it is enough to observe that, for any $v' \in V$,
    \[
        \left.\frac{d}{dt}\right|_{t = 0}(v+tv')^{d} = dv^{d-1}v'.
    \]
    We consider now the quadratic case $d = 2$ and we employ Terracini's Lemma. Let $[v_1],\ldots,[v_s] \in \nu_2(\bbP V)$ be general points. We immediately note that 
    \[
        T_{[v_i]}\nu_2(\bbP V) \cap T_{[v_j]}\nu_2(\bbP V) = [v_iv_j]
    \]
    while, by dimension, we would expect to have empty intersection (if $\dim V \geq 3$). Therefore, we immediately deduce that quadratic Veronese varieties are defective, in particular:
    \[
        \dim \sigma_r(\nu_2(\bbP V)) \leq \min \left\{\dim S^dV - 1, r\dim \bbP V + r - 1 - {r \choose 2}\right\}.  
    \]
\end{example}

%%%%%%%%%%%%%%%%%%%%%%%%%%%%%%%%%%%%
\section{Classification of low-dimensional defective varieties}
\label{geometrySecants-section-classification}
% Author: Alessandro Oneto

In \ref{geometrySecants-proposition-curves_nondefectiveness} we have seen that curves are never defective. We record here other classification results for low-dimensional varieties.

Recall that in \ref{geometrySecants-lemma-cones_defective} we have seen that cones are always defective, as well as quadratic Veronese varieties, see \ref{geometrySecants-lemma-quadratic_Veronese_defective}. 

\subsection{Defective Surfaces}
\label{geometrySecants-section-defective_surfaces}
In \cite{Sev01}, Severi proved that the only surfaces in $\mathbb{P}^5$ which are not cones and for which the secant variety of lines does not fill the ambient space is the quadratic Veronese embedding of $\mathbb{P}^2$. A complete classification of defective surfaces is due to Terracini, see \cite{Ter21}, who filled the gaps of a previous tentative classification by Palatini, see \cite{Pal06}. 

\begin{theorem}
\label{geometrySecants-theorem-defective_surfaces}
%Author: Alessandro Oneto
    Let $X \subset \bbP^N$ be an irreducible surface. Then, $X$ is defective if and only if $X$ is a cone or a quadratic Veronese embedding of $\bbP^2$.
\end{theorem}

A modern approach can be found in \cite{CC02} where the notion of {\it weakly defective varieties} is introduced.

\subsection{Defective Threefolds}
In \cite{Sco08}, Scorza classified irreducible projective threefolds whose secant variety of lines is defective. 

\begin{theorem}
\label{geometrySecants-theorem-defective_threefolds}
%Author: Alessandro Oneto
    Let $X \subset \bbP^N$ be an irreducible non-degenerate threefold. Then, $X$ is $1$-defective if and only if:
    \begin{enumerate}
        \item $X$ is a cone;
        \item $X$ is contained in a $4$-dimensional cone over a curve;
        \item $N = 7$ and $X$ is contained in a $4$-dimensional cone over a quadratic Veronese surface;
        \item $X$ is the quadraic Veronese threefold of $\bbP^9$ or a projection of it in $\bbP^n$, $n = 7,8$;
        \item $N = 7$ and $X$ is a hyperplane section of the Segre embedding of $\bbP^2 \times \bbP^2$ in $\bbP^8$.
    \end{enumerate}
\end{theorem}


% \chapter{Defectivity results for secant varieties}
% \label{classicalAG-chapter-defectivity}
% 
% \chapter{Weak-defectivity results for secant varieties}
% \label{classicalAG-chapter-weakdefectivity}
% 
% \chapter{Apolarity theory}
% \label{classicalAG-chapter-apolarity}
% 
% \chapter{Alexander-Hirschowitz theorem}
% \label{classicalAG-chapter-AlexanderHirschowitz}
% 
% \chapter{AH type theorems for other varieties}
% \label{classicalAG-chapter-AHlike}
% 
% \chapter{Other notions of rank}
% \label{classicalAG-chapter-otherranks}
%
% \chapter{Strassen's additivity conjecture}
% \label{classicalAG-chapter-Strassenadditivity}
% 
% \chapter{Comon's conjecture}
% \label{classicalAG-chapter-Comon}




\part{Apolarity Theory}
\label{part-apolarity}
In this part, we study apolarity theory, in many of its forms. The part starts introducing the notion of Macaulay's inverse systems. Some basics of commutative algebra are assumed.

Following the historical development of the subject, we present most of the algorithms first in the setting of symmetric tensors, then illustrate generalizations to the more general setting. 

\chapter{Classical apolarity theory}
\label{tensorRank-chapter-apolarity}
% Author: Fulvio Gesmundo

Classical apolarity theory is one of the most used methods in the study of symmetric tensor rank. Its origins lie in seminal work by Cayley \cite{Cay45} and Sylvester \cite{Syl52,Syl53} and in the Macaulay's work on inverse systems. In this chapter, we briefly review the theory of inverse systems, we introduce the apolar ideal of a homogeneous polynomial and we state the classical apolarity lemma for Waring decompositions. Moreover, we discuss Sylvester's catalecticant method and we illustrate some of its consequences.

Given a vector space $V$, the symmetric algebra $\Sym V^*$ can be identified with the $\bbC[V]$ of polynomial functions on $V$, or equivalently the homogeneous coordinate ring of $\bbP V$; similarly, the symmetric algebra $\Sym V$ can be identified with the polynomial ring $\bbC[V^*]$. On the other hand, there is a natural action of $\Sym V^*$ on $\Sym V$ induced by the natural pairing of $V^*$ and $V$: more precisely, $\Sym V^*$ can be thought of as the ring of differential operators with constant coefficients on $\bbC[V]$. Apolarity theory is the study of the interplay of these ``two natures'' of the symmetric algebra $\Sym V^*$: as the coordinate ring on $\bbP V$ and as a ring of differential operators acting on $\Sym V^*$.

\section{Inverse systems}
\label{apolarity-section-inverseSystems}
% Author: Fulvio Gesmundo
The theory of Macaulay's inverse systems studies the algebraic properties of the actions of $\Sym V^*$ on $\Sym V$ by differentiation. We provide a brief overview before introducing classical apolarity, following \cite{Ger96}. Consider $\Sym V$ as the ring of polynomials in a fixed set of variables $x_0 \vvirg x_n$ and $\Sym V^*$ as the ring of polynomials in the dual variables $\partial_0 \vvirg \partial_n$. The ring $\Sym V$ is naturally a $\Sym V^*$-module with the action given by differentiation.

\begin{definition}
\label{tensorRank-definition-apolarityAction}
% Author: Fulvio Gesmundo
Let $V$ be a vector space and let $x_0 \vvirg x_n$ be a basis of $V$. Let $\xi_0 \vvirg \xi_n$ be the dual basis of $V^*$. The {\it apolarity action} of $\Sym V^*$ on $\Sym V$ is defined by 
\[
\xi_i \cdot f = \frac{\partial}{\partial x_i}f
\]
and extended to be an algebra action of $\Sym V^*$. 
\end{definition}
It is easy to prove that the apolarity action does not depend on the fixed choice of basis. In fact, it is equivariant with respect to the action of $\GL V$ on $V$ and $V^*$. The inverse system of an ideal of $\Sym V^*$ is the annihilator with respect to the apolarity action.
\begin{definition}[Inverse system]
 \label{apolarity-definition-inverseSystem}
% Author: Fulvio Gesmundo
 Let $I \subseteq \Sym V^*$ be an ideal. The {\it Macaulay inverse system} of $I$, denoted $I^{-1}$ is the $\Sym V^*$-submodule of $\Sym V$ defined by 
 \[
 I^{-1} = \{ g \in \Sym V : \Delta (g) = 0 \text{ for all $\Delta \in I$}\}.
 \]
\end{definition}
We point out that $ I^{-1}$ is a $\Sym V^*$-module, but not in general an ideal of $\Sym V$, that is it is not necessarily a $\Sym V$-module under multiplication. For instance, if $n = 1$, and $I = (\partial_0)$ then $I^{-1} = \bigoplus_{k \geq 0} \langle x_1^k \rangle$, which is not closed under multiplication, and in fact it is not even finitely generated as a $\Sym V^*$-module. We record some basic properties of the inverse system.
\begin{proposition}
 \label{apolarity-proposition-inverseSystemBasics}
% Author: Fulvio Gesmundo
Let $I,J \subseteq \Sym V^*$ be two ideals and let $I^{-1}, J^{-1}$ be their inverse systems. Then 
\begin{itemize}
 \item $I^{-1}$ is a graded $\Sym V^*$-module if and only if $I$ is homogeneous;
 \item if $I$ is homogeneous, then for every $k$, $I^{-1}_k = I_k^\perp$, where $I_k$ denotes the homogeneous component of degree $k$ of $I$;
 \item $I^{-1}$ is finitely generated as an $\Sym V^*$-module if and only if $I$ is Artinian;
 \item $(I \cap J)^{-1} = I^{-1} + J^{-1}$.
\end{itemize}
\end{proposition}

The inverse system gives a correspondence between $\Sym V^*$-submodules of $\Sym V$ and $0$-dimensional ideals of $\Sym V^*$. More precisely, we have the following result, for which we refere to \cite[Theorem 21.6]{Eis95}.
\begin{theorem}[Macaulay Correspondence]
 \label{apolarity-theorem-MacaulayCorrespondence}
 % Author: Fulvio Gesmundo
There is an inclusion reversing correspondence between finitely generated $\Sym V^*$-submodules of $\Sym V$ and Artinian ideals of $\Sym V^*$ containing $S^1 V^*$. An ideal $I \subseteq \Sym V^*$ corresponds to its inverse system $I^{-1}$. A finitely generated module $M \subseteq \Sym V$ corresponds to its socle $\soc(M) = (0: M) = \{ \Delta \in \Sym V^*: \Delta \cdot M = 0\}$.
\end{theorem}

\section{The classical apolarity lemma}
\label{tensorRank-section-apolarityLemma}
% Author: Fulvio Gesmundo

Of particular interest are ideals arising as annihilator of single elements $f \in \Sym V$, that is with the property that the inverse system is generated by a single element. 
\begin{definition}[Apolar ideal]
\label{tensorRank-definition-apolarIdeal}
% Author: Fulvio Gesmundo
Let $V$ be a vector space and let $f \in \Sym V$. The {\it apolar ideal of} $f$ is
\[
\Ann(f) = \{ D \in \Sym V^* : D (f) = 0 \}.
\]
\end{definition}
We say that an Artinian algebra is Gorenstein if it is of the form $\Sym V^* / \Ann(f)$ for some element $f \in \Sym V$.  We record the following immediate properties:
\begin{lemma}
\label{tensorRank-definition-apolarIdealBasics}
% Author: Fulvio Gesmundo
Let $V$ be a vector space and let $f \in Sym V$. Then
\begin{itemize}
\item the inverse system $(\Ann(f))^{-1}$ is the $\Sym V^*$-submodule of $\Sym V$ generated by $f$, via differentiation;
 \item if $\deg f = d$, then $S^e V^* \subseteq \Ann(f)$ for $e > d$;
\item $f$ is homogeneous if and only if $\Ann(f)$ is a homogeneous ideal;
\item if $f$ is homogeneous, for every $e \leq d$, we have $\Ann(f)_e = (\Flat_e(f) : S^e V^* \to S^{d-e} V)$, where $\Flat_e(f)$ is the $e$-th catalecticant map of $f$, in the sense of \ref{RepTheory-chapter-flattenings}.
\end{itemize}
\end{lemma}
It is natural to ask to what extent it is possible to recover the polynomial $f$ from its apolar ideal $\Ann(f)$. This is completely understood and we record here the result following \cite[Lemma 3.33A]{IE78} and \cite[Prop. 2.14]{Jel17}.

\begin{proposition}
\label{apolarity-proposition-sameApolar}
% Author: Fulvio Gesmundo
 Let $f,g \in \Sym V$ be two polynomials. The following are equivalent:
 \begin{itemize}
  \item $\Ann(f) = \Ann(g)$;
  \item there exists $\Theta \in \Sym(V^*)$ with nonzero constant term such that $\Theta \cdot f = g$. 
 \end{itemize}
\end{proposition}

The classical apolarity lemma relates the apolar ideal of a homogeneous polynomial to the existence of a Waring decomposition. Recall the definition of Waring decomposition and Waring rank from \ref{geometrySecants-definition-symmetric_tensor_rank}: given a homogeneous polynomial $f \in S^d V$ of degree $d$, a Waring decomposition of $f$ is an expression of $f$ of the form 
\[
f = \ell_1^d + \cdots + \ell_r^d
\]
for some $\ell_i \in V$, and the Waring rank $\rank_d(f)$ of $f$ is the smallest possible number of summands for which such a decomposition exists. Projectively, a Waring decomposition is a set of $r$ points $\bbX = \{ \ell_1 \vvirg \ell_r \} \subseteq \bbP V$ with the property that $f \in \langle \nu_d(\bbX) \rangle$, where $\nu_d$ is the Veronese embedding of \ref{introduction-definition-Veronese}. 

\begin{lemma}[Apolarity Lemma]
 \label{tensorRank-lemma-apolarityLemma}
 % Author: Fulvio Gesmundo
 Let $f \in \bbP S^d V$ and let $\bbX = \{ \ell_1 \vvirg \ell_r\}$ be a set of $r$ points in $\bbP V$. The following are equivalent:
 \begin{enumerate}[(i)]
  \item $f \in \langle \nu_d(\bbX) \rangle$;
  \item $I(\bbX) \subseteq \Ann(f)$.
 \end{enumerate}
 In particular $\rank_d(f)$ is the smallest $r$ such that $\Ann(f)$ contains the ideal of $r$ distinct points.
\end{lemma}



\section{The catalecticant method}
\label{tensorRank-section-catalecticant}
% Author: Fulvio Gesmundo
In this section, we discuss Sylvester's catalecticant method \cite{Syl52,IK99}, an algorithm to compute the ideal of a Waring decomposition which is the foundational method of many other more advanced algorithms, for instance the ones discussed in \cite{BCMT10,BBCM13,BT20,LMR23}.




\part{Representation Theory}
\label{part-RepTheory}
We assume some background on representation theory, to the level of \cite[Ch.6]{Lan12}; for a detailed introduction, we refer to \cite{FH91}. We introduce some notation and important facts.

Let $\GL(V)$ denote the general linear group of a vector space $V$ and $\SL(V)$ denote the special linear group. Let $\frakS_d$ denote the symmetric group on $d$ elements. 

A representation of a group $G$ is a group homomorphism $\rho : G \to \GL(V)$ for some vector space $V$. We use the word \emph{representation} to refer to the space $V$ itself. 

A partition $\lambda = (\lambda_1 \vvirg \lambda_n)$ is a non-increasing sequence of positive integers. We say that $\lambda$ is a partition of $d$ if $\sum_i \lambda_i = d$, and we write $|\lambda| = d$; we say that $n$ is the length of $\lambda$ and we write $n = \ell(\lambda)$. Write $\lambda \partinto_n d$ to mean that $\lambda$ is a partition of $d$ of length at most $n$.

The (polynomial) irreducible representations of the general linear group of a vector space of dimension $n$ are indexed by partitions of length at most $n$. Let $\bbS_\lambda V$ be the irreducible representation associated to the partition $n$; this is the Schur module of $\GL(V)$ associated to $\lambda$.

The irreducible representations of the symmetric group $\frakS_d$ are indexed by partitions of $d$. Let $[\lambda]$ be the irreducible representation associated to the partition $\lambda$; this is the Specht module of $\frakS_d$ of type $\lambda$. 

The vector space $V^{\otimes d}$ is acted on naturally by $\frakS_d$, which permutes the tensor factors, and $\GL(V)$ which acts diagonally on all tensor factors; these two actions commute and the Schur-Weyl decomposition theorem expresses the spaces as a direct sum of irreducible representations for $\frakS_d \times \GL(V)$. Such decomposition is as follows:
\[
V^{\otimes d} = \bigoplus_{\lambda \partinto_n d} [\lambda] \otimes \bbS_{\lambda} V.
\]
A fundamental result which will prove useful numerous times is Schur's Lemma. Given $V,W$ representations for an algebraic group $G$, and a linear map $\phi : V \to W$, we say that $\phi$ is $G$-equivariant if it commutes with the action of $G$, that is $\phi( g \cdot v) = g \cdot \phi(v)$ for every $v \in V$, $g \in G$. The subset of $\Hom(V,W)$ consisting of $G$-equivariant maps is a linear subspace, and it is denoted by $\Hom_G(V,W)$. 
\begin{lemma}[Schur's Lemma]
\label{repTheory-lemma-Schur}
Let $G$ be a group, and let $\phi : V \to W$ be an equivariant map between two $G$-representations. Then $\ker(\phi)$ and $\image(\phi)$ are $G$-representations. In particular, if $V$ is irreducible, then $\phi = 0$ or $\phi$ is injective. If $V= W$ then $\phi = \lambda \Id_V$ for some $\lambda \in \bbC$, that is $\dim \Hom _G(V,W) = 1$.
\end{lemma}

By \ref{repTheory-lemma-Schur}, the Schur-Weyl decomposition shows $\dim \Hom_{\GL(V)} ( \bbS_\lambda V, V^{\otimes d}) = \dim [\lambda]$. More generally, a fundamental problem in representation theory and algebraic combinatorics consists in determining the decomposition of the tensor product of two (or more) irreducible representations. For representations of the general linear group we have the following:
\begin{definition}[Littlewood-Richardson coefficients]
 \label{RepTheory-definition-LRcoefficient}
 Let $\lambda,\mu,\nu$ be three partitions. The {\it Littlewood-Richardson coefficient} associated to $(\lambda,\mu;\nu)$ is
 \[
c^\nu_{\lambda,\mu} = \dim \Hom_{\GL(V)} (\bbS_\nu V,  \bbS_\lambda V \otimes \bbS_\mu V ).
\]
\end{definition}
Via \ref{repTheory-lemma-Schur}, the Littlewood-Richardson coefficient $c^\nu_{\lambda,\mu} $ coincides with the multiplicity of $\bbS_\nu V$ as a subrepresentation of $ \bbS_\lambda V \otimes \bbS_\mu V $, that is 
\[
 \bbS_\lambda V \otimes \bbS_\mu V  = \bigoplus_{\nu } (\bbS_{\nu} V )^{\oplus c^\nu_{\lambda,\mu} }.
\]
In particular, if $c^\nu_{\lambda,\mu} \neq 0$, then $|\nu| = |\lambda|+|\mu|$.

Littlewood-Richardson coefficients are hard to compute \cite{Nar06} even though there are polynomial-time algorithms to decide whether they are zero or nonzero as observed in \cite{MNS12} following \cite{KT99}. However, there are several special cases for which they are easy to determine. For instance, when $\mu$ is a partition consisting of a single row of a single column, we have the following classical result.
\begin{proposition}[Pieri's rule]
 Let $\lambda$ be a partition and let $d \geq 0$. Then 
 \[
 \bbS_{\lambda} V \otimes S^d V = \bigoplus_{\nu \in R} \bbS_{\nu} V, \qquad  \bbS_{\lambda} V \otimes \Lambda^d V = \bigoplus_{\nu \in C} \bbS_{\nu} V.
 \]
Here $R$ (resp. $C$) is the set of partitions of $|\lambda|+d$ obtained from $\lambda$ by adding $d$ boxes, no two of them on the same column (resp. row).
\end{proposition}

\chapter{Equations via representation theory}
\label{RepTheory-chapter-equations}
\section{$G$-varieties and equations}
 \label{repTheory-section-Gvarieties}
% Author: Fulvio Gesmundo
In a number of settings, algebraic varieties of interest are invariant under the action of an algebraic group. In these cases, representation theory, and in particular \ref{introduction-lemma-Schur} can be of great help in the study of equations for the variety of interest.

\begin{definition}
 \label{repTheory-definition-Gvariety}
% Author: Fulvio Gesmundo
 Let $G$ be a group and let $V$ be a $G$-representation. We say that a variety $X \subseteq \bbP V$ is a $G$-variety if $X$ is invariant under the action of $G$, that is $G \cdot X = X$. 
\end{definition}
We point out that the points of a $G$-variety $X$ are not in general fixed by the action of $G$. If this is the case, we say that $X$ is point-wise invariant by the action of $G$.

The Segre-Veronese varieties $\nu_{d_1 \vvirg d_k} ( \bbP V_1 \ttimes \bbP V_k)$ of \ref{introduction-definition-SegreVeronese}, their secant varieties (see \ref{geometrySecants-chapter-BasicDefinitions}), tangential varieties, joins of such, and more generally varieties constructed \emph{functorially} from them are $G$-varieties, with $G = \GL(V_1) \ttimes \GL(V_d)$. Orbit-closures for the action of a group $G$, in the sense of \ref{introduction-definition-orbitsdegenerations}, are $G$-variaties as well.

The action of a group $G$ on a space $V$ induces an action of the polynomial ring $\bbC[V]$ via pull-back: given $F \in \bbC[V]$, $g \in G$, one has 
\[
g \cdot F = F \circ g^{-1}.
\]
If the action is linear, it restricts to the homogeneous components of $\bbC[V] = \bigoplus_{d \geq 0} S^d V^*$. If $X$ is a $G$-variety, then the homogeneous components of its ideal are $G$-representations as well.
\begin{lemma}
\label{repTheory-lemma-GactionOnIdeal}
% Author: Fulvio Gesmundo
 Let $X \subseteq \bbP V$ be a $G$-variety. Let $I(X) \subseteq \bbC[V]$ be the ideal of $X$. Then for every $g \in G$ and every $F \in \bbC[V]$, we have that $F \in I(X)$ if and only if $g \cdot F \in I(X)$. In particular, the homogeneous components $I(X)_d \subseteq  S^d V^*$ are subrepresentations of $G$.
\end{lemma}
Because of this result, every equation $F$ of a $G$-variety yields a \emph{module} of equations, that is the smallest $G$-subrepresentation containing $F$. The term equation is often used loosely, in the general sense of \emph{Zariski closed condition}.

In fact, it is often possible to realize the homogeneous components of the ideals of a $G$-variety as kernels (or images) of $G$-equivariant maps defined naturally.  A typical example is the one of equations arising from flattenings, discussed in \ref{RepTheory-chapter-flattenings}, which are particularly effective to find equations of secant varieties. We discuss here two other examples:
\begin{itemize}
 \item Kostant's Theorem, which determines equations for rational homogeneous varieties in purely representation theoretic terms;
 \item the Foulkes map, which appears in algebraic combinatorics and whose kernel describes the ideal of the Chow variety of completely reducible forms.
\end{itemize}

\section{Kostant's Theorem}
\label{RepTheory-section-kostant}
% Author: Fulvio Gesmundo
We say that a $G$-variety $X$ is \emph{homogeneous} for the action of $G$ if the group $G$ acts transitively on $X$. Important examples of homogeneous varieties include the varieties of rank one elements introduced in \ref{introduction-section-decomposable_tensors}. In fact, those varieties are examples of rational homogeneous varieties.
\begin{definition}
 \label{reptheory-definition-rationalhomogeneousvariety}
 Let $X \subseteq \bbP V$ be a $G$-variety, with $V$ irreducible representation of a semisimple algebraic group $G$. We say that $X$ is a {\it rational homogeneous variety} if $X$ is the $G$-orbit of an element $p \in \bbP V$.
\end{definition}
A rational homogeneous variety $X \subseteq \bbP V$ is uniquely characterized by the highest weight $\lambda$ of the $G$-representation $V$. Indeed, if $v \in V$ is a highest weight vector, then it is easy to see $X = G \cdot [v]$. 

\begin{example}
 \label{reptheory-example-RHV}
% Author: Fulvio Gesmundo
 The Segre-Veronese variety $X = \nu_{d_1 \vvirg d_k} (\bbP V_1 \ttimes \bbP V_k) \subseteq \bbP (S^{d_1} V_1 \ootimes S^{d_k}V_k)$ is a rational homogeneous variety under the action of $G = \GL(V_1) \ttimes \GL(V_k)$. Indeed, up to the choice of a torus in $\GL(V_i)$ any vector $v_i \in V_i$ can be regarded as a highest weight vector: then $X = G \cdot (v_1^{d_1} \ootimes v_k^{d_k})$. 
\end{example}

By \ref{repTheory-lemma-GactionOnIdeal}, the ideal of a rational homogeneous variety is a $G$-representation $I(X) \subseteq \Sym( V^*)$. Kostant's Theorem characterizes such ideal:
\begin{theorem}
 \label{reptheory-theorem-kostant}
% Author: Fulvio Gesmundo
Let $G$ be a semisimple algebraic group and let $V_\lambda$ be the irreducible $G$-representation of highest weight $\lambda$. Let $X \subseteq \bbP V_\lambda$ be the rational homogeneous variety in $\bbP V_\lambda$. Then, for every $d \geq 1$, 
\[
I(X)_d = V_{d\lambda}^\perp \subseteq S^d V_\lambda^*.
\]
Moreover, $I(X)_2$ generates the ideal $I(X)$.
\end{theorem}
We refer to \cite[Ch.16]{Lan12}

\section{The Foulkes map and the ideal of the Chow variety}
\label{RepTheory-section-foulkes}
% Author: Fulvio Gesmundo

Let $V$ be a vector space. The Chow variety of completely reducible forms of degree $d$ is 
\[
\Ch^{d}(V) = \{ \ell_1 \cdots \ell_d : \ell_i \in V \} \subseteq \bbP S^d V.
\]
It is not hard to see that $\Ch^d(V)$ is an Zariski-closed. It is evidently closed under the action of $\GL(V)$, hence it is a $\GL(V)$-variety. If $d \leq \dim V -1$, then 
\[
\Ch^d(V) = \overline{\GL(V) \cdot (x_1 \cdots x_d)}
\]
where $x_1 \vvirg x_d$ are $d$ vectors of $V$ in general linear position; in particular if $d\leq \dim V$ they are linearly independent. The Chow variety is the image of the Segre variety $\Seg(\bbP V \ttimes \bbP V) \subseteq \bbP V^{\otimes d}$ under the projection onto the fully symmetric component $S^d V$.

Since $\Ch^d(V)$ is a $\GL(V)$-variety, by \ref{repTheory-lemma-GactionOnIdeal}, the homogeneous components $I(\Ch^d(V))_e \subseteq S^e S^d V^*$ of the ideal of the Chow variety are subrepresentations of $ S^e S^d V^*$. These representations are hard to understand and they are related to long standing problems in representation theory and combinatorics. In particular, the components $I(\Ch^d(V))_e$ can be described in terms of the kernel of a natural map:
\begin{definition}
 \label{RepTheory-definition-foulkesmap}
% Author: Fulvio Gesmundo
 Let $d,e$ be positive integers. Index the tensor factors of $V^{\otimes de}$ with pairs $(i,j) \in [d]\times [e]$. Let $\pi_{d,e}$ be the composition
 \[
\bigotimes_{(i,j)} V_{(i,j)} \to (S^e V_{(1,\ast)}) \ootimes (S^eV_{(d,\ast)}) \to S^d S^e V  
 \]
 of the $\frakS_e$-symmetrization on every set of $e$ spaces with common first index, followed by the symmetrization on the $d$ resulting compies of $S^e V$. Let $\pi_{e,d}$ be the analogous composition obtained by reversing the two symmetrizations and write $S^eS^d V$ for its image. The Foulkes-Howe map is the composition
 \[
 \rmFH_{d,e} : S^e S^d V \to V^{\otimes de} \xto{\pi_{d,e}} S^d S^e V.
 \]
 of the embedding of $S^e S^d V = \image(\pi_{e,d})$ into $V^{\otimes de}$ followed by the projection $\pi_{d,e}$.
 \end{definition}
This map was introduced by Hermite in the case $\dim V = 2$ \cite{Her56}, and it was observed that in that case it is an isomorphism; this result is called Hermite reciprocity today. Hadamard studied it in general and proved that it completely controls the ideal of the Chow varieties \cite{Had97}.
\begin{theorem}
 \label{RepTheory-theorem-foulkeskerChow}
% Author: Fulvio Gesmundo
Let $V$ be a vector space. Let $d,e$ be nonnegative integers. Then 
\[
I(\Ch^d (V)) _ e = \ker \rmFH_{d,e}.
\]
\end{theorem}
\begin{proof}
 We refer to \cite[Prop. 8.6.1.2]{Lan12}.
\end{proof}


In particular, one recovers Hermite's result:
\begin{corollary}[Hermite reciprocity]
\label{RepTheory-corollary-hermiteReciprocity}
% Author: Fulvio Gesmundo
If $\dim V = 2$, then the Foulkes-Howe map $\rmFH_{d,e}$ is an isomorphism.
\end{corollary}
\begin{proof}
By the Fundamental Theorem of Algebra, every binary form splits as product of linear forms. Therefore $\Ch^d (V) = \bbP S^d V$ if $\dim V = 2$, and $I(\Ch^d(V))_e = 0$ for every $e$. By \ref{RepTheory-theorem-foulkeskerChow}, we deduce that $\rmFH_{d,e}$ is injective for every $d,e$. Since domain and codomain have the same dimension, we conclude that it is an isomorphism.
\end{proof}

Hermite reciprocity for arbitrary fields is explained in \cite[Sec.~3.4]{AFPRW19} and \cite{RS21,MW22}.


Another immediate consequence of \ref{RepTheory-theorem-foulkeskerChow} is the following:
\begin{corollary}
\label{RepTheory-corollary-foulkesplethysmbound}
% Author: Fulvio Gesmundo
Let $V$ be a vector space with $\dim V =n$. Let $d,e$ be nonnegative integers and let $\lambda$ be a partition of $ed$ with $\ell(\lambda) \leq n$. If $a_\lambda(e,d) > a_\lambda(d,e)$ then $I(\Ch_d(V))_e \neq 0$.
\end{corollary}
A special case of \ref{RepTheory-corollary-foulkesplethysmbound} is the one where $\ell(\lambda) > d$. In this case, a consequence of Pieri's rule (see \ref{introduction-proposition-Pieri}) is that $a_\lambda(e,d) = 0$. Therefore the entire isotypic component of type $\lambda$ in $S^e S^d V^*$ is contained in $I(\Ch_d(V))_e$. In fact, this is also a consequence of the fact that, by construction, $\Ch_d(V)$, regarded as a subvariety of $V^{\otimes d}$, is contained in the subspace variety of tensors whose multilinear ranks are $d$, hence all modules in $S^eS^d V^*$ of type $\lambda$ with $\ell(\lambda) \geq d$ vanish on $\Ch_d(V)$.

\subsection{Inequalities between plethysm coefficients}
\label{RepTheory-subsection-plethysmInequalities}
% Author: Fulvio Gesmundo
Results on inequalities between different structure coefficients in representation theory are of interest in algebraic combinatorics. For plethysm coefficients, several results are known in the asymptotic setting. A fundamental conjecture is due to Foulkes:
\begin{conjecture}
 \label{RepTheory-conjecture-Foulkes}
% Author: Fulvio Gesmundo
 Let $e \geq d$ be nonnegative integers and let $\lambda$ be a partition of $ed$. Then $a_\lambda (e,d) \geq a_\lambda(d,e)$.
\end{conjecture}
A stronger conjecture was posed by Hadamard, predicting that $\rmFH_{d,e}$ is injective for $e \leq d$ and it is known by the Foulkes-Howe conejcture. Proving this stronger conjecture of a specific pair $(e,d)$ clearly proves \ref{RepTheory-conjecture-Foulkes} for the same pair. However, in \cite{MN05}, it was shown that Hadamard's conjecture is false: $\rmFH_{5,5}$ has nontrivial kernel. The situation is far from understood in general and a better understanding would shed light on structural properties of the ideal of Chow varieties.




\chapter{Flattening methods}
\label{RepTheory-chapter-flattenings}
A typical Zariski closed condition is the boundedness of the rank of a matrix. We refer to \emph{flattening method} for any method which yields modules of equations via rank condition on a matrix. More precisely, a \emph{generalized flattening} of a space $V$ is any (polynomial) map 
\[
\Flat : V \to \Hom ( E,F)
\]
for two vector spaces $E,F$. More generally, one can consider a section of a bundle $\calHom(\calE,\calF)$ where $\calE,\calF$ are two vector bundles over $\bbP V$, see e.g. \cite{EH88}.

We say that a module of equations for a variety $X$ arises from a flattening if, for every $x \in X$, $\Flat(x)$ is a matrix of rank at most $r_X$. The equations can be described explicitly as the minors of size $r_X+1$ of $\Flat(v)$, which are polynomials in the element $v \in V$.

Tautologically, the variety $\sigma_r( \bbP E \times \bbP F)$ of matrices of rank at most $r$ has (ideal-theoretic) equations arising from a flattening. Similarly, the standard flattenings introduced in \ref{introduction-section-representingtensors} are examples of flattening maps; it is a classical fact that they yield equations for Segre-Veronese varieties:
\begin{proposition}
% Author: Fulvio Gesmundo
\label{RepTheory-proposition-standardFlatrank1}
 Let $T \in \bbP (V_1 \ootimes V_k)$ be a tensor. Then $T \in \bbP V_1 \ttimes \bbP V_k$ belongs to the Segre variety if and only if all flattenings $T_i : V_i^* \to \bigotimes _{j \neq i} V_j$ have rank $1$.
\end{proposition}
A similar statement holds for Segre-Veronese varieties and it is discussed in \ref{RepTheory-section-flatteningsSecants} together with an extension to its secant varieties.

\section{Flattenings for secant varieties}
\label{RepTheory-section-flatteningsSecants}
% Author: Fulvio Gesmundo

When the flattening map $\Flat : V \to \Hom(E,F)$ is linear, then rank conditions for a variety $X$ yield rank conditions for the secant varieties of $X$. We state this formally following \cite[Prop. 4.1.1]{LO13}.
\begin{lemma}
 \label{RepTheory-lemma-LandsbergOttaviani411}
% Author: Fulvio Gesmundo
 Let $X \subseteq \bbP V$ be a variety and let $\Flat : V \to \Hom(E,F)$ be a linear map. Let 
 \[
 r_X = \max \{ r : \rank ( \Flat(x)) = r \text{ for some $x \in X$}\}.
 \]
If $p \in \sigma_r(X)$, then $\rank(\Flat(p)) \leq r \cdot r_X$. In particular, if $\rank(\Flat(p)) = s$, then $p \notin \sigma_r (X)$ for $r = \lceil s / r_X \rceil$.
\end{lemma}

\ref{RepTheory-proposition-standardFlatrank1} is a particular case of \ref{RepTheory-lemma-LandsbergOttaviani411}. We discuss it here in the case of Segre-Veronese varieties. In this setting, the standard flattenings defined in \ref{introduction-definition-flattenings} can be written as follows. Fix spaces $V_1 \vvirg V_k$ and integers $d_1 \vvirg d_k, e_1 \vvirg e_k$ with $e_i \leq d_i$; then there is a natural flattening map
\[
\begin{aligned}
\Flat_{(d_1 \vvirg d_k)}^{(e_1 \vvirg e_k)} : S^{d_1} V_1 \ootimes S^{d_k}V_k &\to \Hom ( S^{e_1} V_1^* \ootimes S^{e_k}V_k^* , S^{d_1 - e_1} V_1 \ootimes S^{d_k- e_k}V_k ) \\ 
f_1 \ootimes f_k &\mapsto \left[ \eta_1 \ootimes \eta_k \mapsto  \eta_1(f_1) \ootimes \eta_k(f_k) \right] .
\end{aligned}
\]
In the symmetric setting, the map $\Flat_e : S^e V^* \to S^{d-e} V$ is called \emph{catalecticant map} of $f$.

Then the following holds:
\begin{lemma}
\label{RepTheory-lemma-catalecticantSegreVeronese}
% Author: Fulvio Gesmundo
 Let $T \in  S^{d_1} V_1 \ootimes S^{d_k}V_k$. The following are equivalent:
 \begin{enumerate}[(i)]
  \item $T$ is an element of the Segre-Veronese variety $\nu_{d_1 \vvirg d_k} ( \bbP V_1 \ttimes \bbP V_k)$;
  \item for every $(e_1 \vvirg e_k)$, $\Flat_{(d_1 \vvirg d_k)}^{(e_1 \vvirg e_k)}(T)$ has rank one, as a linear map. 
 \end{enumerate}
In particular, the $2 \times 2$ minors of the standard flattenings give set-theoretic equations for $\nu_{d_1 \vvirg d_k} ( \bbP V_1 \ttimes \bbP V_k)$. Moreover, if $T \in \sigma_r(\nu_{d_1 \vvirg d_k} ( \bbP V_1 \ttimes \bbP V_k))$ then $\rk(\Flat_{(d_1 \vvirg d_k)}^{(e_1 \vvirg e_k)}(T)) \leq r$.
\end{lemma}
The converse of \ref{RepTheory-lemma-catalecticantSegreVeronese} in the case of higher secant varieties is false, except in particular cases. An important case is the one of the second secant variety: in this case the converse was proved in \cite[Theorem 5.1]{LM04} in the case of Segre variety, whereas in the setting of Veronese varieties it is classical; the results of \cite{BL14} and in particular the classification results for elements of the second secant variety allow one to extend it to every Segre-Veronese variety.
\begin{theorem}
 \label{RepTheory-theorem-LandsbergManivelSigma2}
% Author: Fulvio Gesmundo
 Let $T \in  S^{d_1} V_1 \ootimes S^{d_k}V_k$. The following are equivalent:
 \begin{enumerate}[(i)]
  \item $T \in \sigma_2(\nu_{d_1 \vvirg d_k} ( \bbP V_1 \ttimes \bbP V_k))$;
  \item for every $(e_1 \vvirg e_k)$, $\rank( \Flat_{(d_1 \vvirg d_k)}^{(e_1 \vvirg e_k)}(T) ) \leq 2$. 
 \end{enumerate} 
 In particular, $3 \times 3$ minors of the standard flattenings give set-theoretic equations for $ \sigma_2(\nu_{d_1 \vvirg d_k} ( \bbP V_1 \ttimes \bbP V_k))$.
\end{theorem}
Another particular case is the one of the rational normal curve, that is the case where $k = 1$ and $\dim V = 2$. The result is classical, see e.g. \cite[Chapter 7]{IK99}.
\begin{theorem}
 \label{RepTheory-theorem-rationalnormalcurves}
% Author: Fulvio Gesmundo
 Let $f \in  S^{d} V$ with $\dim V = 2$. The following are equivalent:
 \begin{enumerate}[(i)]
  \item $f \in \sigma_r(\nu_{d} \bbP V)$;
  \item for every $e$ with $0 \leq e \leq d$, $\rank( \Flat_d^e(f))  \leq r$. 
 \end{enumerate} 
 In particular, $(r+1) \times (r+1)$ minors of the standard flattenings give set-theoretic equations for $\sigma_r(\nu_d(\bbP^1))$.
\end{theorem}
More generally, only partial results are known. However, in \cite{LO13}, equations for secant varieties of Segre varieties are constructed via \ref{RepTheory-lemma-LandsbergOttaviani411} in a representation theoretic setting, with the introduction of more interesting flattening maps. We present the construction here in slightly broader generality. Let $G$ be a group and let $V$ be a $G$-representation. Suppose $E,F$ are $G$-representations with the property that $E^* \otimes F$ contains $V$ as a subrepresentation. The Young flattening of $V$, associated to the pair $(E,F)$ is the linear map $\YFlat^V_{E,F} : V \to \Hom( E,F)$ mapping the element $v \in V$ to the composition
\begin{align*}
E \xto{\otimes v} E \otimes V \to E \otimes E^* \otimes F \xto{\lrcorner \Id_E} F
\end{align*}
where the first map is tensorising by $v$, the second is the inclusion $V \subseteq E^* \otimes F$ and the third map is contraction by the identigy element $\Id_E \in E \otimes E^*$.

We record explicitly the construction of Koszul-Young flattenings for tensors of order three. In several settings, classical equations for secant varieties can be reinterpreted in terms of this construction: this is the case of the classical Aronhold invariant of ternaty cubics \cite[Prop. 4.4.7]{Stu93} and for Strassen's equations for higher secant varieties \cite{Str83,LO13}.

Let $V_1,V_2,V_3$ be vector spaces with $\dim V_i = n_i+1$. Fix $p \leq n_1$. For $T \in V_1 \otimes V_2 \otimes V_3$, let $T_1^{\wedge p} $ be the composition
\[
\Lambda^p V_1 \otimes V_2^*  \xto{\id_{\Lambda^p V_1} \boxtimes T} \Lambda^p V_1 \otimes V_1 \otimes V_3 \to \Lambda^{p+1} V_1 \otimes V_3.
\]
In this case, one has the following result.
\begin{proposition}
 \label{RepTheory-proposition-KoszulFlat}
% Author: Fulvio Gesmundo
 Let $T \in V_1 \otimes V_2 \otimes V_3$ be a tensor. If $T \in \bbP V_1 \times \bbP V_2 \times \bbP V_3$, then $\rank( T_1^{\wedge p}) = \binom{n}{p}$. In particular, if $\rank ( T_1^{\wedge p} ) \geq R$, then 
 \[
T \notin \sigma_r ( \bbP V_1 \times \bbP V_2 \times \bbP V_3) 
 \]
for $r = \Bigl\lfloor R / \binom{n}{p} \Bigr\rfloor$. In other words $\uR(T) \geq r+1$.
\end{proposition}



Over time, strong \emph{barriers} for flattening methods for secant varieties have been proved. This means that they can only provide equations for secant varieties up to a certain value $r$, much smaller than the value $r_{\gen}$ of the generic rank. These barriers are discussed in \cite{EGOW18} from the point of view of complexity theory. A geometric reason is provided in \cite{Gal17}, in conjunction with results on the value of generic and maximum cactus rank \cite{BR13,BBG19}. 




\section{Equations for Chow varieties}
\label{RepTheory-section-chowvarieties}
% Author: Fulvio Gesmundo

Let $V$ be a vector space. The Chow variety of completely reducible forms of degree $d$ is
\[
\Ch_d(V) = \{ \ell_1 \cdots \ell_d : \ell_i \in \bbP V\} \subseteq \bbP S^d V.
\]
It is not hard to see that $\Ch_d(V)$ is an algebraic variety. 

Note that if $\dim V = 2$, then $\Ch_d(V) = \bbP S^d V$. Brill \cite{Bri98} and Gordan \cite{Gor94} determined set-theoretic equations for $\Ch_d(V)$ which are now known as Brill's equations in terms of a flattening map. In this section, we describe Brill's equations following \cite{Lan12,Gua18}. In particular, they are based on the construction of a flattening map $S^d V \to \Hom(S^{d(d-1)} V^*, \bbS_{(d,d)} V )$, which is identically $0$ on $\Ch_d$. 

In order to describe Brill's map, we introduce the following notation. Let $\dim V = n+1$.
\begin{itemize}
 \item $\pi_{(d,d)}: S^d V \otimes S^d V \to \bbS_{(d,d)} V$ is the equivariant projection onto the component $ \bbS_{(d,d)}$ of $S^d V \otimes S^d V$ which is unique by \ref{introduction-proposition-Pieri}.
 \item For every $k$, and every $e$, define
 \[
 \begin{aligned}
 E_k : S^e V &\to S^{k} V \otimes S^{k(e-1)}V  \\
 f &\mapsto \sum_{\alpha \in \bbN^{n+1}, |\alpha| = k} \binom{k}{\alpha}^{-1} \bfx^\alpha \otimes \left( f^{k-1}\textstyle \frac{\partial^k}{\partial \bfx^\alpha} f \right)
\end{aligned}
 \]
 which is a polynomial map. Elements in the image of $E_k$ can be multiplied according to the natural product 
 \[
( S^a V \otimes S^{a(e-1)} ) \times ( S^b V \otimes S^{b(e-1)} )  \to ( S^{a+b} V \otimes S^{(a+b)(e-1)} ).
\]
 \item Let $\calP_d ( e_1 \vvirg e_d)$ be the polynomial expression of the power sum polynomial of degree $d$ in terms of the elementary symmetric functions $e_1 \vvirg e_d$. 
 \item Let 
 \[
 \begin{aligned}
 \calQ_{d} : S^d V \to S^d V \otimes S^{d(d-1)}V \\
 f \mapsto \calP_d ( E_1(f) \vvirg E_d(f)).
 \end{aligned}
 \]
\end{itemize}
Define the Brill map $\calB(f) \in \Hom ( S^{d(d-1)}V^* , \bbS_{(d,d)} V)$ associated to $f \in S^d V$ to be the composition
\[
\begin{array}{rcccl}
S^{d(d-1)}V^* &\to & S^{d(d-1)}V^* \otimes S^d V \otimes S^d V \otimes S^{d(d-1)} V & \to & \bbS_{(d,d)} V \\
\Delta &\mapsto &\Delta \otimes f \otimes Q_d(f)  & &\\
& & \Delta \otimes g \otimes h \otimes H &\mapsto & D(H) \cdot \pi_{(d,d)} (g \otimes h)
\end{array}
 \]






\chapter{Classification results}
\label{RepTheory-chapter-classifications}
This chapter collects orbit classification results for the action of algebraic groups on spaces of tensors. The classification is usually possible only when the representation structure is either \emph{finite} or \emph{tame}: respectively, this means that there are finitely many orbits, or the orbits are in one-to-one correspondence with finitely many families in finitely many parameters. These definitions can be made more precise following \cite{Vin79} and \cite{Gab72}.

The only tensor spaces with finitely many orbits are subspaces of $\bbC^2 \otimes \bbC^3 \otimes \bbC^6$ or its permutations. The list of the orbits, and a description of the degeneration poset, is given in \ref{RepTheory-orbitclassification-section-236}.

There are, instead, a few cases where the structure is tame:
\begin{itemize}
 \item $\GL_2 \times \GL_m \times \GL_n$ acting on $\bbC^2 \otimes \bbC^m \otimes \bbC^n$;
 \item $\GL_3 \times \GL_3 \times \GL_3$ acting on $\bbC^3 \otimes \bbC^3 \otimes \bbC^3$;
 \item $\GL_2 \times \GL_2 \times \GL_2 \times \GL_2$ acting on $\bbC^2 \otimes \bbC^2 \otimes \bbC^2 \otimes \bbC^2$.
\end{itemize}
The case of format $(2,m,n)$ dates back to Kronecker, see, e.g., \cite[Ch.XIII]{Gan59}. In general, orbit classification results can be achieved using classical representation theoretic methods \cite{Vin79}. More recent references include \cite{Nur00} for the case of format $(3,3,3)$ and \cite{CD12} for the case of format $(2,2,2,2)$.

% In the setting of symmetric tensors, there are two cases with finite representation type:
% \begin{itemize}
%  \item $\GL_n$ acting on $S^2 \bbC^n$;
%  \item $\GL_2$ acting on $S^3 \bbC^n$;
% \end{itemize}
% moreover, there are two cases with tame representation type:
% \begin{itemize}
%  \item $\GL_2$ acting on $S^4 \bbC^2$;
%  \item $\GL_3$ acting on $S^3 \bbC^3$.
% \end{itemize}
% 
% Invariant theoretic aspects are discussed in \ref{RepTheory-chapter-invariantTheory}.

\section{Normal forms for format $(2,3,6)$}
\label{RepTheory-orbitclassification-section-236}
% Author: Fulvio Gesmundo

We follow the presentation of \cite[Sec. 10.3]{Lan12}. Fix bases 
\begin{align*}
&a_0,a_1 \text { of } \bbC^2, \\
&b_0\vvirg b_2 \text{ of } \bbC^3, \\ 
&c_0\vvirg c_5 \text{ of } \bbC^6. \\ 
\end{align*}
The action of $\GL_2 \times \GL_3 \times \GL_6$ on $\bbC^2 \otimes \bbC^3 \otimes \bbC^6$ has $27$ orbits. We list representatives for these orbits highlighting their multilinear. 

\begin{itemize}
 \item Multilinear ranks $(0,0,0)$: 
 \[
  T_0 = 0;
 \]
\item Multilinear ranks $(1,1,1)$:
\[
T_1 = a_0 \otimes b_0 \otimes c_0;
\]
\item Multilinear ranks $(1,2,2)$:
\[
T_2 = a_0 \otimes b_0 \otimes c_0 + a_0 \otimes b_1 \otimes c_1;
\]
\item Multilinear ranks $(2,1,2)$:
\[
T_3 = a_0 \otimes b_0 \otimes c_0 + a_1 \otimes b_0 \otimes c_1;
\]
\item Multilinear ranks $(2,2,1)$:
\[
T_4 = a_0 \otimes b_0 \otimes c_0 + a_1 \otimes b_1 \otimes c_0;
\]
\item Multilinear ranks $(2,2,2)$:
\begin{align*}
T_5 &= a_0 \otimes b_0 \otimes c_1 + a_0 \otimes b_1 \otimes c_0 + a_1 \otimes b_0 \otimes c_0, \\
T_6 &= a_0 \otimes b_0 \otimes c_0 + a_1 \otimes b_1 \otimes c_1;
\end{align*}
\item Multilinear ranks $(2,2,3)$:
\begin{align*}
T_7 &= a_0 \otimes b_0 \otimes c_0 + a_0 \otimes b_1 \otimes c_1 + a_1 \otimes b_0 \otimes c_2, \\
T_8 &= a_0 \otimes b_0 \otimes c_0 + a_1 \otimes b_1 \otimes c_1 + (a_0 + a_1) \otimes (b_0 + b_1) \otimes c_2;
\end{align*}
\item Multilinear ranks $(2,2,4)$:
\[
T_9 = a_0 \otimes b_0 \otimes c_0 + a_0 \otimes b_1 \otimes c_1 + a_1 \otimes b_0 \otimes c_2 + a_1 \otimes b_1 \otimes c_3;
\]
\item Multilinear ranks $(1,3,3)$:
\[
T_{10} = a_0 \otimes (b_0 \otimes c_0 + b_1 \otimes c_1 + b_2 \otimes c_2)
\]
\item Multilinear ranks $(2,3,2)$:
\begin{align*}
T_{11} &= a_0 \otimes b_0 \otimes c_0 + a_0 \otimes b_1 \otimes c_1 + a_1 \otimes b_2 \otimes c_0, \\
T_{12} &= a_0 \otimes b_0 \otimes c_0 + a_1 \otimes b_1 \otimes c_1 + (a_0 + a_1) \otimes b_2 \otimes (c_0 + c_1);
\end{align*}
\item Multilinear ranks $(2,3,3)$:
\begin{align*}
T_{13} &= a_0 \otimes (b_0 \otimes c_1 + b_1 \otimes c_0) + a_1 \otimes (b_0 \otimes c_2 + b_2 \otimes c_0), \\
T_{14} &= a_0 \otimes (b_0 \otimes c_0 +  b_1 \otimes c_1) + a_1 \otimes b_2 \otimes c_2,\\
T_{15} &= a_0 \otimes (b_0 \otimes c_0 +  b_1 \otimes c_1 + b_2 \otimes c_2) + a_1 \otimes b_0 \otimes c_1,\\
T_{16} &= a_0 \otimes (b_0 \otimes c_0 +  b_1 \otimes c_1 + b_2 \otimes c_2) + a_1 \otimes (b_0 \otimes c_1 + b_1 \otimes c_2),\\
T_{17} &= a_0 \otimes (b_0 \otimes c_0 +  b_1 \otimes c_1) + a_1 \otimes (b_2 \otimes c_2 + b_0 \otimes c_1 ),\\
T_{18} &= a_0 \otimes b_0 \otimes c_0 +  (a_0+a_1) \otimes b_1 \otimes c_1 + a_1 \otimes (b_2 \otimes c_2);
\end{align*}
\item Multilinear ranks $(2,3,4)$:
\begin{align*}
T_{19} &= a_0 \otimes (b_0 \otimes c_0 + b_1 \otimes c_1 + b_2 \otimes c_3) + a_1 \otimes (b_0 \otimes c_1 + b_1 \otimes c_2), \\
T_{20} &= a_0 \otimes (b_0 \otimes c_0 +  b_1 \otimes c_2 + b_2 \otimes c_3) + a_1 \otimes b_0 \otimes c_1,\\
T_{21} &= a_0 \otimes (b_0 \otimes c_0 +  b_1 \otimes c_2 + b_2 \otimes c_3) + a_1 \otimes (b_0 \otimes c_1 +b_1 \otimes c_3),\\
T_{22} &= a_0 \otimes (b_0 \otimes c_0 +  b_1 \otimes c_2) + a_1 \otimes (b_0 \otimes c_1 +b_2 \otimes c_3),\\
T_{23} &= a_0 \otimes (b_0 \otimes c_0 +  b_1 \otimes c_1 + b_2 \otimes c_2) + a_1 \otimes (b_0 \otimes c_1 + b_1 \otimes c_2 + b_2 \otimes c_3);\\
\end{align*}
\item Multilinear ranks $(2,3,5)$:
\begin{align*}
T_{24} &= a_0 \otimes (b_0 \otimes c_0 + b_1 \otimes c_1 + b_2 \otimes c_2) + a_1 \otimes (b_1 \otimes c_3 + b_2 \otimes c_4), \\
T_{25} &= a_0 \otimes (b_0 \otimes c_0 +  b_1 \otimes c_1 + b_2 \otimes c_2) + a_1 \otimes (b_0 \otimes c_2 + b_1 \otimes c_3 + b_2 \otimes c_4);\\
\end{align*}
\item Multilinear ranks $(2,3,6)$:
\[
T_{26} = a_0 \otimes (b_0 \otimes c_0 +  b_1 \otimes c_1 + b_2 \otimes c_2) + a_1 \otimes (b_0 \otimes c_3 + b_1 \otimes c_4 + b_2 \otimes c_5).
\]
\end{itemize}

% $$
% \xymatrix{
%  & & & 26 & & & & \\
%  & & 25  & & & & & \\
% } 
% $$
% 
% The degeneration poset is described in [Figure to be inserted]. The proof of the relevant restrictions and degenerations is given by describing the explicit restrictions and degenerations between the tensors of the classification. The linear maps are given in bases; vectors for which the image is not specified are mapped to themselves.
% \begin{itemize}
%  \item $T_{26}$ restricts to $T_{25}$ via
%  \[
%  \begin{array}{l}
%   c_3 \mapsto c_{2}\\
%   c_4 \mapsto c_{3}\\
%   c_5 \mapsto c_{4}
%  \end{array}
% \] 
%  \item $T_{25}$ degenerates to $T_{24}$ via
%  \[
%  \begin{array}{lll}
%   a_1 \mapsto \eps a_1 & ~ & c_3 \mapsto \eps^{-1} c_3 \\
%   & & c_4 \mapsto \eps^{-1} c_4
%  \end{array}
% \] 
% \item $T_{25}$ restricts to $T_{23}$ via 
%  \[
%  \begin{array}{lll}
% b_1 \mapsto b_2 & ~&  c_1 \mapsto c_{2} \\
% b_2 \mapsto b_1 & ~& c_2 \mapsto c_{1} \\
% & & c_4 \mapsto c_{2}
%   \end{array}
% \] 
% \end{itemize}

\subsection{Proof of the classification}


Let $T \in V_1 \otimes V_2 \otimes V_3$ with $\dim V_1 =2, \dim V_2 = 3$ and $\dim V_3 = 6$. Let $(r_1,r_2,r_3)$ be the multilinear ranks of $T$. By \ref{preliminaries-lemma-multilinearranksineq}, we have the three inequalities
\[
r_1 \leq r_2r_3, \quad r_2 \leq r_1r_3, \quad r_3 \leq r_1r_2
\]
and $T$ is concise in $\bbC^{r_1} \otimes \bbC^{r_2} \otimes \bbC^{r_3} \subseteq V_1 \otimes V_2 \otimes V_3$. We deduce that the only possible multilinear ranks are the ones listed in the classification.

Clearly, if $r_i = 0$ for any $i$, then $r_i = 0$ for all $i$ and $T = 0$.

Suppose one of the multilinear ranks is $r_i = 1$. If $r_1 = 1$, then the inequalities guarantee $r_2 = r_3$. Using the action of $\GL(V_1)$, we obtain $T = a_0 \otimes M$ for some $M \in V_2 \otimes V_3$. Since $r_2 = r_3$, we may use the action of $\GL(V_2) \times \GL(V_3)$ to normalize $M = \sum_{0}^{r_2-1} b_i \otimes c_i$. For $r_2 = r_3 = 1,2,3$, these are cases $T_1,T_2,T_{10}$ of the classification.  Similarly, if $r_2 = 1$ (and $r_1 = r_3 = 2$) or $r_3  = 1$ (and $r_1 = r_2 = 2$), we obtain cases $T_3$ and $T_4$ of the classification, respectively.

If $r_1 = 2$, let $L_T = \bbP \Im( T: V_1^* \to V_2 \otimes V_3) \subseteq \bbP (V_2 \otimes V_3)$ be the projective line image of the first flattening. The rest of the proof is based on the position of $L_T$ with respect to the varieties of matrices of bounded rank in $\bbP (V_2 \otimes V_3)$.

{\it (to be completed)}

% \chapter{The GCT approach}
% \label{RepTheory-chapter-GCT}
% 
\chapter{Invariant theory}
\label{RepTheory-chapter-invariantTheory}



\part{Tensor Algorithms}
\label{part-TensorAlgorithms}
\chapter{Algorithms for $ X $-rank}
\label{rankAlgorithms-chapter-intro}
% Alexander Blomenhofer

Throughout this section, let $ X \subseteq \mathbb C^N  $ be a conic irreducible affine variety. 
The $ X $-rank problem is to find, given $ T\in \langle X \rangle $, the minimal $ r\in \mathbb N_0 $ and $ x_1,\ldots,x_r \in X $ such that 
\begin{align*}
	T = x_1 + \ldots + x_r. 
\end{align*}
In general, $ X $-rank problems are hard to solve. However, for several special varieties $ X $, efficient algorithms are known that work in special situations; especially when $ T $ has low $ X $-rank. 
In this chapter, we will collect several known algorithmic results. 


\section{Symmetric tensor rank}
\label{section-symmetricTensorRank}
% Alexander Blomenhofer

The symmetric tensor rank problem asks to write a symmetric tensor $ T\in S^d(\mathbb C^n) $ as
\begin{align*}
	T = a_1^{\otimes d} + \ldots + a_r^{\otimes d}, 
\end{align*}
where $ a_1,\ldots,a_r\in \mathbb C^n $ and $ r \in \mathbb{N}_0$ is minimal. It is an $ X $-rank problem, where $ X = \nu_d(\mathbb C^n) $ is the Veronese variety of degree $ d $. %The symmetric tensor rank problem is equivalent to the Waring rank, where one aims to write a $ d $-form $ f\in \C[x_1,\ldots,x_n] $ as a sum 
%\begin{align*}
%	f = \ell_1^d  + \ldots + \ell_r^d
%\end{align*}
%of linear forms, with minimal $ r $. 



\subsection{Algorithms for symmetric 3-tensors}
\label{subsection-algorithmsSymmetric3Tensors}
% Alexander Blomenhofer

The following algorithm is classical, but attribution to a single source is difficult. %While attribution to a single author is difficult, it is reasonable to credit Sylvester and Prony. Sylvester solved the case $ n = 2 $ (and general $ d $). Prony is (...)


\begin{theorem}
	\label{tensorAlgorithms-theorem-3tensor}
	% Alexander Blomenhofer
	If $ T \in S^3(\mathbb C^n) $ is concise and of rank at most $ n $, then the rank of $ T $ equals $ n $. Its minimum rank decomposition is unique (up to permutations and third roots of unity).
	Furthermore, there exists a linear-time algorithm, stated below, which computes the minimum rank decomposition for all concise tensors $ T\in S^3(\mathbb C^n) $ of rank $ n $. 
\end{theorem}
\begin{proof}[Proof of uniqueness]
	As $ T $ has rank at most $ n $, we may write $ T = \sum_{i = 1}^{n} a_i^{\otimes 3} $. Since $ T $ is concise, we have $ \langle a_i \mid i=1,\ldots,n \rangle = \mathbb{C}^n $. This shows that $ a_1,\ldots,a_n $ are linearly independent. Denote by $ T_f $ the contraction of $ T $ by $ f \in \mathbb{C}^n $, i.e., the matrix obtained from the linear operation $ T\mapsto (f^{T} \otimes I_n \otimes I_n) T $. We consider the space $ \mathcal{L}_T $ of all contractions of $ T $, which is
	\begin{align*}
		\mathcal{L}_T = \{\sum_{i = 1}^{n} \langle a_i, f \rangle a_ia_i^{T} \} = \langle a_1a_1^{T},\ldots,a_na_n^{T} \rangle. 
	\end{align*}
	Here, the first equality is immediate from applying the definition of the contraction to the decomposition of $ T $. For the second equality, note that since $ a_1,\ldots,a_n $ are linearly independent, we may find $ f_j\in \mathbb{C}^n $ such that $ \langle a_i, f_j \rangle = \delta_{ij} $, which shows that $ a_ja_j^{T} $ lies in $ \mathcal{L} $ for each $ j = 1,\ldots,n $. 
	
	\smallskip\noindent
	A general element of $ \mathcal{L}_T $ has rank $ n $. We consider the variety 
	\begin{align*}
		X_T = \{T_f \mid \det(T_f) = 0\}, 
	\end{align*}
	which consists of those matrices $ T_f $ that do not have full rank. We claim that $ X_T $ is a union of subspaces of codimension $ 1 $. Precisely, we have 
	\begin{align*}
		X_T = \bigcup_{i = 1}^n \langle a_i \rangle^{\perp}. 
	\end{align*} 
	Indeed, as $ a_1,\ldots,a_n $ are linearly independent, the identity $ T_f = \langle a_i, f \rangle a_ia_i^{T} $ is a (matrix) rank decomposition of $ T_f $. The matrix $ T_f $ can only have rank lower than $ n $, if some of the inner products $ \langle a_i, f \rangle $ vanish. 
	
	To see uniqueness of the minimum rank decomposition, assume that $ T = \sum_{i = 1}^{r} b_i^{\otimes 3} $ for any $ r\le n $. By conciseness, we also have $ \langle b_i \mid i=1,\ldots,r \rangle = \mathbb{C}^n $. Therefore, $ r = n $ and $ b_1,\ldots,b_n\in \mathbb{C}^n $ are linearly independent. We thus similarly obtain $ X_T = \bigcup_{i = 1}^n \langle b_i \rangle^{\perp}$. By uniqueness of the irreducible decomposition, we conclude $ a_i = b_{\sigma(i)} $ for some permutation $ \sigma $ of $ 1,\ldots,n $ and all $ i = 1,\ldots,n $.  
\end{proof}

The uniqueness result of is a special case of Kruskal's theorem, with a simpler proof. The advantage of the above proof is that it can be algorithmized in a straightforward way; and efficiently solved by methods for generalized eigenvalue problems.
A summary of the algorithm is as follows. 

\begin{algorithm}
	\label{tensorAlgorithms-algorithm-3tensor}
	% Alexander Blomenhofer
	\hfill\\
	\textbf{Input: } A concise tensor $ T\in S^3(\mathbb C^n) $ of rank $ n $. \\
	\textbf{Output: } The unique minimum rank decomposition of $ T $. \\
	\textbf{Procedure: } \\
	\begin{enumerate}
		\item Pick general $ f, g\in \mathbb{C}^n $ and compute the $ n $ eigenvector-eigenvalue pairs $ (x_i, \lambda_i) $ of the generalized eigenvalue problem $ T_{f}x = \lambda T_{g}x $. The $ \lambda_1,\ldots,\lambda_n $ will be pairwise distinct. 
		\item Set $ b_i := T_fx_i $ for $ i = 1,\ldots,m $.	(Each $ b_i $ will be a multiple of some $ a_i $. )
		\item Solve the $ n\times n $ linear system $ T_f f \stackrel{!}{=} \sum_{i = 1}^{n} \alpha_i b_i $ in variables $ \alpha_1,\ldots,\alpha_n\in \mathbb{C} $. 
		%		\item Compute the quantities $ \beta_i = \langle f, b_i \rangle $ for $ i = 1,\ldots,m $
		\item Output the set of vectors $ \sqrt[3]{\frac{\alpha_i}{\langle b_i, f \rangle^2}} b_i $, where $ i = 1,\ldots,n $. 
	\end{enumerate}
\end{algorithm}
\begin{proof}[Analysis]
	\textbf{Runtime:} Computing the contractions $ T_f $ and $ T_g $ takes $ \mathcal{O}(n^3) $ arithmetic operations. Computing the generalized eigenvalues of the two $ n\times n $ matrices takes time at most $ \mathcal{O}(n^3) $. Indeed, one way to do so is to compute the eigenvalues of $ T_g^{-1}T_f $. Both the inversion and the multiplication to compute $ T_g^{-1}T_f $ also take time $ \mathcal{O}(n^3) $. The linear system from step 3 of size $ n\times n $ can be solved in time $ \mathcal{O}(n^3) $. The remaining operations are of negligible complexity. As the encoding size of $ T $ is in $ \mathcal{O}(n^3) $, the algorithm thus runs in linear time. 
	
	\smallskip\noindent
	\textbf{Correctness: } We analyze the algorithm step by step. 
	\begin{enumerate}
		\item The first step computes the $ n $ intersection points of the variety $ X_T $ with the line $ \ell= \{f+\lambda g \mid \lambda\in \mathbb{C}\} $. Indeed, the intersection points are given by the equation $ 0 = \det(T_{f-\lambda g}) = \det(T_{f} - \lambda T_{g}) $, which is a generalized eigenvalue problem. Observe that the generalized eigenvalues $ \lambda_1,\ldots,\lambda_n \in \mathbb{C} $ are (after reordering of the $ \lambda_i $'s) of the form $ \lambda_i = \frac{\langle f, a_j \rangle}{\langle g,a_i \rangle} $. The reason is that from the proof of the previous theorem, $ X_T = \bigcup_{i = 1}^n \langle a_i \rangle^{\perp} $ and therefore, $ 0 = \langle a_j, f-\lambda_ig \rangle $ for some $ j $. 
		This shows that there are unique corresponding unit-length eigenvectors $ x_1,\ldots,x_m $. 
		\item First, note that $ f-\lambda_ig $ is orthogonal to $ a_i $, but not to any other $ a_j $ with $ j\ne i $. Therefore, there are nonzero scalars $ \alpha_1,\ldots,\alpha_n $ such that $ T_{f-\lambda_ig} = \sum_{\substack{j = 1\\j\ne i}}^{n} \alpha_ja_ja_j^{T}$. From the eigenequation $ T_{f-\lambda_ig} x_i = 0 $ and linear independence of $ a_1,\ldots,a_n $, we obtain that $ \langle x_i, a_j \rangle = 0 $ whenever $ i\ne j $. 
		
		Therefore, we have that $ b_i = T_fx_i = \langle f, a_i \rangle \langle x, a_i \rangle a_i$, which is a multiple of $ a_i $. 
		\item It remains to correct the scalar multiples. Precisely, we need to find the values $ \gamma_i \in \mathbb{C}$ such that $ \gamma_i b_i = a_i $. This is done in the last step. We can compute all the scalars $ \langle f, b_i \rangle = \langle a_i, f \rangle^2 \langle a_i, x_i \rangle $ in time $ \mathcal{O}(n^2) $. 
		As $ T_f f  = \sum_{i = 1}^{n} \langle a_i, f \rangle^2 a_i $ and $ a_1,\ldots,a_n $ are linearly independent, we can solve the $ n\times n $ linear system $ T_f f \stackrel{!}{=} \sum_{i = 1}^{n} \alpha_i b_i $ to calculate the unique solution, which is given by $ \alpha_i = \langle a_i, f \rangle \langle x_i, a_i \rangle^{-1}  $ for $ i = 1,\ldots,n $.  Solving the $ n\times n $ system takes time $ \mathcal{O}(n^3) $.  Therefore, we have computed the quantity $ \alpha_i\langle f, b_i \rangle = \langle a_i, f \rangle^3 = \gamma_i^3 \langle b_i, f \rangle^3$. The equation $ \alpha_i = \gamma_i^3 \langle b_i, f \rangle^2 $ determines the correcting factor $ \gamma_i $ up to the $ 3 $ roots of unity. One may choose an arbitrary third root and set $ \gamma_i =  \sqrt[3]{\frac{\alpha_i}{\langle b_i, f \rangle^2}} $. 
	\end{enumerate}
\end{proof}


%\subsection{Algorithms for odd degree symmetric ranks}
%
%\subsection{Algorithms for even degree symmetric ranks}
%
%\section{Higher-order Waring decomposition}
%
%\section{Chow and skew decompositions}



\part{Optimization Theory}
\label{part-optimization}

\chapter{Introduction}
\label{optimization-chapter-intro}

\part{Complexity Theory}
\label{part-complexityTheory}
\input{part-complexitytheory.tex}

\part{Quantum Physics}
\label{part-quantumph}

\chapter{Introduction}
\label{quantumph-chapter-intro}

\part{Data Science}
\label{part-datascience}

\chapter{Introduction}
\label{datascience-chapter-intro}

\bibliographystyle{amsalpha}
\bibliography{tensor}
\end{document}
