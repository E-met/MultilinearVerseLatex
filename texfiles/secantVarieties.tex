\begin{definition}[Join of varieties]
\label{classicalAG-definition-join}
Let $X,Y \subseteq \mathbb{P}^N$ be two algebraic varieties. The {\it (geometric) join} of $X$ and $Y$ is 
\[
\bbJ(X,Y) =  \overline{ \{ p \in \mathbb{P}^N : p \in \langle x , y \rangle \text{ for some $x \in X, y \in Y$}\}}
\]
where the overline denotes the topological closure, equivalently in the Zariski or the Euclidean topology.

Given varieties $X_1 \vvirg X_s$, one can define their geometric join recursively
\[
\bbJ( X_1 \vvirg X_s) = \bbJ( \bbJ(X_1 \vvirg X_{s-1}), X_s).
\]
\end{definition}
There is a standard construction which realizes the join of varieties as the image of an \emph{incidence correspondence}. Given varieties $X_1 \vvirg X_s \subseteq \bbP^N$, consider the set
\[
A\bbJ^\circ(X_1 \vvirg X_s) = \{ (x_1 \vvirg x_s, p) \in X_1 \ttimes X_s \times \bbP^N : p \in \langle x_1 \vvirg x_s\rangle\}
\]
and let $A\bbJ(X_1 \vvirg X_s) = \overline{A\bbJ^\circ(X_1 \vvirg X_s)}$ be its closure in $X_1 \ttimes X_s \times \bbP^N$. The variety $A\bbJ(X_1 \vvirg X_s)$ is called the \emph{abstract join} of $X_1 \vvirg X_s$. The natural projection $A\bbJ(X_1 \vvirg X_s)  \to \bbP^N$ surjects onto $\bbJ(X_1 \vvirg X_s)$. On the other hand, the projection $A\bbJ(X_1 \vvirg X_s) \to X_1 \ttimes X_s$ realizes, locally, $A\bbJ^\circ(X_1 \vvirg X_s)$ as a $\bbP^{s-1}$-bundle over (and open subset of) $X_1 \ttimes X_s$. In particular, this shows that if $X_1 \vvirg X_s$ are irreducible, then $\bbJ(X_1 \vvirg X_s)$ is irreducible as well.


\begin{definition}[Secant variety]
\label{classicalAG-definition-secantvariety}
Let $X \subseteq \mathbb{P}^N$ be an algebraic variety. The $r$-th {\it secant variety} of $X$ is 
\[
\sigma_r(X) = \overline{\{ p \in \mathbb{P}^N : p \in \langle x_1 , \ldots ,  x_r \rangle \text{ for some $x_1 , \ldots , x_r \in X$}
\}}
\]
where the overline denotes the topological closure, equivalently in the Zariski or the Euclidean topology. Equivalently, $\sigma_r(X) = \bbJ(\underbrace{X \vvirg X}_{r \text{ times}})$.
\end{definition}

Similarly to the case of join, one can define an incidence correspondence realizing $\sigma_r(X)$ as the image of a projective morphism. Clearly, this can be done considering the abstract join $A\bbJ( X \vvirg X)$. However, it is often convenient to consider a symmetrized version of it. Let $X^{(\cdot r)} = X^{\times r} /\frakS_r$ be the symmetrized product of $r$ copies of $X$. Define 
\[
A\sigma^\circ_r(X) = \{ (Z , p  ) \in X^{(\cdot r)} \times \bbP^N : p \in \langle Z \rangle \text{ for some } Z \in X^{\cdot r}\}
\]
and let $A\sigma_r(X) = \overline{ A \sigma^\circ_r(X)}$. As in the case of joins, the projection of $A\sigma_r(X) \to \bbP^N$ surjects onto $\sigma_r(X)$ whereas the projection $A\sigma_r(X) \to X^{(\cdot r)}$ is, on an open set, a $\bbP^{r-1}$-bundle. In particular, if $X$ is irreducible, then $A\sigma_r(X)$ and $\sigma_r(X)$ are irreducible as well.

\begin{lemma}
\label{classicalAG-lemma-expecteddimension}
Let $X_1,X_2$ be two varieties in $\bbP^N$ with $\dim X_1 = n_1, \dim X_2 = n_2$. Then 
\[
\dim \bbJ(X_1,X_2) \leq \min\{ N, n_1 + n_2 + 1\}.
\]
In particular, if $X \subseteq \bbP^N$ is a variety with $\dim X = n$, then 
\[
\dim \sigma_r(X) \leq \min\{ N , rn + r-1\}.
\]
\end{lemma}
\begin{proof}
It is easy to verify that $\dim A \bbJ(X_1,X_2) =  n_1 + n_2 + 1$, which immediately yields the result.
\end{proof}
Given varieties $X_1,X_2 \subseteq \bbP^N$ with $\dim X_i = n_i$, we say that the join $\bbJ(X_1,X_2)$ has the expected dimension if $\dim \bbJ(X_1,X_2) =  \min\{ N, n_1 + n_2 + 1\}$ and we say that it is defective otherwise. Similar terminology is used in the case of secant varieties: we say that $\sigma_r(X)$ has the expected dimension if $\dim \sigma_r(X) = \min\{ N , r \dim X + r-1\}$; we say that $\sigma_r(X)$ is defective, or that $X$ is $r$-defective, if it does not have the expected dimension. 

\section{Dimension of secant varieties}
\label{classicalAG-section-dimension}

Throughout more than a century, a great deal of research was devoted in the study of dimension of secant varieties, and in the classification of defective varieties. A very first result in this direction is \ref{classicalAG-lemma-palatini} which appears already in \cite{Pal09}. 

\begin{lemma}[Palatini's Lemma]
\label{classicalAG-lemma-palatini}
Let $X \subseteq \bbP^N$ be an algebraic variety with $\codim X \geq 2$ and let $C \subseteq \bbP^N$ be a linearly non-degenerate algebraic curve. Then either $\dim \bbJ(X,C) = \dim X + 2$. In particular, all secant varieties of algebraic curves have the expected dimension.
\end{lemma}


A fundamental tool in the study of dimension of secant varieties is Terracini's Lemma, dating back to \cite{Ter11}.
\begin{lemma}[Terracini's Lemma]
\label{classicalAG-lemma-terracini}
Let $X, Y \subseteq \mathbb{P}^N$ be algebraic varieties and let $\bbJ(X,Y)$ be their join. Let $z \in \bbJ(X,Y)$ be a smooth point such that $z \in \langle x , y \rangle$ for smooth points $x \in X$ and $y \in Y$. Then
\[
T_{z} \bbJ(X,Y) = \langle T_x X , T_y Y\rangle.
\]
In particular 
\[
\dim \sigma_r(X) = \dim \langle T_{x_1} X \vvirg T_{x_r} X\rangle
\]
for generic points $x_1 \vvirg x_r \in X$.
\end{lemma}
This result offers a \emph{dual} point of view to the problem of determining the dimension of secant varieties via interpolation of fat points. Let $X$ be an algebraic variety and let $\calL$ be a (very ample) line bundle on $X$. Let $\phi_\calL : X \to \bbP H^0(X, \calL)^\vee$ be the embedding defined by $\calL$. 
\begin{proposition}
 \label{classicalAG-proposition-interpolationfatpoints}
 Let $X$ be an irreducible algebraic variety, $\calL$ a very ample line bundle on $X$. Then
 \[
 \dim \sigma_r( \phi_\calL(X)) = h^0(\calL)-1 - h^0( \calI_{Z} \otimes \calL)
 \]
 where $Z$ is the union of $r$ double points supported at generic points of $X$.
\end{proposition}




