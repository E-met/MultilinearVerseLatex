In this chapter, we introduce all basic notions, terminologies and notations that will be used through all the MultilinearVerse. 

\section{Representing tensors}
\label{introduction-section-representingtensors}

\begin{definition}[Tensor product]\label{introduction-definition-tensor_product}
Let $V,W$ be finite dimensional $\Bbbk$-vector spaces.
\begin{itemize}
    \item Let $\Hom(V,W)$ denote the vector space of linear functions from $V$ to $W$.
    \item Let $V^* = \Hom(V,\Bbbk)$ denote the \emph{dual space} of $V$.
    \item Let $V \otimes W$ denote the tensor product of $V$ and $W$: it can be equivalently identified with the space of linear maps $\Hom(V^*,W) \simeq \Hom(W^*,V)$ or with the space of bilinear map $V \times W \to \bbC$.
\end{itemize}
Given $\Bbbk$-vector spaces $V_1 \vvirg V_d$, let $V_1 \ootimes V_d$ denote their tensor product, which is identified with the space of multilinear maps $V_1^* \ttimes V_d^* \to \bbC$. If $V_1 = \cdots = V_d = V$, then we write for short $V^{\otimes d}$. 

The elements of a tensor product of $d$ vector spaces are called \emph{tensors} of \emph{order $d$}.

In coordinates, if $\{v_0,\ldots,v_m\}$ is a basis of $V$ and $\{w_0,\ldots,w_n\}$ is a basis of $W$, then $\{v_i \otimes w_j ~:~ i \in \{0,\ldots,m\}, j \in \{0,\ldots,n\}$ is a basis of $V\otimes W$. Therefore, any element $T \in V\otimes W$ can be written uniquely as 
\[
    T = \sum_{i,j} t_{i,j}v_i\otimes w_j;
\]
therefore, the choice of basis gives an identification of the space $V\otimes W$ with the space of $m\times n$ \emph{matrices} with coefficients in $\Bbbk$. In general, if $\{v_0^{(i)},\ldots,v_{n_i}^{(i)}\}$ is a basis of $V_i$, then any element $T \in V_1 \ootimes V_d$ can be written uniquely as 
\[
    T = \sum_{i_1,\ldots,i_d} t_{i_1,\ldots,i_d}v_{i_1}^{(1)}\ootimes v_{i_d}^{(d)};
\]
therefore, the choice of basis gives an identification of the space $V_1 \ootimes V_d$ with the space of $d$-dimensional arrays of size $n_1 \times\cdots\times n_d$. 
\end{definition}

\begin{definition}[Flattenings]\label{introduction-definition-flattenings}
For every $K \subseteq \{1 \vvirg d\}$, let $K^c = \{1\vvirg d\} \smallsetminus K$. Then, a tensor $T \in V_1 \ootimes V_d$ defines a linear map 
\[
    \Flat_K(T): \bigotimes_{k \in K} V_{k}^* \to  \bigotimes_{k' \in K^c} V_{k'}
\]
called \emph{flattening} of $T$. It is easy to see that $\Flat_K(T) = (\Flat_{K^c}(T))^\bft$. 
\end{definition}

\begin{definition}[Concise tensors]\label{introduction-definition-concise}
    A tensor $T$ is \emph{concise} if the flattenings $T_k : V_k^* \to \bigotimes_{k' \neq k} V_{k'}$ are injective. 
\end{definition}

A tensor $T \in V_1 \ootimes V_d$ can be identified with the image of its flattening $T : V_1^* \to V_2 \ootimes V_d$, up to a change of coordinates on the space $V_1$. Here, we consider the action of $\GL(V_1)$ on $V_1 \ootimes V_d$: for more about group actions, see \ref{introduction-section-groupactions}. In other words, if $T,T'$ are two tensors such that the image of their first flattening is the same subspace of $V_2 \ootimes V_d$, then there exists an element $g \in \GL(V_1)$ such that $T = g\cdot T'$. Similar statements hold with respect to the other simple flattenings.

\subsection{Symmetric Tensors}
\label{introduction-subsection-symmetric_tensors}

On the space of tensors $V^{\otimes d}$, we consider the action of the \emph{symmetric group} $\frakS_d$ which acts by permuting the tensor factors: if $T = (t_{i_1,\ldots,i_d}) \in V^{\otimes d}$, then, for any $\sigma \in \frakS_d$, then \[(\sigma \cdot T)_{i_1,\ldots,i_d} = t_{i_{\sigma(1)},\ldots,i_{\sigma(d)}}\] for any multi-index $(i_1,\ldots,i_d)$. 
\begin{definition}[Symmetric tensors]
    \label{introduction-definition-symmetric_tensors}

    A tensor $T \in V^{\otimes d}$ which is invariant under the action of $\frakS_d$ is said to be \emph{symmetric}. Let $S^d V \subseteq V^{\otimes d}$ denote the space of symmetric tensors. 

    If the base field has characteristic $0$, then $S^d V$ can be identified with the space of \emph{homogeneous polynomials} of degree $d$ on $V^*$. For example, the monomial $xy^2$ is identified with the symmetric tensor $\frac{1}{3}\left(x\otimes y \otimes y + y\otimes x \otimes y + y\otimes y \otimes x\right)$.
\end{definition}

\subsection{Partially Symmetric Tensors}
\label{introduction-subsection-partially_symmetric_tensors}

On the space of tensors $V_1^{\otimes d_1}\ootimes V_m^{\otimes d_m}$, we consider the action of the \emph{symmetric group} $\frakS_{d_1} \ttimes \frakS_{d_m}$ which acts by permuting the tensor factors accordingly to the partition: namely, if $T = T_1 \ootimes T_m \in V_1^{\otimes d_1}\ootimes V_m^{\otimes d_m}$, with $T_i \in V_i^{\otimes d_i}$, then, for any $\sigma = (\sigma_1,\ldots,\sigma_m)\in \frakS_{d_1} \ttimes \frakS_{d_m}$, then \[\sigma \cdot T = (\sigma_1\cdot T_1) \ootimes (\sigma_m\cdot T_m) \in V_1^{\otimes d_1}\ootimes V_m^{\otimes d_m},\] where each $\sigma_i$ acts permuting the factors of $T_i$ as explained in \ref{introduction-definition-symmetric_tensors}.
\begin{definition}[Partially symmetric tensors]
    \label{introduction-definition-partially_symmetric_tensors}
    
    A tensor $T \in V_1^{\otimes d_1}\ootimes V_m^{\otimes d_m}$ which is invariant under the action of $\frakS_{d_1} \ttimes \frakS_{d_m}$ is said to be \emph{partially symmetric}. Let $S^{d_1,\ldots,d_m}(V_1,\ldots,V_m) = S^{d_1} V_1 \ootimes S^{d_m}V_m \subseteq V_1^{\otimes d_1}\ootimes V_m^{\otimes d_m}$ denote the space of partially symmetric tensors. 

    If the base field has characteristic $0$, then $S^{d_1} V_1 \ootimes S^{d_m}V_m \subseteq V_1^{\otimes d_1}\ootimes V_m^{\otimes d_m}$ can be identified with the space of \emph{multi-homogeneous polynomials} of multi-degree $(d_1,\ldots,d_m)$ on $V^*_1 \ttimes V^*_m$. 
\end{definition}

Inside the space of tensors $V^{\otimes d}$, we have all possible spaces of partially symmetric tensors $S^{\underline{d}}(V) = S^{d_1} V \ootimes S^{d_m}V$ for all partitions $\underline{d} = (d_1,\ldots,d_m)\vdash d$. In particular, $S^{\underline{d}}(V) \subset S^{\underline{d'}}(V)$ if $\underline{d}'$ is a refinement of $\underline{d}$.

Fixed a basis $\{w_0,\ldots,w_m\}$ of $W$, any tensor $T \in W \otimes S^{d-1} V$ can be written as $T = \sum_{i=1}^m w_i \otimes T_i$ and then identified with the $m$-tuple $\{T_1,\ldots,T_m\}$ of elements of $S^{d-1} V$. This is also a set of generators for the image of the first flattening $\Flat_1(T)$, see \ref{introduction-definition-flattenings}.

For example, if $F \in S^d V$ is a symmetric tensor, with $\{v_0 \vvirg v_n\}$ basis of $V$, then $F$ is regarded as the element in $S^{(1,d-1)}(V)$ given by 
\[
    \sum_{j=0}^n v_j \otimes \textstyle\frac{\partial }{\partial v_j} F
\]
where the corresponding $(n+1)$-tuple of elements of $S^{d-1} V$ can be identified with the gradient of the homogeneous polynomial $F$.
\subsection{Skew-symmetric Tensors}\label{introduction-subsection-skew_symmetric_tensors}
On the space of tensors $V^{\otimes d}$, we consider the skew-symmetric action of the symmetric group $\frakS_d$: if $T = (t_{i_1,\ldots,i_d}) \in V^{\otimes d}$, then, for any $\sigma \in \frakS_d$, then $(\sigma \cdot T)_{i_1,\ldots,i_d} = \sgn(\sigma)t_{i_{\sigma(1)},\ldots,i_{\sigma(d)}}$ for any multi-index $(i_1,\ldots,i_d)$.
\begin{definition}[Skew symmetric tensors]\label{introduction-definition-skew_symmetric_tensors}
    A tensor $T \in V^{\otimes d}$ which is invariant under the skew-symmetric action of $\frakS_d$ is said to be \emph{skew symmetric}. Let $\Wedge^d V \subseteq V^{\otimes d}$ denote the space of skew-symmetric tensors. 
\end{definition}

\section{Group actions, restrictions and degenerations}
\label{introduction-section-groupactions}

\begin{definition}
\label{introduction-definition-orbitsdegenerations}
Let $G$ be a group acting on a space $V$ and let $v \in V$. The $G$-orbit of $v$ is 
\[
\Omega_v = \{ g(v) : g \in G \}.
\]
The $G$-orbit-closure $\bOmega_v$ is the closure of $\Omega_v$. We say that $w$ is $G$-isomorphic to $v$ if $w \in \Omega_v$. We say that $w$ is a degeneration of $v$ if $w \in \bOmega_v$.

Typically $V$ is a vector space, but the definition applies more generally with $V$ any topological space.
\end{definition}

If $G$ is a complex algebraic group acting algebraically on an algebraic variety $V$ (typically an affine space) then the closure defining $\bOmega_v$ can be taken equivalently in the Zariski or the Euclidean topology, see, e.g., \cite[Thm. 2.33]{Mum76}.

One can give an apparently different definition of degeneration, in terms of formal power series. Let $\bbC[[\eps]]$ be the ring of formal power series in one variable $\eps$, and let $\bbC((\eps))$ denote its quotient field, that is the field of Laurent series in $\eps$. For a vector space $V$ over $\bbC$, let $V^{[\eps]} = V \otimes_{\bbC} \bbC((\eps))$, which is a $\bbC((\eps))$-vector space. If $G$ acts on $V$, then the action extends to the action of a group $G^{[\eps]}$ on $V^{[\eps]}$: when $G$ is algebraic, then $G^{[\eps]}$ is the group of the $\bbC((\eps))$-points of $G$. 
\begin{definition}
\label{introduction-definition-formaldegenerations}

Let $G$ be a linear algebraic group acting linearly on a vector space $V$. Let $v,w \in V$. We say that $w$ is a \emph{formal $G$-degeneration} of $v$ if there exists an element $g_\eps \in G^{[\eps]}$ such that
\[
g_\eps \cdot v = w + \eps w_1 + \cdots .
\]
\end{definition}
In contrast with \ref{introduction-definition-formaldegenerations}, the degeneration defined in \ref{introduction-definition-orbitsdegenerations} is sometimes called \emph{topological degeneration}. It turns out that the two definitions are equivalent.
\begin{theorem}
\label{introduction-theorem-degenerationsequivalence}

Let $G$ be a linear algebraic group acting linearly on a space $V$. Let $v,w \in V$. The following are equivalent:
\begin{itemize}
 \item $w$ is a (topological) $G$-degeneration of $v$;
 \item $w$ is a formal $G$-degeneration of $v$.
\end{itemize}
\end{theorem}
\begin{proof}
The proof is given in \cite[Sec.20.6]{BCS97} in the case of a product of general linear groups acting on a tensor space. In \cite[Sec.2.3]{Kra84}, the proof is given for the action of $\GL(V)$ on an arbitrary vector space. A sketch of the general proof is given in \cite[Rmk.4.4]{CGZ23}.
\end{proof}

\section{Decomposable tensors and classical algebraic varieties}
\label{introduction-section-decomposable_tensors}

\begin{definition}[Decomposable tensor]\label{introduction-definition-decomposable_tensor}
    In $V_1\ootimes V_d$, tensors that are products of vectors, e.g., $T = v_1 \ootimes v_d$, are called \emph{decomposable tensors} or \emph{rank-one tensors}.
\end{definition}
Decomposable tensors parametrize classical algebraic varieties. Since the property of being decomposable is clearly invariant under scalar multiples, these algebraic varieties are naturally defined inside \emph{projective spaces}.

We follow the usual notation $[v] \in \bbP V$ for the projective point corresponding to a vector $v \in V \smallsetminus 0$.

\begin{definition}[Segre variety]
\label{introduction-definition-Segre}

Let $V_1 \vvirg V_d$ be vector spaces. The \emph{Segre embedding} is the algebraic map defined by 
\begin{align*}
\Seg: \bbP V_1 \ttimes \bbP V_d &\to \bbP (V_1 \ootimes V_d) \\
([v_1] \vvirg [v_d]) &\mapsto [v_1 \ootimes v_d].
\end{align*}
The image of the Segre embedding is the \emph{Segre variety} of rank-one tensors:
\[
\Seg( \bbP V_1 \ttimes \bbP V_d) =\{ [v_1 \ootimes v_d] : v_j \in V_j\}.
\]
\end{definition}
When $d = 2$, the Segre variety is the variety of rank-one matrices in $\bbP (V_1 \otimes V_2)$.

\begin{definition}[Veronse variety]
\label{introduction-definition-Veronese}

Let $V$ be vector spaces. The \emph{$d$-th Veronese embedding} is the algebraic map defined by 
\begin{align*}
    \nu_d : \bbP V &\to \bbP S^d V \\
    [v] &\mapsto [v^{\otimes d}].
\end{align*}
The image of the Veronese embedding is the \emph{Veronese variety}:
\[
    \nu_d( \bbP V ) =\{ [v^d] :  v \in V\}.
\]
\end{definition}
When $\dim V = 2$, then $\bbP V = \bbP^1$ and $\nu_d( \bbP^1)$ is the \emph{rational degree-$d$ normal curve} in $\bbP S^d V = \bbP^d$.

\begin{definition}[Segre-Veronese variety]
\label{introduction-definition-SegreVeronese}

Let $V_1 \vvirg V_m$ be vector spaces. The \emph{Segre-Veronese embedding} of multidegree $\underline{d} = (d_1 \vvirg d_m)$ is the algebraic map defined by 
\begin{align*}
\nu_{\underline{d}} : \bbP V_1 \ttimes \bbP V_m &\to \bbP S^{\underline{d}}(V_1,\ldots,V_m) \\
([v_1] \vvirg [v_m]) &\mapsto [v_1^{\otimes d_1} \ootimes v_m^{\otimes d_m}].
\end{align*}
The image of the Segre-Veronese embedding is the \emph{Segre-Veronese variety}:
\[
\nu_{\underline{d}}( \bbP V_1 \ttimes \bbP V_k ) = \{ [v_1^{\otimes d_1} \ootimes v_k^{\otimes d_k}] :  v_j \in V_j\}.
\]
\end{definition}
Note that the Segre variety is a Segre-Veronese variety of multidegree $(1 \vvirg 1)$, while the Veronese variety is a Segre-Veronese variety of multidegree $(d)$.

\begin{definition}[Grassmannian]\label{introduction-definition-Grassmannian}
    Let $V$ be a vector space. The \emph{Grassmannian} $\Gr(d,V)$ is the set of $d$-dimensional subspaces of $V$: this is identified with the set of decomposable skew-symmetric tensors in $\Wedge^dV$. The Grassmannian can be embedded as an algebraic subvariety of $\bbP \Wedge^dV$ via the \emph{Pl\"ucker embedding} given by 
    \begin{align*}
    \wedge_{d} : \bbP V \ttimes \bbP V &\to \bbP \Wedge^{d}V \\
    ([v_1] \vvirg [v_d]) &\mapsto [v_1 \wwedge v_d].
    \end{align*}
\end{definition}    

%%% AleO: to be added: these are orbit closures
