In this chapter, we introduce all basic notions, terminologies and notations that will be used through all the MultilinearVerse. 

\section{Representing tensors}
\label{introduction-section-representingtensors}
% Author: Fulvio Gesmundo, Alessandro Oneto

\begin{definition}[Tensor product]
\label{introduction-definition-tensor_product}
% Author: Fulvio Gesmundo, Alessandro Oneto
Let $V,W$ be finite dimensional $\Bbbk$-vector spaces.
\begin{itemize}
    \item Let $\Hom(V,W)$ denote the vector space of linear functions from $V$ to $W$.
    \item Let $V^* = \Hom(V,\Bbbk)$ denote the \emph{dual space} of $V$.
    \item Let $V \otimes W$ denote the tensor product of $V$ and $W$: it can be equivalently identified with the space of linear maps $\Hom(V^*,W) \simeq \Hom(W^*,V)$ or with the space of bilinear map $V \times W \to \bbC$.
\end{itemize}
Given $\Bbbk$-vector spaces $V_1 \vvirg V_d$, let $V_1 \ootimes V_d$ denote their tensor product, which is identified with the space of multilinear maps $V_1^* \ttimes V_d^* \to \bbC$. If $V_1 = \cdots = V_d = V$, then we write for short $V^{\otimes d}$. 

The elements of a tensor product of $d$ vector spaces are called \emph{tensors} of \emph{order $d$}.

In coordinates, if $\{v_0,\ldots,v_m\}$ is a basis of $V$ and $\{w_0,\ldots,w_n\}$ is a basis of $W$, then $\{v_i \otimes w_j ~:~ i \in \{0,\ldots,m\}, j \in \{0,\ldots,n\}$ is a basis of $V\otimes W$. Therefore, any element $T \in V\otimes W$ can be written uniquely as 
\[
    T = \sum_{i,j} t_{i,j}v_i\otimes w_j;
\]
therefore, the choice of basis gives an identification of the space $V\otimes W$ with the space of $m\times n$ \emph{matrices} with coefficients in $\Bbbk$. In general, if $\{v_0^{(i)},\ldots,v_{n_i}^{(i)}\}$ is a basis of $V_i$, then any element $T \in V_1 \ootimes V_d$ can be written uniquely as 
\[
    T = \sum_{i_1,\ldots,i_d} t_{i_1,\ldots,i_d}v_{i_1}^{(1)}\ootimes v_{i_d}^{(d)};
\]
therefore, the choice of basis gives an identification of the space $V_1 \ootimes V_d$ with the space of $d$-dimensional arrays of size $n_1 \times\cdots\times n_d$. 
\end{definition}

\begin{definition}[Flattenings]
\label{introduction-definition-flattenings}
% Author: Fulvio Gesmundo, Alessandro Oneto
For every $K \subseteq \{1 \vvirg d\}$, let $K^c = \{1\vvirg d\} \smallsetminus K$. Then, a tensor $T \in V_1 \ootimes V_d$ defines a linear map 
\[
    \Flat_K(T): \bigotimes_{k \in K} V_{k}^* \to  \bigotimes_{k' \in K^c} V_{k'}
\]
called \emph{flattening} of $T$. It is easy to see that $\Flat_K(T) = (\Flat_{K^c}(T))^\bft$. 
\end{definition}

\begin{definition}[Concise tensors]
\label{introduction-definition-concise}
% Author: Fulvio Gesmundo, Alessandro Oneto
A tensor $T$ is \emph{concise} if the flattenings $T_k : V_k^* \to \bigotimes_{k' \neq k} V_{k'}$ are injective. 
\end{definition}

A tensor $T \in V_1 \ootimes V_d$ can be identified with the image of its flattening $T : V_1^* \to V_2 \ootimes V_d$, up to a change of coordinates on the space $V_1$. Here, we consider the action of $\GL(V_1)$ on $V_1 \ootimes V_d$: for more about group actions, see \ref{introduction-section-groupactions}. In other words, if $T,T'$ are two tensors such that the image of their first flattening is the same subspace of $V_2 \ootimes V_d$, then there exists an element $g \in \GL(V_1)$ such that $T = g\cdot T'$. Similar statements hold with respect to the other simple flattenings.

\subsection{Symmetric Tensors}
\label{introduction-subsection-symmetric_tensors}
% Author: Alessandro Oneto

On the space of tensors $V^{\otimes d}$, we consider the action of the \emph{symmetric group} $\frakS_d$ which acts by permuting the tensor factors: if $T = (t_{i_1,\ldots,i_d}) \in V^{\otimes d}$, then, for any $\sigma \in \frakS_d$, then \[(\sigma \cdot T)_{i_1,\ldots,i_d} = t_{i_{\sigma(1)},\ldots,i_{\sigma(d)}}\] for any multi-index $(i_1,\ldots,i_d)$. 
\begin{definition}[Symmetric tensors]
\label{introduction-definition-symmetric_tensors}
% Author: Alessandro Oneto
A tensor $T \in V^{\otimes d}$ which is invariant under the action of $\frakS_d$ is said to be \emph{symmetric}. Let $S^d V \subseteq V^{\otimes d}$ denote the space of symmetric tensors. 

If the base field has characteristic $0$, then $S^d V$ can be identified with the space of \emph{homogeneous polynomials} of degree $d$ on $V^*$. For example, the monomial $xy^2$ is identified with the symmetric tensor $\frac{1}{3}\left(x\otimes y \otimes y + y\otimes x \otimes y + y\otimes y \otimes x\right)$.
\end{definition}

\subsection{Partially Symmetric Tensors}
\label{introduction-subsection-partially_symmetric_tensors}
% Author: Alessandro Oneto
On the space of tensors $V_1^{\otimes d_1}\ootimes V_m^{\otimes d_m}$, we consider the action of the \emph{symmetric group} $\frakS_{d_1} \ttimes \frakS_{d_m}$ which acts by permuting the tensor factors accordingly to the partition: namely, if $T = T_1 \ootimes T_m \in V_1^{\otimes d_1}\ootimes V_m^{\otimes d_m}$, with $T_i \in V_i^{\otimes d_i}$, then, for any $\sigma = (\sigma_1,\ldots,\sigma_m)\in \frakS_{d_1} \ttimes \frakS_{d_m}$, then \[\sigma \cdot T = (\sigma_1\cdot T_1) \ootimes (\sigma_m\cdot T_m) \in V_1^{\otimes d_1}\ootimes V_m^{\otimes d_m},\] where each $\sigma_i$ acts permuting the factors of $T_i$ as explained in \ref{introduction-definition-symmetric_tensors}.

\begin{definition}[Partially symmetric tensors]
\label{introduction-definition-partially_symmetric_tensors}    
% Author: Alessandro Oneto
A tensor $T \in V_1^{\otimes d_1}\ootimes V_m^{\otimes d_m}$ which is invariant under the action of $\frakS_{d_1} \ttimes \frakS_{d_m}$ is said to be \emph{partially symmetric}. Let $S^{d_1,\ldots,d_m}(V_1,\ldots,V_m) = S^{d_1} V_1 \ootimes S^{d_m}V_m \subseteq V_1^{\otimes d_1}\ootimes V_m^{\otimes d_m}$ denote the space of partially symmetric tensors. 

If the base field has characteristic $0$, then $S^{d_1} V_1 \ootimes S^{d_m}V_m \subseteq V_1^{\otimes d_1}\ootimes V_m^{\otimes d_m}$ can be identified with the space of \emph{multi-homogeneous polynomials} of multi-degree $(d_1,\ldots,d_m)$ on $V^*_1 \ttimes V^*_m$. 
\end{definition}

Inside the space of tensors $V^{\otimes d}$, we have all possible spaces of partially symmetric tensors $S^{\underline{d}}(V) = S^{d_1} V \ootimes S^{d_m}V$ for all partitions $\underline{d} = (d_1,\ldots,d_m)\vdash d$. In particular, $S^{\underline{d}}(V) \subset S^{\underline{d'}}(V)$ if $\underline{d}'$ is a refinement of $\underline{d}$.

Fixed a basis $\{w_0,\ldots,w_m\}$ of $W$, any tensor $T \in W \otimes S^{d-1} V$ can be written as $T = \sum_{i=1}^m w_i \otimes T_i$ and then identified with the $m$-tuple $\{T_1,\ldots,T_m\}$ of elements of $S^{d-1} V$. This is also a set of generators for the image of the first flattening $\Flat_1(T)$, see \ref{introduction-definition-flattenings}.

For example, if $F \in S^d V$ is a symmetric tensor, with $\{v_0 \vvirg v_n\}$ basis of $V$, then $F$ is regarded as the element in $S^{(1,d-1)}(V)$ given by 
\[
    \sum_{j=0}^n v_j \otimes \textstyle\frac{\partial }{\partial v_j} F
\]
where the corresponding $(n+1)$-tuple of elements of $S^{d-1} V$ can be identified with the gradient of the homogeneous polynomial $F$.

\subsection{Skew-symmetric Tensors}
\label{introduction-subsection-skew_symmetric_tensors}
% Author: Alessandro Oneto
On the space of tensors $V^{\otimes d}$, we consider the skew-symmetric action of the symmetric group $\frakS_d$: if $T = (t_{i_1,\ldots,i_d}) \in V^{\otimes d}$, then, for any $\sigma \in \frakS_d$, then $(\sigma \cdot T)_{i_1,\ldots,i_d} = \sgn(\sigma)t_{i_{\sigma(1)},\ldots,i_{\sigma(d)}}$ for any multi-index $(i_1,\ldots,i_d)$.
\begin{definition}[Skew symmetric tensors]
\label{introduction-definition-skew_symmetric_tensors}
% Author: Alessandro Oneto
A tensor $T \in V^{\otimes d}$ which is invariant under the skew-symmetric action of $\frakS_d$ is said to be \emph{skew symmetric}. Let $\Lambda^d V \subseteq V^{\otimes d}$ denote the space of skew-symmetric tensors. 
\end{definition}

\section{Representation theory}
\label{introduction-section-repTheory}
% Author: Fulvio Gesmundo

We introduce some notation and basic notions from representation theory. We assume some background, to the level of \cite[Ch.6]{Lan12}; for a detailed introduction, we refer to \cite{FH91}. We are primarily interested in the representation theory of the general linear group and of the symmetric group, over the field of complex numbers.

Let $\GL(V)$ denote the general linear group of a vector space $V$ and $\SL(V)$ denote the special linear group. Let $\frakS_d$ denote the symmetric group on $d$ elements. 

\begin{definition}
\label{introduction-definition-grouprepresentation} 
% Author: Fulvio Gesmundo
A {\it representation} of a group $G$ is a group homomorphism $\rho : G \to \GL(V)$ for some vector space $V$. We use the word \emph{representation} to refer to the space $V$ itself. For $g \in G$ and $v \in V$, we often write $g \cdot v$ of $gv$ for the image of $v$ via the linear map $\rho(g)$. A representation $V$ of $G$ is also called a {\it $G$-module} of a $G$-representation.

A {\it subrepresentation} of a $G$-representation $V$ is a linear subspace $V' \subseteq V$ which is closed under the action of $G$. A $G$-representation is called {\it irreducible} if it has no $G$-subrepresentations, except for the trivial ones, $0$ and $V$ itself.
\end{definition}

An important fact in representation theory is that the groups $\GL(V)$, $\SL(V)$, every finite group, other classical groups such as the orthogonal and the symplectic group, are {\it reductive}. This property is characterized by the fact that every $G$-representation splits into direct sum of irreducible representations. We refer to \cite[Ch.9]{FH91} for a discussion. 

\begin{definition}
 \label{introduction-definition-equivariantmap}
% Author: Fulvio Gesmundo
Given $V,W$ representations for a group $G$, and a linear map $\phi : V \to W$, we say that $\phi$ is {\it $G$-equivariant} if it commutes with the action of $G$, that is $\phi( g \cdot v) = g \cdot \phi(v)$ for every $v \in V$, $g \in G$. The subset of $\Hom(V,W)$ consisting of $G$-equivariant maps is a linear subspace, and it is denoted by $\Hom_G(V,W)$. 
\end{definition}

A fundamental result which will prove useful numerous times is Schur's Lemma. 
\begin{lemma}[Schur's Lemma]
\label{introduction-lemma-Schur}
% Author: Fulvio Gesmundo
Let $G$ be a group, and let $\phi : V \to W$ be an equivariant map between two $G$-representations. Then $\ker(\phi)$ and $\image(\phi)$ are $G$-representations. In particular, if $V$ is irreducible, then $\phi = 0$ or $\phi$ is injective. If $V= W$ then $\phi = \lambda \Id_V$ for some $\lambda \in \bbC$, that is $\dim \Hom _G(V,W) = 1$.
\end{lemma}

If $H$ is a subgroup of a group $G$, $V$ is a representation of $H$ and $W$ is a representation of $G$, write 
\[
\Ind_H^G ( V) \qquad \Res_H^G (W)
\]
for the $G$-representation \emph{induced} by $V$ and the \emph{restriction} of $W$ to $H$. We refer to \cite[Sec. 4.4.1]{GW09} for details on these constructions. Frobenius reciprocity links these two constructions, see \cite[Thm. 4.4.1]{GW09}:
\begin{lemma}
 \label{introduction-lemma-Frobeniusreciprocity}
 Let $G$ be a group with a subgroup $H$. Let $V$ be a representation of $H$ and let $W$ be a representation of $G$. There is a canonical isomorphism
 \[
\Hom_G(W,\Ind_H^G ( V)) \simeq \Hom_H (\Res_H^G (W) , V).
 \]
\end{lemma}
If $V$ is a representation of a group $G$, write $V^G$ for the subspace of $G$-invariants. 

\subsection{Irreducible representations of the general linear group and of the symmetric group}
\label{introduction-subsection-irrepsGLandSn}
% Author: Fulvio Gesmundo

We are particularly interested in representations of the general linear group and of the symmetric group. These are described in terms of combinatorics of \emph{partitions}. A partition $\lambda = (\lambda_1 \vvirg \lambda_n)$ is a non-increasing sequence of positive integers. We say that $\lambda$ is a partition of $d$ if $\sum_i \lambda_i = d$, and we write $|\lambda| = d$; we say that $n$ is the length of $\lambda$ and we write $n = \ell(\lambda)$. Write $\lambda \partinto_n d$ to mean that $\lambda$ is a partition of $d$ of length at most $n$. Repeated numbers in partitions are usually expressed as exponents: for instance $\lambda= (3,3,1,1,1)$ can be written as $\lambda = (3^2,1^3)$. Partitions are represented by Young diagrams, which are top-left justified diagrams of boxes, having $\lambda_j$ boxes in their $j$-th row. Given a partition $\lambda$, denote by $\lambda^{\bft}$ the \emph{transpose} (or conjugate) partition of $\lambda$, obtained by transposing its Young diagram; for instance if $\lambda = (3^2,1^3)$ then $\lambda^\bft = (4,2,2)$. 

The (polynomial) irreducible representations of the general linear group of a vector space of dimension $n$ are indexed by partitions of length at most $n$. Let $\bbS_\lambda V$ be the irreducible representation associated to the partition $n$; this is the Schur module of $\GL(V)$ associated to $\lambda$. If $\lambda = (d)$ consists of a single row then $\bbS_\lambda V = S^d V$ is the representation of symmetric tensors of order $d$; if $\lambda = (1^d)$ consists of a single column then $\bbS_\lambda V = \Lambda^d V$ is the representation of skew-symmetric tensors. Schur modules can be constructed explicitly via \emph{highest weight theory}: we refer to \cite{FH91} for this construction in the more general setting of reductive algebraic groups.

The irreducible representations of the symmetric group $\frakS_d$ are indexed by partitions of $d$. Let $[\lambda]$ be the irreducible representation associated to the partition $\lambda$; this is the Specht module of $\frakS_d$ of type $\lambda$. 

The vector space $V^{\otimes d}$ is acted on naturally by $\frakS_d$, which permutes the tensor factors, and $\GL(V)$ which acts diagonally on all tensor factors; these two actions commute and the Schur-Weyl decomposition theorem expresses the spaces as a direct sum of irreducible representations for $\frakS_d \times \GL(V)$. Such decomposition is as follows:
\[
V^{\otimes d} = \bigoplus_{\lambda \partinto_n d} [\lambda] \otimes \bbS_{\lambda} V.
\]
The projections $V^{\otimes d} \to [\lambda] \otimes \bbS_\lambda V$ are canonical and they can be described combinatorially via Young symmetrizers, see \cite[Lecture 4]{FH91}.

By \ref{introduction-lemma-Schur}, the Schur-Weyl decomposition shows $\dim \Hom_{\GL(V)} ( \bbS_\lambda V, V^{\otimes d}) = \dim [\lambda]$. More generally, a fundamental problem in representation theory and algebraic combinatorics consists in determining the decomposition of the tensor product of two (or more) irreducible representations. It is convenient to give these definitions in terms of representations of the symmetric group, so that they do not depend on the dimension of the vector space $V$.

\begin{definition}[Littlewood-Richardson coefficients]
 \label{introduction-definition-LRcoefficient}
% Author: Fulvio Gesmundo
 Let $\lambda,\mu,\nu$ be three partitions of three integers $k,m, k+m$ respectively. The {\it Littlewood-Richardson coefficient} associated to $(\lambda,\mu;\nu)$ is 
 \[
c^\nu_{\lambda,\mu} = 
\dim \Hom_{\frakS_{k+m}} ([\nu], \Ind_{\frakS_k \times \frakS_m}^{\frakS_{k+m}} ([\lambda] \otimes [\mu]) = 
\dim \Hom_{\frakS_{k} \times \frakS_m} ([\lambda] \otimes [\mu] , \Res_{\frakS_k \times \frakS_m}^{\frakS_{k+m}} [\nu]).
\]
\end{definition}
Note that the two dimensions in \ref{introduction-definition-LRcoefficient} are equal by \ref{introduction-lemma-Frobeniusreciprocity}. From the definition, it is clear that $c^\nu_{\lambda,\mu} = c^\nu_{\mu,\lambda}$. 

We have the following result:
\begin{lemma}
 \label{introduction-lemma-LRcoefficientGL}
% Author: Fulvio Gesmundo
  Let $\lambda,\mu,\nu$ be three partitions. If $V$ is a vector space with $\dim V \geq \ell(\nu)$, then 
 \[
c^\nu_{\lambda,\mu} = \dim \Hom_{\GL(V)} (\bbS_\nu V,  \bbS_\lambda V \otimes \bbS_\mu V ).
\]
\end{lemma}
Via \ref{introduction-lemma-Schur}, the Littlewood-Richardson coefficient $c^\nu_{\lambda,\mu} $ coincides with the multiplicity of $\bbS_\nu V$ as a subrepresentation of $ \bbS_\lambda V \otimes \bbS_\mu V $, that is 
\[
 \bbS_\lambda V \otimes \bbS_\mu V  = \bigoplus_{\nu } (\bbS_{\nu} V )^{\oplus c^\nu_{\lambda,\mu} }.
\]

Littlewood-Richardson coefficients are hard to compute \cite{Nar06} even though there are polynomial-time algorithms to decide whether they are zero or nonzero as observed in \cite{MNS12} following \cite{KT99}. However, there are several special cases for which they are easy to determine. For instance, when $\mu$ is a partition consisting of a single row of a single column, we have the following classical result.
\begin{proposition}[Pieri's rule]
\label{introduction-proposition-Pieri}
% Author: Fulvio Gesmundo
 Let $\lambda$ be a partition and let $d \geq 0$. Then 
 \[
 \bbS_{\lambda} V \otimes S^d V = \bigoplus_{\nu \in R} \bbS_{\nu} V, \qquad  \bbS_{\lambda} V \otimes \Lambda^d V = \bigoplus_{\nu \in C} \bbS_{\nu} V.
 \]
Here $R$ (resp. $C$) is the set of partitions of $|\lambda|+d$ obtained from $\lambda$ by adding $d$ boxes, no two of them on the same column (resp. row).
\end{proposition}
There are variants of \ref{introduction-proposition-Pieri} for other groups, such as the orthogonal group or the symplectic group.

We introduce other two structure coefficients which appear in the literature on algebraic combinatorics and play an important role in the study of equations for varieties of tensors. For integers $d,e$, denote by $\frakS_d \wr \frakS_e$ the wreath product of $\frakS_e$ with $\frakS_d$, that is the semidirect product
\[
\frakS_d \wr \frakS_e = (\frakS_d ^{\times e}) \rtimes \frakS_e
\]
where $\frakS_e$ acts by permuting the $e$ copies of $\frakS_d$. The group $\frakS_d \wr \frakS_e $ is naturally a subgroup of $\frakS_{de}$, acting via permutation on pairs $(i,j) \in [d] \times [e]$.
\begin{definition}
 \label{introduction-definition-plethysm}
% Author: Fulvio Gesmundo
 Let $d,e$ be integers and let $\lambda$ be partitions of $de$. The {\it plethysm coefficient} associated to $(\lambda;d,e)$ is
 \[
a_\lambda(e,d) = \dim [\lambda]^{\frakS_d \wr \frakS_e}.
\]
\end{definition}
We have the following result:
\begin{lemma}
 \label{introduction-lemma-plethysmGL}
% Author: Fulvio Gesmundo
 Let $d,e$ be integers and let $\lambda$ be a partition of $de$. Let $V$ be a vector space with $\dim V \geq \ell(\lambda)$. Then
 \[
 a_\lambda(e,d) = \dim \Hom_{\GL(V)} ( \bbS_\lambda V , S^e S^d V).
 \]
\end{lemma}
Sometimes the word \emph{plethysm} is used more generally for the \emph{composition} of two representations, and the expression \emph{plethysm coefficient} is used for the corresponding multiplicities.


\begin{definition}
 \label{introduction-definition-kroncoefficient}
% Author: Fulvio Gesmundo
 Let $\lambda,\mu,\nu$ be partitions of an integer $d$. The {\it Kronecker coefficient} associated to $(\lambda,\mu,\nu)$ is 
 \[
 g_{\lambda,\mu,\nu} = \dim ([\lambda] \otimes [\mu] \otimes [\nu])^{\frakS_d}
 \]
\end{definition}
We have the following result:
\begin{lemma}
 \label{introduction-lemma-kroncoefficientGL}
 Let $\lambda,\mu,\nu$ be partitions of an integer $d$. Let $V,W$ be vector spaces with $\dim V \geq \ell(\lambda)$, $\dim W \geq \ell(\mu)$. Then 
 \[
 g_{\lambda,\mu,\nu} = \dim \Hom_{\GL(V)} ( \bbS_\lambda V \otimes \bbS_\mu W , \bbS_\nu(V \otimes W)).
 \]
\end{lemma}


\section{Group actions, restrictions and degenerations}
\label{introduction-section-groupactions}
% Author: Fulvio Gesmundo

\begin{definition}
\label{introduction-definition-orbitsdegenerations}
% Author: Fulvio Gesmundo
Let $G$ be a group acting on a space $V$ and let $v \in V$. The $G$-orbit of $v$ is 
\[
\Omega_v = \{ g(v) : g \in G \}.
\]
The $G$-orbit-closure $\bOmega_v$ is the closure of $\Omega_v$. We say that $w$ is $G$-isomorphic to $v$ if $w \in \Omega_v$. We say that $w$ is a degeneration of $v$ if $w \in \bOmega_v$.

Typically $V$ is a vector space, but the definition applies more generally with $V$ any topological space.
\end{definition}

If $G$ is a complex algebraic group acting algebraically on an algebraic variety $V$ (typically an affine space) then the closure defining $\bOmega_v$ can be taken equivalently in the Zariski or the Euclidean topology, see, e.g., \cite[Thm. 2.33]{Mum76}.

One can give an apparently different definition of degeneration, in terms of formal power series. Let $\bbC[[\eps]]$ be the ring of formal power series in one variable $\eps$, and let $\bbC((\eps))$ denote its quotient field, that is the field of Laurent series in $\eps$. For a vector space $V$ over $\bbC$, let $V^{[\eps]} = V \otimes_{\bbC} \bbC((\eps))$, which is a $\bbC((\eps))$-vector space. If $G$ acts on $V$, then the action extends to the action of a group $G^{[\eps]}$ on $V^{[\eps]}$: when $G$ is algebraic, then $G^{[\eps]}$ is the group of the $\bbC((\eps))$-points of $G$. 
\begin{definition}
\label{introduction-definition-formaldegenerations}
% Author: Fulvio Gesmundo
Let $G$ be a linear algebraic group acting linearly on a vector space $V$. Let $v,w \in V$. We say that $w$ is a \emph{formal $G$-degeneration} of $v$ if there exists an element $g_\eps \in G^{[\eps]}$ such that
\[
g_\eps \cdot v = w + \eps w_1 + \cdots .
\]
\end{definition}
In contrast with \ref{introduction-definition-formaldegenerations}, the degeneration defined in \ref{introduction-definition-orbitsdegenerations} is sometimes called \emph{topological degeneration}. It turns out that the two definitions are equivalent.
\begin{theorem}
\label{introduction-theorem-degenerationsequivalence}
% Author: Fulvio Gesmundo
Let $G$ be a linear algebraic group acting linearly on a space $V$. Let $v,w \in V$. The following are equivalent:
\begin{itemize}
 \item $w$ is a (topological) $G$-degeneration of $v$;
 \item $w$ is a formal $G$-degeneration of $v$.
\end{itemize}
\end{theorem}
\begin{proof}
The proof is given in \cite[Sec.20.6]{BCS97} in the case of a product of general linear groups acting on a tensor space. In \cite[Sec.2.3]{Kra84}, the proof is given for the action of $\GL(V)$ on an arbitrary vector space. A sketch of the general proof is given in \cite[Rmk.4.4]{CGZ23}.
\end{proof}

\section{Decomposable tensors and classical algebraic varieties}
\label{introduction-section-decomposable_tensors}
% Author: Alessandro Oneto
\begin{definition}[Decomposable tensor]\label{introduction-definition-decomposable_tensor}
    In $V_1\ootimes V_d$, tensors that are products of vectors, e.g., $T = v_1 \ootimes v_d$, are called \emph{decomposable tensors} or \emph{rank-one tensors}.
\end{definition}
Decomposable tensors parametrize classical algebraic varieties. Since the property of being decomposable is clearly invariant under scalar multiples, these algebraic varieties are naturally defined inside \emph{projective spaces}.

We follow the usual notation $[v] \in \bbP V$ for the projective point corresponding to a vector $v \in V \smallsetminus 0$.

\begin{definition}[Segre variety]
\label{introduction-definition-Segre}
% Author: Alessandro Oneto
Let $V_1 \vvirg V_d$ be vector spaces. The \emph{Segre embedding} is the algebraic map defined by 
\begin{align*}
\Seg: \bbP V_1 \ttimes \bbP V_d &\to \bbP (V_1 \ootimes V_d) \\
([v_1] \vvirg [v_d]) &\mapsto [v_1 \ootimes v_d].
\end{align*}
The image of the Segre embedding is the \emph{Segre variety} of rank-one tensors:
\[
\Seg( \bbP V_1 \ttimes \bbP V_d) =\{ [v_1 \ootimes v_d] : v_j \in V_j\}.
\]
\end{definition}
When $d = 2$, the Segre variety is the variety of rank-one matrices in $\bbP (V_1 \otimes V_2)$.

\begin{definition}[Veronese variety]
\label{introduction-definition-Veronese}
% Author: Alessandro Oneto
Let $V$ be vector spaces. The \emph{$d$-th Veronese embedding} is the algebraic map defined by 
\begin{align*}
    \nu_d : \bbP V &\to \bbP S^d V \\
    [v] &\mapsto [v^{\otimes d}].
\end{align*}
The image of the Veronese embedding is the \emph{Veronese variety}:
\[
    \nu_d( \bbP V ) =\{ [v^{\otimes d}] :  v \in V\}.
\]
Recall that the symmetric tensors can be identified with homogeneous polynomials. Hence, the Veronese variety corresponds to the variety parametrized by $d$-th powers of linear forms in the projective space of degree-$d$ polynomials. 

When $\dim V = 2$, then $\bbP V = \bbP^1$ and $\nu_d( \bbP^1)$ is the \emph{rational degree-$d$ normal curve} in $\bbP S^d V = \bbP^d$.
\end{definition}

\begin{definition}[Segre-Veronese variety]
\label{introduction-definition-SegreVeronese}
% Author: Alessandro Oneto
Let $V_1 \vvirg V_m$ be vector spaces. The \emph{Segre-Veronese embedding} of multidegree $\underline{d} = (d_1 \vvirg d_m)$ is the algebraic map defined by 
\begin{align*}
\nu_{\underline{d}} : \bbP V_1 \ttimes \bbP V_m &\to \bbP S^{\underline{d}}(V_1,\ldots,V_m) \\
([v_1] \vvirg [v_m]) &\mapsto [v_1^{\otimes d_1} \ootimes v_m^{\otimes d_m}].
\end{align*}
The image of the Segre-Veronese embedding is the \emph{Segre-Veronese variety}:
\[
\nu_{\underline{d}}( \bbP V_1 \ttimes \bbP V_k ) = \{ [v_1^{\otimes d_1} \ootimes v_k^{\otimes d_k}] :  v_j \in V_j\}.
\]
\end{definition}
Note that the Segre variety is a Segre-Veronese variety of multidegree $(1 \vvirg 1)$, while the Veronese variety is a Segre-Veronese variety of multidegree $(d)$.

\begin{definition}[Grassmannian]
\label{introduction-definition-Grassmannian}
% Author: Alessandro Oneto
    Let $V$ be a vector space. The \emph{Grassmannian} $\Gr(d,V)$ is the set of $d$-dimensional subspaces of $V$. The Grassmannian is a projective variety and it has a natural embedding in $\bbP \Lambda^d V$, called the \emph{Pl\"ucker embedding}. It is given by
    \begin{align*}
    \wedge_{d} : \Gr(d,V) &\to \bbP \Lambda^{d}V \\
    \langle v_1 \vvirg v_d \rangle &\mapsto [v_1 \wwedge v_d].
    \end{align*}
\end{definition}    

%%% Important to add that skew tensors of rank 1 are not tensors of rank one, unlike the other cases.
%%% AleO: to be added: these are orbit closures
%%% FG: Sono proprio orbite!
