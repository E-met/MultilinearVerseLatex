 %%%%%%%%%%%%%%%%%%%%%%%%%%%%%%%%%%%%
 In this chapter we introduce the basic notions of secant varieties and ranks. 
 
 \section{Secant varieties}
 \label{geometrySecants-section-secants}
 % Author: Alessandro Oneto
 
 \begin{definition}[Join of varieties]
 \label{geometrySecants-definition-join}
 % Author: Alessandro Oneto
 Let $X,Y \subseteq \mathbb{P}^N$ be two algebraic varieties. Consider the incidence correspondence 
 \[
     \itJoin^\circ(X,Y) = \{(x,y,p) ~:~ x \neq y,~p \in \langle x,y \rangle\} \subset X \times Y \times \bbP^N.
 \]
 This is a quasi-projective variety and its Zariski closure is the {\it abstract join} $\itJoin(X,Y) = \overline{\itJoin^\circ(X,Y)}$ of $X$ and $Y$. Consider the projections 
 \[
     \xymatrix{
         X \times Y \times \bbP^N \ar[d]^{\pi_1} \ar[r]^(0.65){\pi_2} & \bbP^N\\
         X \times Y & .
     }
 \]
 The \emph{join} of $X$ and $Y$ is the scheme-theoretic image 
 \[
     j(X,Y) = \pi_2(\itJoin(X,Y))
 \]
 This can also be regarded as the the closure of the union of all possible lines joining a point of $X$ and a point of $Y$, i.e.,
 \[
     j(X,Y) = \overline{ \bigcup_{\substack{x \in X, y \in Y \\ x \neq y}} \langle x,y \rangle} \subset \bbP^N.
     % \{ p \in \mathbb{P}^N : p \in \langle x , y \rangle \text{ for some $x \in X, y \in Y$}\}}
 \]
 The projection $\pi_1$ realizes, locally, $\itJoin(X,Y)$ as a $\bbP^1$-bundle over (an open subset) of $X \times Y$. In particular, if $X$ and $Y$ are irreducible, than $Join(X,Y)$ is irreducible and, consequently, $j(X,Y)$ is irreducible as well.
 
 Given varieties $X_1 \vvirg X_s \subseteq \bbP^N$, one can define their join recursively
 \[
     j( X_1 \vvirg X_s) = j( j(X_1 \vvirg X_{s-1}), X_s);
 \]
 geometrically this is 
 \[
     j( X_1 \vvirg X_s) = \overline{\bigcup_{\substack{x_1 \in X_1,\ldots,x_s \in X_s \\ \{x_1,\ldots,x_s\} \text{ indipendent}}} \langle x_1,\ldots,x_s \rangle} \subset \bbP^N.
 \]
 \end{definition}
 
 Secant varieties are obtained by consecutive joins of a variety with itself.
 \begin{definition}[Secant variety]
 \label{geometrySecants-definition-secantvariety}
 % Author: Alessandro Oneto
 Let $X \subseteq \mathbb{P}^N$ be an projective variety. The {\it $s$-th secant variety} of $X$ is 
 \[
 \sigma_s(X) = \overline{\bigcup_{\substack{x_1,\ldots,x_s \in X \\ \{x_1,\ldots,x_s\} \text{ indipendent}}} \langle x_1,\ldots,x_s \rangle} \subset \bbP^N.
 \]
 Recursively, $\sigma_1(X) = X$ and $\sigma_s(X) = j(X,\sigma_{s-1}(X))$.
 
 The abstract $s$-th secant variety of $X$ can be defined as the abstract join of $s$-copies of $X$ as defined above. However, if is often convenient to consider a symmetrized version. Let $X^{\cdot s} = X^{\times s} / \frakS_s$ be the symmetrized product of $s$ copies of $X$. Then, the {\it abstract $s$-th secant variety} of $X$ is 
 \[
     \Sec_s(X) = \overline{\left\{(\{x_1,\ldots,x_s\},p) ~:~ \substack{\{x_1,\ldots,x_s\} \text{ independent} \\ p \in \langle x_1,\ldots,x_s \rangle}\right\}} \subset X^s \times \bbP^N.
 \]
 Then, $\sigma_s(X) = \pi_2(\Sec_s(X))$, where $\pi_2 : X^s \times \bbP^N \rightarrow \bbP^N$ is the projection on the last factor.

 As explained in \label{geometrySecants-definition-join} for join varieties, if $X$ is irreducible then all its secant varieties are irreducible.  
 \end{definition}
 
 %%%%%%%%%%%%%%%%%%%%%%%%%%%%%%%%%%%%
 \section{Rank and Border Rank}
 \label{geometrySecants-section-rank}
 % Author: Alessandro Oneto
 
 One of the main motivations to study secant varieties in relation to tensors is because they allow for a geometric definition of {\it rank} with respect to any projective variety and, in particular, to the ones defined in \ref{introduction-section-decomposable_tensors}.
 
 \begin{definition}
 \label{geometrySecants-definition-Xrank}
 % Author: Alessandro Oneto
     Let $X \subseteq \bbP^N$ be a projective variety and let $p \in \bbP^N$. The \emph{$X$-rank} of $p$ is the smallest number of points of $X$ whose linear span contains $p$. In other words,
     \[
         \rank_X(p) = \min\{s ~:~ \exists x_1,\ldots,x_s \in X, ~ p \in \langle x_1,\ldots,x_s \rangle\}.
     \]
 \end{definition}
 
 It is natural to ask if the $X$-rank is \emph{finite} for any point of the ambient space. The first observation is that $X$ needs to be \emph{non-degenerate}, i.e., not contained in any proper linear subspace: this directly follows from the fact that linear spaces are invariant under taking secant varieties, see \ref{geometrySecants-lemma-palatini_2} and \ref{geometrySecants-lemma-secants_of_non_degenerate}. Under non-degeneracy assumption we can well-define the \emph{generic rank}.
 
 \begin{definition}
     \label{geometrySecants-definition-generic_rank}
     Let $X \subset \bbP^N$ be a non-degenerate algebraic variety. The \emph{generic $X$-rank} is the rank of a generic point of $\bbP^N$, i.e., the rank that occur on a Zariski-dense subset of the ambient space. Equivalently, the generic $X$-rank corresponds to the first secant variety of $X$ filling the ambient space, i.e., 
     \[
         \rank_X^\circ = \min\{s ~:~ \sigma_s(X) = \bbP^N\}.
     \]
     We denote by $\rank_X^{\max}$ the \emph{maximal $X$-rank} among all points $p \in \bbP^N$. 
 \end{definition}
 Now, the fact that the maximal $X$-rank with respect to a non-degenerate variety is finite follows from the following fact. 
 
 \begin{theorem}[\cite{BT15}]
     Let $X \subset \bbP^N$ be a non-degenerate variety. Then, $\rank_X^{\max} \leq 2\rank_X^\circ$. Moreover, in characteristic zero, if $\sigma_{\rank^\circ_X-1}(X)$ is an hypersurface, then $\rank_X^{\max} \leq 2\rank_X^\circ-1$.
 \end{theorem}
 
 Clearly, if a point $p \in \bbP^N$ has $X$-rank equal to $r$, then $p \in \sigma_r(X)$. However, the converse is not true in general. The following example is classical. Actually, for most of varieties of tensors defined in \ref{introduction-section-decomposable_tensors} the notion of rank is not lower semicontinuous.
 
 \begin{example}[$X$-rank is not always lower semicontinuous]
 \label{geometrySecants-example-Xrank_semicontinuous}
 % Author: Alessandro Oneto
   Let $X = \nu_3(\bbP^1) \subset \bbP^3$ be the rational normal cubic. If we identify the ambient space with the projective space of degree-$3$ binary homogeneous polynomials $\bbP(S^3\Bbbk^2)$, then $X$ is the variety of cubes of binary linear forms, see \ref{introduction-definition-symmetric_tensors}. Consider $xy^2 \in S^3\Bbbk^2$. Then, 
   \[
      xy^2 = \lim_{\epsilon \to 0} \frac{1}{3\epsilon}\left( x^3 - (x-\epsilon y)^3\right),
 \]
  in particular, $xy^2 \in \sigma_2(X)$ since it is the limit of points lying on secant lines. However, it is an easy exercise to check that $\rank_3(xy^2) = 3$. 
 \end{example}
 
 The existence of examples as the one of \ref{geometrySecants-example-Xrank_semicontinuous} motivate the following notion of rank.

 \begin{definition}[Border rank]
 % Author: Alessandro Oneto
     \label{geometrySecants-definition-border_rank}
     Let $X \subseteq \bbP^N$ be a projective variety and let $p \in \bbP^N$. The \emph{border $X$-rank} of $p$ is the smallest $s$ such that $p$ belongs to the $s$-th secant variety of $X$, that is 
     \[
         \brank_X(p) = \min\{s ~:~ p \in \sigma_s(X)\}.
     \]
 \end{definition}
 
 %% add historical remark on definition of border rank
 %% add reference to Bini example on matrix multiplication 
 
 \subsection{Ranks of tensors}
 If $X$ is one of the varieties of decomposable tensors defined in \ref{introduction-section-decomposable_tensors}, we obtain the following notions of ranks for tensors.
 
 \begin{definition}[Tensor rank]
 \label{geometrySecants-definition-tensor_rank}
 % Author: Alessandro Oneto
     The \emph{tensor rank} is the rank of a tensor $T \in V_1 \ootimes V_d$ with respect to the Segre variety $\Seg(V_1,\ldots,V_d)$ (see \ref{introduction-definition-Segre}); i.e., 
     \[
         \rank(T) = \min\left\{s ~:~ T = \sum_{i=1}^s v_{i,1}\ootimes v_{i,d}, ~ v_{i,j} \in V_j\right\}.
     \]
     \begin{example}
         In the case of matrices, that is when $d=2$, the notion of tensor rank coincides with the usual notion of rank of matrices. However, the case of matrices is very much different than the case of higher-order tensors ($d\geq 3$): for example, the tensor rank is not lower-semicontinuous, as seen in \ref{geometrySecants-example-Xrank_semicontinuous}, while the semicontinuity of the matrix rank is well-known. 
     \end{example}
 \end{definition}
 
 \begin{definition}[Symmetric tensor rank]
 \label{geometrySecants-definition-symmetric_tensor_rank}
 % Author: Alessandro Oneto
     The \emph{symmetric tensor rank} is the rank of a tensor $T \in S^d(V)$ with respect to the Veronese variety $\nu_d(\bbP V)$ (see \ref{introduction-definition-Veronese}); i.e., 
     \[
         \rank_d(T) = \min\left\{s ~:~ T = \sum_{i=1}^s v_{i}^{\otimes d}, ~ v_{i} \in V\right\}.
     \]
     By interpreting symmetric tensors as homogeneous polynomials, as explained in \ref{introduction-subsection-symmetric_tensors}, the symmetric tensor rank is often called \emph{Waring rank}: this is the smallest possible length of a decompositions of a homogeneous polynomial as sum of powers of linear forms.
 \end{definition}  
 
 \begin{definition}[Partially-symmetric tensor rank]
 \label{geometrySecants-definition-partially_symmetric_tensor_rank}
 % Author: Alessandro Oneto
     Let $\underline{d} = (d_1,\ldots,d_m)$. The \emph{partially-symmetric tensor rank} is the rank of $T \in S^{\underline{d}}(V_1,\ldots,V_m)$ with respect to the Segre-Veronese variety $\nu_{\underline{d}}(V_1,\ldots,V_m)$ (see \ref{introduction-definition-SegreVeronese}); i.e., 
     \[
         \rank_{\underline{d}}(T) = \min\left\{s ~:~ T = \sum_{i=1}^s v_{i,1}^{d_1}\ootimes v_{i,m}^{d_m}, ~ v_{i,j} \in V_j\right\}.
     \]
 \end{definition}
 
 \begin{definition}[Skew-symmetric tensor rank]
 \label{geometrySecants-definition-skewsymmetric_tensor_rank}
 % Author: Alessandro Oneto
     The \emph{skew-symmetric tensor rank} is the rank of $T \in \Lambda^dV$ with respect to the Grassmannian $\Gr_d(V)$ (see \ref{introduction-definition-Grassmannian}) in its Pl\"ucker embedding; i.e., 
     \[
         \rank_{\wedge}(T) = \min\left\{s ~:~ T = \sum_{i=1}^s v_{i,1}\wwedge v_{i,d}, ~ v_{i,j} \in V_j\right\}.
     \]
 \end{definition}
 
 
 