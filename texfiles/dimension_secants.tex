%%%%%%%%%%%%%%%%%%%%%%%%%%%%%%%%%%%%
\section{Expected Dimension}
\label{geometrySecants-section-expectedDimension}
% Author: Alessandro Oneto
A first question we might ask on an algebraic variety is its dimension. 

\begin{lemma}
\label{geometrySecants-lemma-expecteddimension}
% Author: Alessandro Oneto
Let $X_1\vvirg X_s \subseteq \bbP^N$ be algebraic varieties. Then 
\[
    \dim j(X_1\vvirg X_s) \leq \min\{ N, \smallsum_{i=1}^s \dim(X_i) + s-1\}.
\]
In particular, for any $X \subseteq \bbP^N$ algebraic variety, then 
\[
    \dim \sigma_s(X) \leq \min\{ N , s\dim(X) + s - 1\}.
\]
\end{lemma}
\begin{proof}
As they can be obtained as projection, the join $j(X_1,\ldots,X_s)$ and the secant variety $\sigma_s(X)$ have dimension smaller than their abstract analogous.

The dimension of the abstract join $\itJoin(X_1,\ldots,X_s)$, as defined in \ref{geometrySecants-section-secants}, easily follows from the fact that locally it can be interpreted as a $\bbP^{s-1}$-bundle over $X_1 \ttimes X_s$: therefore, 
    \[
        \dim \itJoin(X_1,\ldots,X_s) = \sum_{i=1}^s \dim(X_i) + s-1.
    \]
    In particular, $\dim \Sec_s(X) = s\dim(X) + s-1$.
\end{proof}
One may {\it expect} that the upper bound of \ref{geometrySecants-lemma-expecteddimension} is the actual dimension. This leads the the following definition. 
\begin{definition}
\label{geometrySecants-definition-expecteddimension}
% Author: Alessandro Oneto
    Let $X_1\vvirg X_s \subseteq \bbP^N$ be algebraic varieties. The {\it virtual dimension} of $j(X_1,\ldots,X_s)$ is the dimension of the abstract join. Its {\it expected dimension} is the minimum between the virtual dimension and the dimension of the ambient space. In other words,
    \begin{align*}
        \virdim j(X_1,\ldots,X_s) & = \sum_{i=1}^s \dim(X_i) + s-1; \\ 
        \expdim j(X_1,\ldots,X_s) & = \min \{N,\virdim j(X_1,\ldots,X_s)\}.
    \end{align*}
    In particular, if $X \subseteq \bbP^N$ is an algebraic variety, 
    \begin{align*}
        \virdim \sigma_s(X) & = s\dim(X) + s-1; \\ 
        \expdim \sigma_s(X) & = \min \{N,\virdim \sigma_r(X)\}.
    \end{align*}
By \ref{geometrySecants-lemma-expecteddimension} the expected dimension is always an upper bound for the dimension of a secant variety. We say that the algebraic variety $X$ is \emph{$s$-defective} if $\dim \sigma_s(X) < \expdim\sigma_s(X)$. 
\end{definition}

%%%%%%%%%%%%%%%%%%%%%%%%%%%%%%%%%%%%
\section{First geometric properties}
\label{geometrySecants-section-first_properties}

We collect here the first properties on secant varieties. 

The following lemmas were all classically know, see \cite{Pal09}. For a recent reference, we refer to \cite{Rus16}.

First of all, we observe that linear spaces are the \emph{idempotent} varieties in terms of secant varieties. 

\begin{lemma}
    \label{geometrySecants-lemma-palatini_1}
    Let $X\subset \bbP^N$ be an algebraic variety. If $\sigma_{r+1}(X) = \sigma_r(X)$, then $\sigma_r(X)$ is a projective linear space. 
\end{lemma}

\begin{lemma}
    \label{geometrySecants-lemma-palatini_2}
    Let $X\subset \bbP^N$ be an algebraic variety. If $\dim\sigma_{r+1}(X) = \dim\sigma_r(X)+1$, then $\sigma_{r+1}(X)$ is a projective linear space. 
\end{lemma}

An immediate but meaningful corollary is the fact that, if $X$ is \emph{non-degenerate}, i.e., it is not contained in any proper linear subspace, then the secant varieties of $X$ eventually fill the ambient space. 

\begin{lemma}
    \label{geometrySecants-lemma-secants_of_non_degenerate}
    Let $X\subset \bbP^N$ be a non-degenerate algebraic variety. Then, 
    \[
        X \subsetneq \sigma_2(X) \subsetneq \cdots \subsetneq \sigma_r(X) = \bbP^N.
    \]
\end{lemma}

Another important corollary is that \emph{curves are never defective.}
\begin{lemma}
    \label{geometrySecants-lemma-palatini_curves}
    Let $X\subset \bbP^N$ be an algebraic curve. Then, $\dim \sigma_s(X) = \min\{N, 2s+1\}$.
\end{lemma}





% Throughout more than a century, a great deal of research was devoted in the study of dimension of secant varieties, and in the classification of defective varieties. 

% A very first result in this direction is \ref{geometrySecants-lemma-palatini} which appears already in \cite{Pal09}. 

% \begin{lemma}[Palatini's Lemma]
% \label{geometrySecants-lemma-palatini}
% Let $X \subseteq \bbP^N$ be an algebraic variety with $\codim X \geq 2$ and let $C \subseteq \bbP^N$ be a linearly non-degenerate algebraic curve. Then either $\dim \bbJ(X,C) = \dim X + 2$. In particular, all secant varieties of algebraic curves have the expected dimension.
% \end{lemma}

%%%%%%%%%%%%%%%%%%%%%%%%%%%%%%%%%%%%
\section{Terracini's Lemma}
\label{geometrySecants-section-Terracini}

A fundamental tool in the study of dimension of secant varieties is Terracini's Lemma, dating back to \cite{Ter11}.
\begin{lemma}[Terracini's Lemma]
\label{geometrySecants-lemma-terracini}
Let $X, Y \subseteq \mathbb{P}^N$ be algebraic varieties and let $\bbJ(X,Y)$ be their join. Let $z \in \bbJ(X,Y)$ be a smooth point such that $z \in \langle x , y \rangle$ for smooth points $x \in X$ and $y \in Y$. Then
\[
T_{z} \bbJ(X,Y) = \langle T_x X , T_y Y\rangle.
\]
In particular 
\[
\dim \sigma_r(X) = \dim \langle T_{x_1} X \vvirg T_{x_r} X\rangle
\]
for generic points $x_1 \vvirg x_r \in X$.
\end{lemma}
This result offers a \emph{dual} point of view to the problem of determining the dimension of secant varieties via interpolation of fat points. Let $X$ be an algebraic variety and let $\calL$ be a (very ample) line bundle on $X$. Let $\phi_\calL : X \to \bbP H^0(X, \calL)^\vee$ be the embedding defined by $\calL$. 

\begin{proposition}
 \label{geometrySecants-proposition-interpolationfatpoints}
 Let $X$ be an irreducible algebraic variety, $\calL$ a very ample line bundle on $X$. Then
 \[
 \dim \sigma_r( \phi_\calL(X)) = h^0(\calL)-1 - h^0( \calI_{Z} \otimes \calL)
 \]
 where $Z$ is the union of $r$ double points supported at generic points of $X$.
\end{proposition}
