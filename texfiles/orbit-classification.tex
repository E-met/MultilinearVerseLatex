This chapter collects orbit classification results for the action of a product of general linear groups on a tensor space. We list cases with \emph{finite} or \emph{tame} orbit structure.

To do: Explain finite and tame -- What is a good definition of tame in this setting?

The only tensor spaces with finitely many orbits are subspaces of $\bbC^2 \otimes \bbC^3 \otimes \bbC^6$ or its permutations. The list of the orbits, and a description of the degeneration poset, is given in \ref{RepTheory-orbitclassification-section-236}.

There are, instead, a few cases where the structure is tame:
\begin{itemize}
 \item $\GL_2 \times \GL_m \times \GL_n$ acting on $\bbC^2 \otimes \bbC^m \otimes \bbC^n$;
 \item $\GL_3 \times \GL_3 \times \GL_3$ acting on $\bbC^3 \otimes \bbC^3 \otimes \bbC^3$;
 \item $\GL_2 \times \GL_2 \times \GL_2 \times \GL_2$ acting on $\bbC^2 \otimes \bbC^2 \otimes \bbC^2 \otimes \bbC^2$.
\end{itemize}
The case of format $(2,m,n)$ dates back to Kronecker, see, e.g., \cite[Ch.XIII]{Gan59}. In general, orbit classification results can be achieved using classical representation theoretic methods \cite{Vin79}. More recent references include \cite{Nur00} for the case of format $(3,3,3)$ and \cite{CD12} for the case of format $(2,2,2,2)$.

In the setting of symmetric tensors, there are two cases with finite representation type:
\begin{itemize}
 \item $\GL_n$ acting on $S^2 \bbC^n$;
 \item $\GL_2$ acting on $S^3 \bbC^n$;
\end{itemize}
moreover, there are two cases with tame representation type:
\begin{itemize}
 \item $\GL_2$ acting on $S^4 \bbC^2$;
 \item $\GL_3$ acting on $S^3 \bbC^3$.
\end{itemize}

To do: Segre Veronese (and there should be none) and skew-symmetric cases.

Invariant theoretic aspects are discussed in \ref{RepTheory-chapter-invariantTheory}.

\section{Normal forms for format $(2,3,6)$}
\label{RepTheory-orbitclassification-section-236}

We follow the presentation of \cite[Sec. 10.3]{Lan12}. Fix bases 
\begin{align*}
&a_0,a_1 \text { of } \bbC^2, \\
&b_0\vvirg b_2 \text{ of } \bbC^3, \\ 
&c_0\vvirg c_5 \text{ of } \bbC^6. \\ 
\end{align*}
The action of $\GL_2 \times \GL_3 \times \GL_6$ on $\bbC^2 \otimes \bbC^3 \otimes \bbC^6$ has $28$ orbits. We list representatives for these orbits highlighting their multilinear ranks and recording the corresponding space of $3 \times 6$ matrices, that is the image of the flattening ${\bbC^2}^* \to \bbC^3 \otimes \bbC^6$; we only draw a relevant submatrix in the non-concise cases.

\begin{itemize}
 \item Multilinear ranks $(0,0,0)$: 
 \[
  T_0 = 0;
 \]
\item Multilinear ranks $(1,1,1)$:
\[
T_1 = a_0 \otimes b_0 \otimes c_0;
\]
\item Multilinear ranks $(1,2,2)$:
\[
T_2 = a_0 \otimes b_0 \otimes c_0 + a_0 \otimes b_1 \otimes c_1;
\]
\item Multilinear ranks $(2,1,2)$:
\[
T_3 = a_0 \otimes b_0 \otimes c_0 + a_1 \otimes b_0 \otimes c_1;
\]
\item Multilinear ranks $(2,2,1)$:
\[
T_4 = a_0 \otimes b_0 \otimes c_0 + a_1 \otimes b_1 \otimes c_0;
\]
\item Multilinear ranks $(2,2,2)$:
\begin{align*}
T_5 &= a_0 \otimes b_0 \otimes c_1 + a_0 \otimes b_1 \otimes c_0 + a_1 \otimes b_0 \otimes c_0, \\
T_6 &= a_0 \otimes b_0 \otimes c_0 + a_1 \otimes b_1 \otimes c_1;
\end{align*}
\item Multilinear ranks $(2,2,3)$:
\begin{align*}
T_7 &= a_0 \otimes b_0 \otimes c_0 + a_0 \otimes b_1 \otimes c_1 + a_1 \otimes b_0 \otimes c_2, \\
T_8 &= a_0 \otimes b_0 \otimes c_0 + a_1 \otimes b_1 \otimes c_1 + (a_0 + a_1) \otimes (b_0 + b_1) \otimes c_2;
\end{align*}
\item Multilinear ranks $(2,2,4)$:
\[
T_9 = a_0 \otimes b_0 \otimes c_0 + a_0 \otimes b_1 \otimes c_1 + a_1 \otimes b_0 \otimes c_2 + a_1 \otimes b_1 \otimes c_3;
\]
\item Multilinear ranks $(1,3,3)$:
\[
T_{10} = a_0 \otimes (b_0 \otimes c_0 + b_1 \otimes c_1 + b_2 \otimes c_2)
\]
\item Multilinear ranks $(2,3,2)$:
\begin{align*}
T_{11} &= a_0 \otimes b_0 \otimes c_0 + a_0 \otimes b_1 \otimes c_1 + a_1 \otimes b_2 \otimes c_0, \\
T_{12} &= a_0 \otimes b_0 \otimes c_0 + a_1 \otimes b_1 \otimes c_1 + (a_0 + a_1) \otimes b_2 \otimes (c_0 + c_1);
\end{align*}
\item Multilinear ranks $(2,3,3)$:
\begin{align*}
T_{13} &= a_0 \otimes (b_0 \otimes c_1 + b_1 \otimes c_0) + a_1 \otimes (b_0 \otimes c_2 + b_2 \otimes c_0), \\
T_{14} &= a_0 \otimes (b_0 \otimes c_0 +  b_1 \otimes c_1) + a_1 \otimes b_2 \otimes c_2,\\
T_{15} &= a_0 \otimes (b_0 \otimes c_0 +  b_1 \otimes c_1 + b_2 \otimes c_2) + a_1 \otimes b_0 \otimes c_1,\\
T_{16} &= a_0 \otimes (b_0 \otimes c_0 +  b_1 \otimes c_1 + b_2 \otimes c_2) + a_1 \otimes (b_0 \otimes c_1 + b_1 \otimes c_2),\\
T_{17} &= a_0 \otimes (b_0 \otimes c_0 +  b_1 \otimes c_1) + a_1 \otimes (b_2 \otimes c_2 + b_0 \otimes c_1 ),\\
T_{18} &= a_0 \otimes b_0 \otimes c_0 +  (a_0+a_1) \otimes b_1 \otimes c_1 + a_1 \otimes (b_2 \otimes c_2);
\end{align*}
\item Multilinear ranks $(2,3,4)$:
\begin{align*}
T_{19} &= a_0 \otimes (b_0 \otimes c_0 + b_1 \otimes c_1 + b_2 \otimes c_3) + a_1 \otimes (b_0 \otimes c_1 + b_1 \otimes c_2), \\
T_{20} &= a_0 \otimes (b_0 \otimes c_0 +  b_1 \otimes c_2 + b_2 \otimes c_3) + a_1 \otimes b_0 \otimes c_1,\\
T_{21} &= a_0 \otimes (b_0 \otimes c_0 +  b_1 \otimes c_2 + b_2 \otimes c_3) + a_1 \otimes (b_0 \otimes c_1 +b_1 \otimes c_3),\\
T_{22} &= a_0 \otimes (b_0 \otimes c_0 +  b_1 \otimes c_2) + a_1 \otimes (b_0 \otimes c_1 +b_2 \otimes c_3),\\
T_{23} &= a_0 \otimes (b_0 \otimes c_0 +  b_1 \otimes c_1 + b_2 \otimes c_2) + a_1 \otimes (b_0 \otimes c_1 + b_1 \otimes c_2 + b_2 \otimes c_3);\\
\end{align*}
\item Multilinear ranks $(2,3,5)$:
\begin{align*}
T_{24} &= a_0 \otimes (b_0 \otimes c_0 + b_1 \otimes c_1 + b_2 \otimes c_2) + a_1 \otimes (b_1 \otimes c_3 + b_2 \otimes c_4), \\
T_{25} &= a_0 \otimes (b_0 \otimes c_0 +  b_1 \otimes c_1 + b_2 \otimes c_2) + a_1 \otimes (b_0 \otimes c_2 + b_1 \otimes c_3 + b_2 \otimes c_4);\\
\end{align*}
\item Multilinear ranks $(2,3,6)$:
\[
T_{26} = a_0 \otimes (b_0 \otimes c_0 +  b_1 \otimes c_1 + b_2 \otimes c_2) + a_1 \otimes (b_0 \otimes c_3 + b_1 \otimes c_4 + b_2 \otimes c_5).
\]
\end{itemize}

% $$
% \xymatrix{
%  & & & 26 & & & & \\
%  & & 25  & & & & & \\
% } 
% $$
% 
% The degeneration poset is described in [Figure to be inserted]. The proof of the relevant restrictions and degenerations is given by describing the explicit restrictions and degenerations between the tensors of the classification. The linear maps are given in bases; vectors for which the image is not specified are mapped to themselves.
% \begin{itemize}
%  \item $T_{26}$ restricts to $T_{25}$ via
%  \[
%  \begin{array}{l}
%   c_3 \mapsto c_{2}\\
%   c_4 \mapsto c_{3}\\
%   c_5 \mapsto c_{4}
%  \end{array}
% \] 
%  \item $T_{25}$ degenerates to $T_{24}$ via
%  \[
%  \begin{array}{lll}
%   a_1 \mapsto \eps a_1 & ~ & c_3 \mapsto \eps^{-1} c_3 \\
%   & & c_4 \mapsto \eps^{-1} c_4
%  \end{array}
% \] 
% \item $T_{25}$ restricts to $T_{23}$ via 
%  \[
%  \begin{array}{lll}
% b_1 \mapsto b_2 & ~&  c_1 \mapsto c_{2} \\
% b_2 \mapsto b_1 & ~& c_2 \mapsto c_{1} \\
% & & c_4 \mapsto c_{2}
%   \end{array}
% \] 
% \end{itemize}

\subsection*{Symmetric and partially symmetric cases}

In this section we discuss orbit structure of the spaces $S^3 \bbC^2$ and $\bbS^2 \bbC^2 \otimes \bbC^3$, regarded as subspaces of $\bbC^2 \otimes \bbC^3 \otimes \bbC^6$.
