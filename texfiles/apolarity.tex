Classical apolarity theory is one of the most used methods in the study of symmetric tensor rank. Its origins lie in seminal work by Cayley \cite{Cay45} and Sylvester \cite{Syl52,Syl53} and in the Macaulay's work on inverse systems. In this chapter, we review the classical apolarity lemma for Waring decompositions, we discuss Sylvester's catalecticant method and we illustrate some of its consequences.

Given a vector space $V$, the symmetric algebra $\Sym V^*$ can be identified with the $\bbC[V]$ of polynomial functions on $V$, or equivalently the homogeneous coordinate ring of $\bbP V$; similarly, the symmetric algebra $\Sym V$ can be identified with the polynomial ring $\bbC[V^*]$. On the other hand, there is a natural action of $\Sym V^*$ on $\Sym V$ induced by the natural pairing of $V^*$ and $V$: more precisely, $\Sym V^*$ can be thought of as the ring of differential operators with constant coefficients on $\bbC[V]$. Apolarity theory is the study of the interplay of these ``two natures'' of the symmetric algebra $\Sym V^*$: as the coordinate ring on $\bbP V$ and as a ring of differential operators acting on $\Sym V^*$.

\section{The classical apolarity lemma}
\label{tensorRank-section-apolarityLemma}
% Author: Fulvio Gesmundo
We start by giving a formal definition of the action of the apolarity action of $\Sym V^*$ on $\Sym V$: 
\begin{definition}
\label{tensorRank-definition-apolarityAction}
% Author: Fulvio Gesmundo
Let $V$ be a vector space and let $x_0 \vvirg x_n$ be a basis of $V$. Let $\xi_0 \vvirg \xi_n$ be the dual basis of $V^*$. The {\it apolarity action} of $\Sym V^*$ on $\Sym V$ is defined by 
\[
\xi_i \cdot f = \frac{\partial}{\partial x_i}f
\]
and extended to be an algebra action of $\Sym V^*$. 
\end{definition}
It is easy to prove that the apolarity action does not depend on the fixed choice of basis. In fact, it is equivariant with respect to the action of $\GL V$ on $V$ and $V^*$. A fundamental object associated to en element $f \in \Sym V$ is the annihilator of $f$ with respect of the apolarity action:
\begin{definition}[Apolar ideal]
\label{tensorRank-definition-apolarIdeal}
% Author: Fulvio Gesmundo
Let $V$ be a vector space and let $f \in \Sym V$. The {\it apolar ideal of} $f$ is
\[
\Ann(f) = \{ D \in \Sym V^* : D (f) = 0 \}.
\]
\end{definition}
We record the following immediate properties:
\begin{lemma}
\label{tensorRank-definition-apolarIdealBasics}
% Author: Fulvio Gesmundo
Let $V$ be a vector space and let $f \in Sym V$. Then
\begin{itemize}
 \item $\Ann(f)$ is an ideal in $\Sym V^*$;
 \item if $\deg f = d$, then $S^e V^* \subseteq \Ann(f)$ for $e > d$;
 \item if $\deg f = d$, then the vector space
\[
\Ann(f)_{\leq d} + S^{\leq (d-1)}V^* \subseteq S^{\leq d} V^*
\]
has codimension $1$;
\item if $f$ is homogeneous, then $\Ann(f)$ is a homogeneous ideal.
\item If $f$ is homogeneous, for every $e \leq d$, we have $\Ann(f)_e = (\Flat_e(f) : S^e V^* \to S^{d-e} V)$, where $\Flat_e(f)$ is the $e$-th catalecticant map of $f$, in the sense of \ref{RepTheory-chapter-flattenings}.
\end{itemize}
\end{lemma}
It is natural to ask to what extent it is possible to reconstruct the polynomial $f$ from its apolar ideal $\Ann(f)$. This is completely understood and we record here the result following \cite[Lemma 3.33A]{IE78}.

\begin{proposition}
 Let $f,g \in \Sym V$ be polynomials of degree $d$ and let $f = \sum_0^d f_j$, $g = \sum_0^d g_j$ be their decomposition as sum of homogeneous components. The following are equivalent:
 \begin{itemize}
  \item $\Ann(f) = \Ann(g)$;
  \item There exists $\lambda \neq 0$ such that $f_d = \lambda g_d$, and $g$ is generated by the partial derivatives of $f$.
 \end{itemize}
\end{proposition}

The classical apolarity lemma relates the apolar ideal of a homogeneous polynomial to the existence of a Waring decomposition. Recall the definition of Waring decomposition and Waring rank from \ref{geometrySecants-definition-symmetric_tensor_rank}: given a homogeneous polynomial $f \in S^d V$ of degree $d$, a Waring decomposition of $f$ is an expression of $f$ of the form 
\[
f = \ell_1^d + \cdots + \ell_r^d
\]
for some $\ell_i \in V$, and the Waring rank $\rank_d(f)$ of $f$ is the smallest possible number of summands for which such a decomposition exists. Projectively, a Waring decomposition is a set of $r$ points $\bbX = \{ \ell_1 \vvirg \ell_r \} \subseteq \bbP V$ with the property that $f \in \langle \nu_d(\bbX) \rangle$, where $\nu_d$ is the Veronese embedding of \ref{introduction-definition-Veronese}. 

\begin{lemma}[Apolarity Lemma]
 \label{tensorRank-lemma-apolarityLemma}
 % Author: Fulvio Gesmundo
 Let $f \in \bbP S^d V$ and let $\bbX = \{ \ell_1 \vvirg \ell_r\}$ be a set of $r$ points in $\bbP V$. The following are equivalent:
 \begin{enumerate}[(i)]
  \item $f \in \langle \nu_d(\bbX) \rangle$;
  \item $I(\bbX) \subseteq \Ann(f)$.
 \end{enumerate}
 In particular $\rank_d(f)$ is the smallest $r$ such that $\Ann(f)$ contains the ideal of $r$ distinct points.
\end{lemma}



\section{The catalecticant method}
\label{tensorRank-section-catalecticant}
% Author: Fulvio Gesmundo
In this section, we discuss Sylvester's catalecticant method \cite{Syl52,IK99}, an algorithm to compute the ideal of a Waring decomposition which is the foundational method of many other more advanced algorithms, for instance the ones discussed in \cite{BCMT10,BBCM13,BT20,LMR23}.

